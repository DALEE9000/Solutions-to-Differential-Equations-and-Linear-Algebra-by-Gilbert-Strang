\documentclass[a4paper]{article}

\def\nbook {Differential Equations and Linear Algebra}
\def\nbookshort {DeqLA}

\input{../headerdeqla}
\hfuzz=50pt

\begin{document}
\maketitle

\newpage
\tableofcontents

\newpage
\section{\textsf{2 - Second Order Equations}}

\subsection{\textsf{2.1 - Second Derivatives in Science and Engineering}}

% QUESTION 1
\qs{2.1.1}{Find a cosine and a sine that solve \(d^{2}y / dt^{2} = -9y\). This is a second order equation so we expect \textit{two constants} \(C\) and \(D\) (from integrating twice):
\[
	\textbf{Simple harmonic motion} \quad y \!\left( t \right) = C \cos\left( \omega t \right) + D \sin\left( \omega t \right) 
\]

What is \(\omega\)? If the system starts from rest (this means \(dy/dt = 0\) at \(t = 0\)), which constant \(C\) or \(D\) will be zero?
}

Differentiating \(y \!\left( t \right) \) twice, we get a \(\omega^{2}\) term. Making the necessary substitutions, we can see that \(\omega^{2} = 9\), implying \(\omega = 3\). Thus we have \(y = \sin\left( 3t \right)\) and \(y = \cos\left( 3t \right)\). The constants \(C\) and \(D\) are determined by the initial conditions. Assuming the system starts at rest, we must have

\[
	\frac{d y}{d t} = - 3C \sin\left( 3t \right) + 3D \cos\left( 3t \right) = 0
\]
which implies that \(dy / dt_{t = 0} = 3D = 0\), thus \(D = 0\).

\vspace{12pt}

% QUESTION 2
\qs{2.1.2}{In Problem 1, which \(C\) and \(D\) will give the starting values \(y \!\left( 0 \right) = 0\) and \(y' \!\left( 0 \right) = 1\)?}

We have \(y \!\left( 0 \right) = C = 0\) and \(y' \!\left( 0 \right) = 3D = 1\), or \(D = 1/3\).

\newpage

% QUESTION 3
\qs{2.1.3}{Draw Figure 2.3 to show simple harmonic motion \(y = A \cos\left( \omega t - \alpha  \right)\) with phases \(\alpha = \pi / 3\) and \(\alpha = -\pi /2\).}

\begin{center}
\includegraphics[width=1.0\textwidth]{../chap2/sec2.1/chap2sec2.1ex3.eps}
\end{center}

% QUESTION 4
\qs{2.1.4}{Suppose the circle in Figure 2.4 has radius 3 and circular frequency \(f = 60\) Hertz. If the moving point starts at the angle \(-45^{\circ}\), find its \(x\)-coordinate \(A \cos\left( \omega t - \alpha  \right)\). The phase lag is \(\alpha = 45^{\circ}\). When does the point first hit the \(x\)-axis?}

The circular motion of the point is expressed by the sinusoidal
\[
	3 \cos\left( 120 \pi t - \frac{\pi }{4} \right)
\]
Note that since the frequency is \(f = 60\) Hertz, the angular frequency is \(2 \pi \cdot 60 = 120 \pi \) radians s\(^{-1}\). The point hits the \(x\)-axis when the argument of the cosine is zero, namely at
\[
	120 \pi t - \frac{\pi }{4} = 0 \quad \implies \quad t = \frac{1}{480} \text{s}
\]

% QUESTION 5
\qs{2.1.5}{If you drive at 60 miles per hour on a circular track with radius \(R = 3\) miles, what is the time \(T\) for one complete circuit? Your circular frequency is \(f = \) \underline{\qquad} and your angular frequency is \(\omega = \) \underline{\qquad} (with what units?). The period is \(T\).}

Using dimensional analysis, the period is
\[
	T = \frac{1 \text{ hr}}{60 \text{ mi}} \!\left( 3 \text{ mi} \right) = \frac{1}{20} \text{ hr}
\]
The circular frequency is
\[
	f = \frac{60 \text{ mi}}{\text{ hr}} \frac{1 \text{ hr}}{3600 \text{ s}} \frac{1 \text{ cycle}}{3 \text{ mi}} = \frac{1}{180} \text{ s}^{-1}
\]
with angular frequency
\[
	\omega = 2 \pi f = \frac{\pi }{90} \text{ rad s}^{-1}
\]

% QUESTION 6
\qs{2.1.6}{The total energy \(E\) in the oscillating spring-mass system is
\begin{alignat*}{1}
	E & = \textbf{kinetic} \text{ energy in mass} + \textbf{potential} \text{ energy in spring} \\
	  & = \frac{m}{2} \!\left( \frac{d y}{d t}  \right)^{2} + \frac{k}{2}y^{2}
\end{alignat*}
Compute \(E\) when \(y = C \cos\left( \omega t \right) + D \sin\left( \omega t \right)\). The energy is constant!}

Given \(y \!\left( t \right) \), we have first time-derivative
\[
	\frac{d y}{d t} = - \omega C \sin\left( \omega t \right) + \omega D \cos\left( \omega t \right)
\]
squaring gives us
\[
	\!\left( \frac{d y}{d t}  \right)^{2} = \omega^{2} \!\left( C^{2} \sin^{2}\left( \omega t \right) - 2 CD \sin\left( \omega t \right) \cos\left( \omega t \right) + D^{2}\cos^{2}\left( \omega t \right) \right)
\]
Lastly, the \(y^{2}\) term is
\[
	y^{2} = C^{2} \cos^{2}\left( \omega t \right) + 2CD \sin\left( \omega t \right)\cos\left( \omega t \right) + D^{2}\sin^{2}\left( \omega t \right)
\]
Combining our ingredients, with \(\omega = \sqrt{k / m}\), observe that energy \(E\) reduces to a constant:
\begin{alignat*}{1}
	E & = \frac{m}{2} \!\left( \frac{d y}{d t}  \right)^{2} + \frac{k}{2} y^{2} \\
	  & = \frac{m}{2} \!\left( \frac{k}{m} \right) \!\left( C^{2}\sin^{2}\left( \omega t \right) - 2CD \sin\left( \omega t \right) \cos\left( \omega t \right) + D^{2} \cos^{2}\left( \omega t \right)\right)  \\ 
	  & \qquad + \frac{k}{2} \!\left( C^{2} \cos^{2}\left( \omega t \right) + 2CD \sin\left( \omega t \right)\cos\left( \omega t \right) + D^{2}\sin^{2}\left( \omega t \right) \right) \\
	  & = C^{2} + D^{2}
\end{alignat*}

\newpage

% QUESTION 7
\qs{2.1.7}{Another way to show that the total energy \(E\) is constant:

Multiply \(\bm{my'' + ky = 0}\) by \(\bm{y'}\). Then integrate \(my'y''\) and \(kyy'\).}

Take the first term and integrate:
\[
	m \int y'y'' \, dt
\]
By integration by parts, let
\[
	u = \frac{d y}{d t}, \quad \frac{d u}{d t} = \frac{d^{2}y}{d t^{2}} \quad \implies \quad du = \frac{d^{2}y}{d t^{2}} dt 
\]
Then we can rewrite the first term as
\[
	m \int u \, du = \frac{m}{2}u^{2} + C = \frac{m}{2} \!\left( y' \right)^{2} + C 
\]
as for the second term:
\[
	k \int yy' \, dt = k \int y \frac{d y}{d t}  \, dt = k \int y \, dy = \frac{k}{2}y^{2} + C 
\]
Summing the two pieces (sans the constants) restores the original energy function:
\[
	E = \frac{m}{2} \!\left( \frac{d y}{d t}  \right)^{2} + \frac{k}{2}y^{2}
\]
and since the derivative of a constant is zero, it must be the case that \(E\) is a constant.

\vspace{12pt}

% QUESTION 8
\qs{2.1.8}{A \textbf{forced oscillation} has another term in the equation and \(A \cos\left( \omega t \right)\) in the solution:
\[
	\frac{d^{2}y}{dt^{2}} + 4y = F \cos\left( \omega t \right) \quad \text{has} \quad y = C \cos\left( 2 t \right) + D \sin\left( 2 t \right) + A \cos\left( \omega t \right)
\]
\begin{enumerate}[label=(\alph*)]
	\item Substitute \(y\) into the equation to see how \(C\) and \(D\) disappear (they give \(y_{n}\). Find the forced amplitude \(A\) in the particular solution \(y_{p} = A \cos\left( \omega t \right)\).
	\item In case \(\omega = 2\) (forcing frequency \(=\) natural frequency), what answer does your formula give for \(A\)? The solution formula for \(y\) breaks down in this case.
\end{enumerate}}

The second time-derivative is
\[
	\frac{d^{2}y}{dt^{2}} = -4C \cos\left( 2t \right) - 4D \sin\left( 2t \right) - \omega^{2}A \cos\left( \omega t \right)
\]
\begin{enumerate}[label=(\alph*)]
	\item We have
		\[
			\frac{d^{2}y}{dt^{2}} + 4y = \!\left( 4 - \omega^{2} \right) A \cos\left( \omega t \right) = F \cos\left( \omega t \right)
		\]
	implying \(A = \dfrac{F}{4 - \omega^{2}}\).
	\item When \(\omega = 2\), \(A\) is undefined. 
\end{enumerate}

% QUESTION 9
\qs{2.1.9}{Following Problem 8, write down the complete solution \(y_{n} + y_{p}\) to the equation
\[
	m \frac{d^{2}y}{dt^{2}} + ky = F \cos\left( \omega t \right) \quad \text{with} \quad \omega \neq \omega_{n} = \sqrt{k / m} \quad \text{(no resonance)}
\]
The answer \(y\) has free constants \(C\) and \(D\) to match \(y \!\left( 0 \right) \) and \(y'\!\left( 0 \right) \) (\(A\) \textit{is fixed} by \(F\).}

Per Problem 8, we have solution
\[
	y = \underbrace{C \cos\left( \sqrt{\frac{k}{m}} t \right) + D \sin\left( \sqrt{\frac{k}{m}} t \right)}_{y_{n}} + \underbrace{\frac{F}{k - m \omega^{2}} \cos\left( \omega t \right)}_{y_{p}}
\]

All this involves is dividing the equation through by \(m\), understanding that the angular frequency of the sinusoidals in the null solution is the square root of \(k / m\), and making the changes to our formula for \(A\) accordingly.

\vspace{12pt}

% QUESTION 10
\qs{2.1.10}{Suppose Newton's Law \(F = ma\) has the force \(F\) in the \textit{same} direction as \(a\):
\[
	my'' = + ky \quad \text{including} \quad y'' = 4y
\]
Find two possible choices of \(s\) in the exponential solutions \(y = e^{st}\). The solution is not sinusoidal and \(s\) is real and the oscillations are gone. Now \(y\) is unstable.}

Substituting in \(y\), we find
\[
	ms^{2}e^{st} = ke^{st}
\]
This forces
\[
	s = \pm \sqrt{\frac{k}{m}}
\]
\newpage

% QUESTION 11
\qs{2.1.11}{Here is a \textit{fourth} order equation: \(d^{4}y / dt^{4} = 16y\). Find \textit{four} values of \(s\) that give exponential solutions \(y = e^{st}\). You could expect four initial conditions on \(y\): \(y \!\left( 0 \right) \) is given along with what three other conditions?}

Equivalently, we find the four complex roots of \(16\):
\[
	s^{4} = 16
\]
which are \(s = \pm 2, \pm 2i\).

\vspace{12pt}

% QUESTION 12
\qs{2.1.12}{To find a particular solution to \(y'' + 9y = e^{ct}\), I would look for a multiple \(y_{p}\!\left( t \right) = Ye^{ct}\) of the forcing function. What is that number \(Y\)? When does your formula give \(Y = \infty\)? (Resonance needs a new formula for \(Y\).)}

Let \(y_{p} = Ye^{ct}\). Substituting, we find
\[
	Yc^{2} e^{ct} + 9Ye^{ct} = e^{ct}
\]
Solving for \(Y\) yields \(Y = \dfrac{1}{c^{2} + 9}\). When \(c \rightarrow \pm3\), we have resonance, and \(Y \rightarrow \infty\).

\vspace{12pt}

% QUESTION 13
\qs{2.1.13}{In a particular solution \(y = Ae^{i \omega t}\) to \(y'' + 9y = e^{i \omega t}\), what is the amplitude \(A\)? The formula blows up when the forcing frequency \(\omega =\) what natural frequency?}

Substituting, we derive
\[
	-\omega^{2}A e^{i \omega t} + 9A e^{i \omega t} = e^{i \omega t}
\]
which gives us \(A = \dfrac{1}{9 - \omega^{2}}\). Resonance is when \(\omega = 3\).

\vspace{12pt}

\newpage
% QUESTION 14
\qs{2.1.14}{Equation (10) says that the tangent of the phase angle is \(\tan\left( \alpha  \right) = y' \!\left( 0 \right) / \omega y \!\left( 0 \right) \). First, check that \(\tan\left( \alpha  \right)\) is dimensionless when \(y\) is in meters and time is in seconds. Next, if that ratio is \(\tan\left( \alpha  \right) = 1\), should you choose \(\alpha = \pi / 4\) or \(\alpha = 5 \pi / 4\)? Answer:
\[
	\text{Separately you want } R \cos \left( \alpha  \right) = y \!\left( 0 \right)  \text{ and } R \sin \left( \alpha  \right) = y' \!\left( 0 \right) / \omega 
\]
If those right hand sides are positive, choose the angle \(\alpha \) between \(0\) and \(\pi / 2\). 

If those right hand sides are negative, add \(\pi \) and choose \(\alpha = 5 \pi / 4\).

\vspace{12pt}

\textit{Question}: If \(y \!\left( 0 \right) > 0\) and \(y' \!\left( 0 \right) < 0\), does \(\alpha \) fall between \(\pi / 2\) and \(\pi \) or between \(3 \pi / 2\) and \(2 \pi \)? If you plot the vector from \( \!\left( 0, 0 \right) \) to \(\!\left( y \!\left( 0 \right) , y' \!\left( 0 \right) / \omega  \right) \), its angle is \(\alpha \).}

As \(y \!\left( 0 \right) > 0\) and \(y'\!\left( 0 \right) < 0\) requires positive cosine and negative sine, \(\alpha \) falls between \(3 \pi / 2\) and \(2 \pi \).

\vspace{12pt}

% QUESTION 15
\qs{2.1.15}{Find a point on the sine curve in Figure 2.1 where \(y > 0\) but \(v = y' < 0\) and also \(a = y'' < 0\). The curve is sloping down and bending down.

\vspace{12pt}

Find a point where \(y < 0\) but \(y' > 0\) and \(y'' > 0\). The point is below the \(x\)-axis but the curve is sloping \underline{\qquad} and bending \underline{\qquad}.}

One area corresponding to the first set of conditions is \(\pi / 2 < t < \pi \). As for the second, we have \(3 \pi / 2 < t < 2 \pi \).

\vspace{12pt}

% QUESTION 16
\qs{2.1.16}{
\begin{enumerate}[label=(\alph*)]
	\item Solve \(y'' + 100y = 0\) starting from \(y \!\left( 0 \right) = 1\) and \(y' \!\left( 0 \right) = 10\). (\( \textbf{This is } \bm{y_{n}} \).)
	\item Solve \(y'' + 100y = \cos\left( \omega t \right)\) with \(y \!\left( 0 \right) = 0\) and \(y' \!\left( 0 \right) = 0\). (\( \textbf{This can be }\bm{y_{p}} \).)
\end{enumerate}}

\begin{enumerate}[label=(\alph*)]
	\item Let \(y = c_{1} \cos\left( 10 t \right) + c_{2} \sin\left( 10 t \right)\). Then \(y \!\left( 0 \right) = c_{1} = 1\) and \(y' \!\left( 0 \right) = 10 c_{2} = 10\) implies \(c_{2} = 1\). Then the null solution is
	\[
		y_{n} = \cos\left( 10 t \right) + \sin\left( 10 t \right)
	\]
\item Let \(y_{p} = R \cos\left( \omega t \right)\). Substitute to find
	\[
		-\omega^{2} R \cos\left( \omega t \right) + 100 R \cos\left( \omega t \right) = \cos\left( \omega t \right)
	\]
	Isolate \(R\) to derive
	\[
		R = \frac{1}{100 - \omega^{2}}
	\]
	The solution to this set of initial conditions requires \(y \!\left( 0 \right) = 0 \) and \(y' \!\left( 0 \right) = 0\). Begin with
	\[
		y \!\left( t \right) = c_{1} \cos\left( 10 t \right) + c_{2} \sin\left( 10 t \right) + \frac{1}{100 - \omega^{2}} \cos\left( \omega t \right)	
	\]
	From the first condition, we have \(c_{1} = -\dfrac{1}{100 - \omega^{2}}\). From the second, we have \(c_{2} = 0\). The full solution is
	\[
		y \!\left( t \right) = \frac{1}{100 - \omega^{2}} \!\left( \cos\left( \omega t \right) - \cos\left( 10t \right) \right) 
	\]
\end{enumerate}

% QUESTION 17
\qs{2.1.17}{Find a particular solution \(y_{p} = R \cos\left( \omega t - \alpha  \right)\) to \(y'' + 100y = \cos\left( \omega t \right) - \sin\left( \omega t \right)\).}

Substituting \(y_{p}\) gives us

\begin{alignat*}{2}
	& -\omega^{2} R \cos\left( \omega t - \alpha  \right) + 100 R \cos\left( \omega t - \alpha  \right) \\
		 & \qquad = \!\left( 100R - \omega^{2}R \right) \!\left[ \cos\left( \omega t \right)\cos\left( \alpha  \right) + \sin\left( \omega t \right)\sin\left( \alpha  \right) \right]  \\
\end{alignat*}

This implies that
\begin{alignat*}{1}
	R \cos\left( \alpha  \right) \!\left( 100 - \omega^{2} \right) & = 1 \\
	R \sin\left( \alpha  \right) \!\left( 100 - \omega^{2} \right) & = -1 \\ 
\end{alignat*}
enabling us to conclude that \(\alpha = 7 \pi / 4\), and amplitude
\[
	R = \frac{\sqrt{2}}{100 - \omega^{2}}
\]
Ergo, the particular solution is
\[
	y_{p} = \frac{\sqrt{2}}{100 - \omega^{2}} \cos\left( \omega t - \frac{7 \pi }{4} \right)
\]

% QUESTION 18
\qs{2.1.18}{Simple harmonic motion also comes from a linear pendulum (like a grandfather clock). At time \(t\), the height is \(A \cos\left( \omega t \right)\). What is the frequency \(\omega \) if the pendulum comes back to the start after 1 second? The period does not depend on the amplitude (a large clock or a small metronome or the movement in a watch can all have \(T = 1\)).}

The angular frequency is \(2 \pi \cdot f = 2 \pi \) radians s\(^{-1}\).

\vspace{12pt}

% QUESTION 19
\qs{2.1.19}{If the phase lag is \(\alpha \), what is the time lag in graphing \(\cos\left( \omega t - \alpha \right)\)?}

Put differently, we want to find the value of \(t'\) such that we are able to restore \(\omega t\) as the cosine's argument. If we have
\[
	t' = t + \alpha / \omega 
\]
then we get
\[
	\cos\left( \omega t' - \alpha  \right) = \cos\left( \omega \!\left( t + \alpha / \omega  \right) - \alpha  \right) = \cos\left( \omega t \right)
\]
Thus the time lag term is \(\alpha / \omega \).

\vspace{12pt}

% QUESTION 20
\qs{2.1.20}{What is the response \(y \!\left( t \right) \) to a delayed impulse if \(my'' + ky = \delta \!\left( t - T \right) \)?}

The full solution will be a factor of the step function, given by:
\[
	y \!\left( t \right) = \int^{t - T}_{0} \frac{\sin\left( \omega_{n} \!\left( t - T - s \right)  \right)}{m \omega_{n}} \delta \!\left( s \right)  \, ds = \frac{\sin\left( \omega_{n} \!\left( t - T \right)  \right)}{m \omega_{n}} H \!\left( t - T \right) 
\]
Intuitively, when \(t \le T\), the right-hand side vanishes. No impulse is imparted, and thus there is no response. But when we are at time \(t \ge T\) -- after the threshold -- the response kicks in.
\vspace{12pt}

% QUESTION 21
\qs{2.1.21}{(Good challenge) Show that \(\displaystyle{y = \int^{t}_{0} g \!\left( t - s \right) f \!\left( s \right)  \, ds }\) has \(my'' + ky = f \!\left( t \right) \).

\begin{enumerate}[label=\arabic*.]
	\item Why is \(\displaystyle{y' = \int^{t}_{0} g'\!\left( t - s \right) f \!\left( s \right)  \, ds + g \!\left( 0 \right) f \!\left( t \right)  }\)? Notice the two \(t\)'s in \(y\).
	\item Using \(g \!\left( 0 \right) = 0\), explain why \(\displaystyle{y'' = \int^{t}_{0} g''\!\left( t - s \right) f \!\left( s \right)  \, ds + g'\!\left( 0 \right) f \!\left( t \right)  }\).
	\item Now use \(g'\!\left( 0 \right) = 1 / m\) and \(mg'' + kg = 0\) to confirm \(my'' + ky = f \!\left( t \right) \).
\end{enumerate}}

Use the Leibniz integral rule:
\[
	\frac{d }{d t} \!\left( \int^{t}_{0} g \!\left( t - s \right) f \!\left( s \right)  \, ds \right) = g\!\left( 0 \right) f \!\left( t \right) + \int^{t}_{0} \frac{\partial }{\partial t} g \!\left( t - s \right) f \!\left( s \right)  \, ds 
\]

\begin{enumerate}[label=(\arabic*)]
	\item By above, applying the partial derivative inside the second term yields the desired result.
	\item One more application of the Leibniz rule gives us
	\[
		y'' = \int^{t}_{0} g''\!\left( t - s \right) f \!\left( s \right)  \, ds + g \!\left( 0 \right) f'\!\left( t \right) + g'\!\left( 0 \right) f \!\left( t \right) 
	\]
	With the premise \(g \!\left( 0 \right) = 0\), the second term vanishes and gives us the expected result.
	\item Derive
	\[
		my'' + ky = \int^{t}_{0} \!\left[ mg''\!\left( t - s \right) + kg \!\left( t - s \right)  \right] f \!\left( s \right)  \, ds + f \!\left( t \right) = f \!\left( t \right)   \\
	\]
	where the last equality follows by appealing to the nullity of \(g \!\left( t \right) \).
\end{enumerate}

% QUESTION 22
\qs{2.1.22}{With \(f = 1\) (direct current has \(\omega = 0\)) verify that \(my'' + ky = 1\) for this \(y\):
\[
	\textbf{Step response}
\]
\[
	y \!\left( t \right) = \int^{t}_{0} \frac{\sin\left( \omega_{n} \!\left( t - s \right)  \right)}{m \omega_{n}} \cdot 1 \, ds = y_{p} + y_{n} = \pmb{\frac{1}{k} - \frac{1}{k} \cos\left( \omega_{n}t \right)}
\]}

We have second derivative
\[
	y''\!\left( t \right) = \frac{\omega_{n}^{2}}{k}\cos\left( \omega_{n}t \right)	
\]
Since \(\omega_{n} = \sqrt{k / m}\), \(my''\!\left( t \right) = \cos\left( \omega_{n}t \right)\). Ergo, we have
\[
	my'' + ky = \cos\left( \omega_{n}t \right) + k \!\left[ \frac{1}{k} - \frac{1}{k}\cos\left( \omega_{n}t \right) \right] = 1
\]

% QUESTION 23
\qs{2.1.23}{(Recommended) For the equation \(d^{2}y / dt^{2} = 0\) find the null solution. Then for \(d^{2}g / dt^{2} = \delta \!\left( t \right) \) find the fundamental solution (start the null solution with \(g \!\left( 0 \right) = 0\) and \(g'\!\left( 0 \right) = 1\)). For \(y'' = f \!\left( t \right) \) find the particular solution using formula (16).}

Integrating twice, the null solution is
\[
	y \!\left( t \right) = C_{1} t + C_{2}
\]
To find the fundamental solution \(g \!\left( t \right) \), imposing the initial conditions \(g \!\left( 0 \right) = 0\) and \(g'\!\left( 0 \right) = 1\) forces \(C_{1} = 1\) and \(C_{2} = 0\), thus
\[
	g \!\left( t \right) = t
\]
Lastly, if \(y'' = f \!\left( t \right) \), then
\[
	y_{p}\!\left( t \right) = \int^{t}_{0} \!\left( t - s \right) f \!\left( s \right)  \, ds 
\]
% QUESTION 24
\qs{2.1.24}{For the equation \(d^{2}y/ dt^{2} = e^{i \omega t}\) find a particular solution \(y = Y \!\left( \omega  \right) e^{i \omega t}\). Then \(Y \!\left( \omega  \right) \) is the frequency response. Note the ``resonance'' when \(\omega = 0\) with the null solution \(y_{n} = 1\).}

One particular solution is
\[
	y \!\left( t \right) = -\frac{e^{i \omega t}}{\omega^{2}}
\]
meaning that \(Y \!\left( \omega  \right) = - 1 / \omega^{2}\). When \(\omega = 0\), we have resonance, and this particular solution breaks down.

\vspace{12pt}

% QUESTION 25
\qs{2.1.25}{Find a particular solution \(Ye^{i \omega t}\) to \(my'' - ky = e^{i \omega t}\). The equation has \(-ky\) instead of \(ky\). What is the frequency response \(Y \!\left( \omega  \right) \)? For which \(\omega \) is \(Y\) infinite?}

We have
\[
	\frac{d^{2} }{d t^{2}} \!\left( Y e^{i \omega t} \right) = - Y \omega^{2} e^{i \omega t}
\]
Substituting and canceling the \(e^{i \omega t}\) terms, we get
\[
	-mY \omega^{2} - kY = 1
\]
Implying
\[
	Y \!\left( \omega  \right) = -\frac{1}{m \omega^{2} + k}
\]
If we have
\[
	\omega = i \sqrt{\frac{k}{m}}
\]
then \(Y \!\left( \omega  \right) \) diverges to infinity -- meaning all real frequencies will not lead to this!

\newpage

\subsection{\textsf{2.2 - Key Facts About Complex Numbers}}

% QUESTION 1
\qs{2.2.1}{Mark the numbers \(s_{1} = 2 + i\) and \(s_{2} = 1 - 2i\) as points in the complex plane. (The plane has a real axis and an imaginary axis.) Then mark the sum \(s_{1} + s_{2}\) and the difference \(s_{1} - s_{2}\).}

Arithmetic involving complex numbers simply is vector superposition:

\begin{center}
\includegraphics[width=1.0\textwidth]{../chap2/sec2.2/chap2sec2.2ex1.eps}
\end{center}

\vspace{12pt}

% QUESTION 2
\qs{2.2.2}{Multiply \(s_{1} = 2 + i\) times \(s_{2} = 1 - 2i\). Check absolute values: \(\abs{s_{1}} \abs{s_{2}} = \abs{s_{1}s_{2}}\).}

We have
\[
	\!\left( 2 + i \right) \!\left( 1 - 2i \right) = 2 - 4i + i + 2 = 4 - 3i
\]
with \(\abs{s_{1}} = \abs{s_{2}} = \sqrt{5}\), \(\abs{s_{1}s_{2}} = \sqrt{16 + 9} = 5\), ascertaining \(\abs{s_{1}}\abs{s_{2}} = \abs{s_{1}s_{2}}\).

\vspace{12pt}

% QUESTION 3
\qs{2.2.3}{Find the real and imaginary parts of \(1 / \!\left( 2 + i \right) \). 

Multiply by \(\!\left( 2 - i \right) / \!\left( 2 - i \right) \):
\[
	\frac{1}{2 + i} \frac{2 - i}{2 - i} = \frac{2 - i}{\abs{2 + i}^{2}} = \quad ?
\]
}

The denominator is \(5\), so we have
\[
	\underbrace{\frac{2}{5}}_{\text{real}} - \underbrace{\frac{i}{5}}_{\text{imaginary}}
\]

% QUESTION 4
\qs{2.2.4}{\textit{Triple angles} \quad Multiply equation (10) by another \(e^{i \theta } = \cos\left( \theta  \right) + i \sin\left( \theta  \right)\) to find formulas for \(\cos\left( 3 \theta  \right)\) and \(\sin\left( 3 \theta  \right)\).}

Derive
\begin{alignat*}{1}
	\!\left( \cos\left( \theta  \right) + i \sin\left( \theta  \right) \right)^{3} & = \!\left( \cos^{2}\left( \theta  \right) - \sin^{2}\left( \theta  \right) + 2i \cos\left( \theta  \right) \sin\left( \theta  \right) \right) \!\left( \cos\left( \theta  \right) + i \sin\left( \theta  \right) \right)  \\
	 & = \cos^{3}\left( \theta  \right) - \sin^{2}\left( \theta  \right) \cos\left( \theta  \right) + 2i \cos^{2}\left( \theta  \right) \sin\left( \theta  \right) \\ 
	 & \qquad + i \cos^{2}\left( \theta  \right) \sin\left( \theta  \right) - i \sin^{3}\left( \theta  \right) - 2 \cos\left( \theta  \right) \sin^{2}\left( \theta  \right) \\
	 & = \underbrace{\cos^{3}\left( \theta  \right) - 3 \sin^{2}\left( \theta  \right) \cos\left( \theta  \right) }_{\text{real}} + i \underbrace{\!\left[ 3 \cos^{2}\left( \theta  \right) \sin\left( \theta  \right) - \sin^{3}\left( \theta  \right) \right] }_{\text{imaginary}}
\end{alignat*}

% QUESTION 5
\qs{2.2.5}{\textit{Addition formulas} Multiply 
\(e^{i \theta } = \cos\left( \theta  \right) + i \sin\left( \theta  \right)\) times \(e^{i \phi } = \cos\left( \phi  \right) + i \sin\left( \phi  \right)\) to get \(e^{i \!\left( \theta + \phi  \right) }\). Its real part is \(\cos\left( \theta + \phi  \right) = \cos\left( \theta  \right) \cos\left( \phi  \right) - \sin\left( \theta  \right) \sin\left( \phi  \right)\). What is its imaginary part \(\sin\left( \theta + \phi  \right)\)?}

Derive
\begin{alignat*}{1}
	\!\left[ \cos\left( \theta  \right) + i \sin\left( \theta  \right) \right] \!\left[ \cos\left( \phi  \right) + i \sin\left( \phi  \right) \right]  & = \underbrace{\cos\left( \theta  \right)\cos\left( \phi  \right) - \sin\left( \theta  \right)\sin\left( \phi  \right)}_{\text{real}} \\
																			   & \qquad + i \underbrace{\!\left[ \cos\left( \theta  \right) \sin\left( \phi  \right) + \sin\left( \theta  \right) \cos\left( \phi  \right) \right]}_{\text{imaginary}}  \\
\end{alignat*}

% QUESTION 6
\qs{2.2.6}{Find the real part and the imaginary part of each cube root of 1. Show directly that the three roots add to zero, as equation (11) predicts.}

The cube roots, in polar form, are given by
\begin{alignat*}{1}
	e^{i 2 \pi / 3} & = -\frac{1}{2} + i \frac{\sqrt{3}}{2} \\
	e^{i 4 \pi / 3} & = -\frac{1}{2} - i \frac{\sqrt{3}}{2}\\
	 & \qquad 1 
\end{alignat*}

which vanish when summed.

\newpage

% QUESTION 7
\qs{2.2.7}{The three cube roots of 1 are \(z\) and \(z^{2}\) and 1, when \(z = e^{2 \pi i / 3}\). What are the three cube roots of 8 and the three cube roots of \(i\)? (The angle for \(i\) is \(90^{\circ}\) or \(\pi / 2\), so the angle for one of its cube roots will be \underline{\qquad}. The roots are spaced by \(120^{\circ}\).}

In polar form, we can express 8 as \(8e^{2 \pi i}\). Let \(z = e^{2 \pi i / 3}\). Its cube roots are then \(2z\), \(2z^{2}\), and \(2\). 

For \(i\), we have polar form \(e^{\pi i / 2}\). Then its cube roots are \(e^{\pi i / 6}\), \(e^{5 \pi i / 6}\), and \(e^{3 \pi i / 2}\). 

Argue by vector superposition that the cube roots in all cases sum to zero.

\vspace{12pt}

% QUESTION 8
\qs{2.2.8}{
	\begin{enumerate}[label=(\alph*)]
	\item The number \(i\) is equal to \(e^{\pi i / 2}\). Then its \(i^{\text{th}}\) power \(i^{i}\) comes out equal to a real number, using the fact that \(\!\left( e^{s} \right)^{t} = e^{st}\). What is that real number \(i^{i}\)?
	\item \(e^{i \pi / 2}\) is also equal to \(e^{5 \pi i / 2}\). Increasing the angle by \(2 \pi \) does not change \(e^{i \theta }\) - it comes around a full circle and back to \(i\). Then \(i^{i}\) has another real value \(\!\left( e^{5 \pi i / 2} \right)^{i} = e^{-5 \pi / 2}\). What are all the possible values of \(i^{i}\)?
\end{enumerate}}

\begin{enumerate}[label=(\alph*)]
	\item We have \(\!\left( e^{\pi i / 2} \right)^{i} = e^{-\pi / 2}\).
	\item All possible values of \(i^{i}\) are \(e^{\!\left( - \pi \pm 4n \pi  \right) / 2}\).
\end{enumerate}

% QUESTION 9
\qs{2.2.9}{The numbers \(s = 3 + i\) and \(\overline{s} = 3 - i \) are complex conjugates. Find their sum \(s + \overline{s} = -B\) and their product \(\!\left( s \right) \!\left( \overline{s} \right) = C\). Then show that \(s^{2} + Bs + C = 0\) and also \(\overline{s}^{2} + B \overline{s} + C = 0\). Those numbers \(s\) and \(\overline{s}\) are the two roots of the quadratic equation \(x^{2} + Bx + C = 0\).}

The sum is \(s + \overline{s} = 6\), so \(B = -6\). Their product is \(s \overline{s} = 10 = C\). Then we have
\[
	s^{2} + Bs + C = s^{2} - \!\left( s + \overline{s} \right) s + s \overline{s} = 0
\]
\[
	\overline{s}^{2} + B \overline{s} + C = \overline{s}^{2} - \!\left( s + \overline{s} \right) \overline{s} + s \overline{s} = 0
\]

% QUESTION 10
\qs{2.2.10}{The numbers \(s = a + i \omega \) and \(\overline{s} = a - i \omega \) are complex conjugates. Find their sum \(s + \overline{s} = -B\) and their product \(\!\left( s \right) \!\left( \overline{s} \right) = C\). Then show that \(s^{2} + Bs + C = 0\). The two solutions of \(x^{2} + Bx + C = 0\) are \(s\) and \(\overline{s}\).}

The sum is \(s + \overline{s} = 2a\), so \(B = -2a\). Their product is \(s \overline{s} = a^{2} + \omega^{2}\). By the same argument as the previous problem, the solutions of the given polynomial are \(s\) and \(\overline{s}\).

\vspace{12pt}

% QUESTION 11
\qs{2.2.11}{\begin{enumerate}[label=(\alph*)]
	\item Find the numbers \(\!\left( 1 + i \right)^{4} \) and \(\!\left( 1 + i \right)^{8}\).
	\item Find the polar form \(re^{i \theta }\) of \(\!\left( 1 + i \sqrt{3} \right) / \!\left( \sqrt{3} + i \right) \).
\end{enumerate}}

\begin{enumerate}[label=(\alph*)]
	\item We have \(1 + i = \sqrt{2} e^{i \pi / 4}\). Then
	\[
		\!\left( \sqrt{2} e^{i \pi / 4} \right)^{4} = 4 e^{i \pi } = -4
	\]
	Raising to the eighth power, we simply square our previous answer to get \(\!\left( 1 + i \right)^{8} = 16\).
	\item The numerator is \(2e^{i \pi / 3}\) and the denominator \(2 e^{i \pi / 6}\). The quotient is \(e^{i \pi / 6}\).
\end{enumerate}

% QUESTION 12
\qs{2.2.12}{The number \(z = e^{2 \pi i / n}\) solves \(z^{n} = 1\). The number \(Z = e^{2 \pi i / 2n}\) solves \(Z^{2n} = 1\). How is \(z\) related to \(Z\)? (This plays a big part in the Fast Fourier Transform.)}

They are related by \(z = Z^{2}\).

\vspace{12pt}

% QUESTION 13
\qs{2.2.13}{\begin{enumerate}[label=(\alph*)]
	\item If you know \(e^{i \theta }\) and \(e^{-i \theta }\), how can you find \(\sin\left( \theta  \right)\)?
	\item Find all angles \(\theta \) with \(e^{i \theta } = -1\) and all angles \(\phi \) with \(e^{i \phi } = -i\).
\end{enumerate}}

\begin{enumerate}[label=(\alph*)]
	\item One way to write \(\sin\left( \theta  \right)\) is
	\[
		\frac{e^{i \theta } - e^{-i \theta }}{2} = \sin\left( \theta  \right)
	\]
\item We have \(\theta = \pi \pm 2n \pi \) (which eliminates the imaginary term), and \(\phi = - \frac{\pi }{2} \pm 2n \pi \). 
\end{enumerate}

\newpage
% QUESTION 14
\qs{2.2.14}{Locate all these points on one complex plane:
\begin{enumerate}[label=(\alph*)]
	\item \(2 + i\)
	\item \( \!\left( 2 + i \right)^{2} \)
	\item \(\displaystyle{\frac{1}{2 + i}}\)
	\item \(\abs{2 + i}\)
\end{enumerate}}

\begin{enumerate}[label=(\alph*)]
	\item \(2 + i\)
	\item \(\!\left( 2 + i \right)^{2} = 3 + 4i\)
	\item \(\displaystyle{\frac{1}{2 + i}} = \frac{2 - i}{5}\)
	\item \(\abs{2 + i} = \sqrt{5}\)
\end{enumerate}

\begin{center}
\includegraphics[width=1.0\textwidth]{../chap2/sec2.2/chap2sec2.2ex14.eps}
\end{center}

\newpage
% QUESETION 15
\qs{2.2.15}{Find the absolute values \(r = \abs{z}\) of these four numbers. If \(\theta \) is the angle for \(6 + 8i\), what are the angles for these four numbers?
\begin{enumerate}[label=(\alph*)]
	\item \(6 - 8i\)
	\item \(\!\left( 6 - 8i \right)^{2}\)
	\item \(\displaystyle{\frac{1}{6 - 8i}}\)
	\item \(8i + 6\)
\end{enumerate}}

\begin{enumerate}[label=(\alph*)]
	\item \(r = 10\), \(\phi = -\theta \), in polar form \(z = 10e^{-i \theta }\)
	\item \(r = 100\), \(\phi = -2 \theta \), in polar form \(z = 100e^{-2 \theta }\)
	\item \(r = 1 / 10\), \(\phi = \theta \), in polar form \(\displaystyle{z = \frac{1}{10} e^{i \theta }}\)
	\item \(r = 10\), \(\phi = \theta \), the original number is unchanged
\end{enumerate}

% QUESTION 16
\qs{2.2.16}{What are the real and imaginary parts of \(e^{a + i \pi }\) and \(e^{a + i \omega }\)?}

The real parts are \(-e^{a}\) and \(e^{a} \cos\left( \omega  \right)\) and the imaginary parts are \(0\) and \(e^{a} \sin\left( \omega  \right)\).

\vspace{12pt}

% QUESTION 17
\qs{2.2.17}{\begin{enumerate}[label=(\alph*)]
	\item If \(\abs{s} = 2\) and \(\abs{z} = 3\), what are the absolute values of \(sz\) and \(s / z\)?
	\item Find upper and lower bounds in \(L \le \abs{s + z} \le U\). When does \(\abs{s + z} = U\)?
\end{enumerate}}

\begin{enumerate}[label=(\alph*)]
	\item \(\abs{sz} = \abs{s}\abs{z} = 6\), \(\abs{s / z} = \abs{s}/\abs{z} = 2/3\)
	\item The lower bound is achieved when \(s\) and \(z\) are antiparallel, meaning the modulus of their vector sum is at best \(L = \abs{s + z} = 1\). The upper bound is reached when they are parallel, so \(U = \abs{s + z} = 5\).
\end{enumerate}

\newpage
% QUESTION 18
\qs{2.2.18}{
\begin{enumerate}[label=(\alph*)]
	\item Where is the product \(\!\left( \sin\left( \theta  \right) + i \cos\left( \theta  \right) \right) \!\left( \cos\left( \theta  \right) + i \sin\left( \theta  \right) \right) \) in the complex plane?
	\item Find the absolute value  \(\abs{S}\) and the polar angle \(\phi \) for \(S = \pmb{\sin\left( \theta  \right) + i \cos\left( \theta  \right)}\).
\end{enumerate}
This is my favorite problem, because \(S\) combines \(\cos\left( \theta  \right)\) and \(\sin\left( \theta  \right)\) in a new way. To find \(\phi \), you could plot \(S\) or add angles in the multiplication of part (a).}

\begin{enumerate}[label=(\alph*)]
	\item Note that \(\sin\left( \theta  \right) = \displaystyle{\cos\left( \frac{\pi }{2} - \theta  \right)}\) and \(\cos\left( \theta  \right) = \displaystyle{\sin\left( \frac{\pi }{2} - \theta  \right)}\). Then the first term is equal to \(\displaystyle{\cos\left( \frac{\pi }{2} - \theta  \right) + i \sin\left( \frac{\pi }{2} - \theta  \right) = e^{i \!\left( \pi / 2 - \theta  \right) }}\). Then the product is
	\[
		e^{i \!\left( \pi / 2 - \theta  \right) } e^{i \theta } = e^{i \pi / 2} = i
	\]
	\item \(\abs{S} = 1\) and \(\displaystyle{\phi = \frac{\pi }{2} - \theta }\).
\end{enumerate}

% QUESTION 19
\qs{2.2.19}{Draw the spirals \(e^{\!\left( 1 - i \right)t}\) and \(e^{\!\left( 2 - 2i \right)t}\). Do those follow the same curves? Do they go clockwise or anticlockwise? When the first one reaches the negative \(x\)-axis, what is the time \(T\)? What point has the second one reached at that time?}

As \(t \rightarrow \infty\), the spirals progress clockwise. When the imaginary part vanishes, we have \(T = \pi \), and \(e^{\!\left( 1 - i \right) \pi } = e^{\pi }\). At that point in time, we also have \(e^{\!\left( 2 - 2i \right) \pi } = e^{2 \pi }\).

\begin{center}
\includegraphics[width=1.0\textwidth]{../chap2/sec2.2/chap2sec2.2ex19.eps}
\end{center}

% QUESTION 20
\qs{2.2.20}{The solution to \(d^{2}y / dt^{2} = -y\) is \(y = \cos\left( t \right)\) if the initial conditions are \(y \!\left( 0 \right) = \underline{\qquad}\) and \(y' \!\left( 0 \right) = \underline{\qquad}\). The solution is \(y = \sin\left( t \right)\) when \(y \!\left( 0 \right) = \underline{\qquad}\) and \(y'\!\left( 0 \right) = \underline{\qquad}\). Write each of those solutions in the form \(c_{1}e^{i t} + c_{2}e^{-i t}\), to see that real solutions can come from complex \(c_{1}\) and \(c_{2}\).}

For the case of \(y = \cos\left( t \right)\), the initial conditions are \(y \!\left( 0 \right) = 1\) and \(y'\!\left( 0 \right) = 0\). In the case of \(y  = \sin\left( t \right)\), we have \(y \!\left( 0 \right) = 0\) and \(y'\!\left( 0 \right) = 1\). In the given form, the initial conditions require

\underline{\(y = \cos\left( t \right)\)}
\begin{align*}
	y \!\left( 0 \right) & = c_{1} + c_{2} = 1 \\
	y' \!\left( 0 \right) & = i \!\left( c_{1} - c_{2} \right) = 0
\end{align*}
Implying \(c_{1} = c_{2} = 1/2\).

\underline{\( y = \sin\left( t \right) \)}
\begin{align*}
	y \!\left( 0 \right)  & = c_{1} + c_{2} = 0 \\
	y'\!\left( 0 \right)  & = i \!\left( c_{1} - c_{2} \right) = 1  
\end{align*}
Implying \(c_{1} = 1 / 2i\) and \(c_{2} = -1 / 2i\).

Thus we can write \(\cos\left( t \right)\) and \(\sin\left( t \right)\) as
\[
	\cos\left( t \right) = \frac{e^{i t} + e^{-i t}}{2}, \quad \sin\left( t \right) = \frac{e^{i t} - e^{-i t}}{2i}
\]

The general solution is given by
\[
	y \!\left( t \right) = C_{1} \cos\left( t \right) + C_{2} \sin\left( t \right)
\]
which we rewrite as
\begin{align*}
	y \!\left( t \right)  & = C_{1} \cos\left( t \right) + C_{2} \sin\left( t \right) \\
	 & = C_{1} \!\left[ \frac{e^{i t} + e^{-i t}}{2} \right] + C_{2} \!\left[ \frac{e^{i t} - e^{-i t}}{2i} \right]   \\ 
	 & = \!\left[ \frac{C_{1} - iC_{2}}{2} \right] e^{i t} + \!\left[ \frac{C_{1} + iC_{2}}{2} \right] e^{-i t}
\end{align*}

% QUESTION 21
\qs{2.2.21}{Suppose \(y \!\left( t \right) = e^{-t}e^{i t}\) solves \(y'' + By' + Cy = 0\). What are \(B\) and \(C\)? If this equation is solved by \(y = e^{3it}\), what are \(B\) and \(C\)?}

Note that \(y \!\left( t \right) = e^{-t}e^{i t} = e^{\!\left( -1 + i \right) t}\). Then we have
\[
	y'' + By' + Cy = \!\left( -1 + i \right)^{2} e^{\!\left( -1 + i \right) t} + B \!\left( -1 + i \right)  e^{\!\left( -1 + i \right) t} + Ce^{\!\left( -1 + i \right) t} = 0
\]
from which we derive
\[
	\!\left( -1 + i \right)^{2} + B \!\left( -1 + i \right) + C = 0
\]
Now wait -- this looks suspiciously familiar to the form seen in question 9:
\[
	s^{2} + Bs + C = 0
\]
Then we have that \(s + \overline{s} = -B\) and \(s \overline{s} = C\). Thus
\[
	s + \overline{s} = -2 = -B \qquad s \overline{s} = 2 = C
\]
Therefore, \(B = C = 2\), and the equation is \(y'' + 2y' + 2y = 0\).

In the case of \(y = e^{3it}\), we have polynomial
\[
	-9 + 3Bi + C = 0
\]
where it follows that \(s = 3i\), \(s + \overline{s} = 0 = -B\), and \(s \overline{s} = 9 = C\). The equation is \(y'' + 9y = 0\).


% QUESTION 22
\qs{2.2.22}{From the multiplication \(e^{iA}e^{-iB} = e^{i \!\left( A - B \right) }\), find the ``subtraction formulas'' for \(\cos\left( A - B \right)\) and \(\sin\left( A - B \right)\).}

Derive
\begin{align*}
	e^{iA}e^{-iB} & = \!\left[ \cos\left( A \right) + i \sin\left( A \right) \right] \!\left[ \cos\left( B \right) - i \sin\left( B \right) \right]  \\
	 & =  \!\left[ \cos\left( A \right)\cos\left( B \right) + \sin\left( A \right) \sin\left( B \right) \right] + i \!\left[ \sin\left( A \right) \cos\left( B \right) - \cos\left( A \right) \sin\left( B \right) \right] \\ 
	 & = \cos\left( A - B \right) + i \sin\left( A - B \right) \\
	 & = e^{i \!\left( A - B \right) }
\end{align*}

% QUESTION 23
\qs{2.2.23}{
\begin{enumerate}[label=(\alph*)]
	\item If \(r\) and \(R\) are the absolute values of \(s\) and \(S\), show that \(rR\) is the absolute value of \(sS\). (Hint: Polar form!)
	\item If \(\overline{s}\) and \(\overline{S}\) are the complex conjugates of \(s\) and \(S\), show that \(\overline{s}\overline{S}\) is the complex conjugate of \(sS\). (Polar form!)
\end{enumerate}}

\begin{enumerate}[label=(\alph*)]
	\item Write \(s = re^{i \theta }\) and \(S = Re^{i \phi }\). Then \(sS = rR e^{i \!\left( \theta + \phi  \right) }\). Then \(rR = \abs{sS}\).
	\item Using the aforementioned definitions of \(s\) and \(S\), we have 
	\[
		\overline{sS} = rRe^{-i \!\left( \theta + \phi  \right)} = \!\left( re^{-i \theta } \right) \!\left( Re^{-i \phi } \right) = \overline{s}\overline{S}
	\]
\end{enumerate}

\newpage
% QUESTION 24
\qs{2.2.24}{Suppose a complex number \(s\) solves a real equation \(s^{3} + As^{2} + Bs + C = 0\) (with \(A\), \(B\), \(C\) real). Why does the complex conjugate \(\overline{s}\) also solve this equation? ``\textit{Complex solutions to real equations come in conjugate pairs \(s\) and \(\overline{s}\)}.''}

We generalize the results of question 23 and claim that the complex conjugate of a product is equal to the product of the complex conjugates. For instance, suppose \(s = re^{i \theta }\). Then we prove \(\overline{s^{n}} = \overline{s}^{n}\):
\[
	\overline{s^{n}} = r^{n}e^{-in \theta } = \prod_{k=1}^{n} re^{-i \theta } = \overline{s}^{n}
\]
We can also prove that the complex conjugate of a sum is the sum of the complex conjugates. Define \(z_{k} = a_{k} + i b_{k}\) for \(k \in \!\left\{ 1, ..., n \right\} \). Then
\[
	\overline{\sum_{k=1}^{n} z_{k}} = \overline{\sum_{k=1}^{n} \!\left( a_{k} + ib_{k} \right)} = \overline{\sum_{k=1}^{n} a_{k} + i \sum_{k=1}^{n} b_{k}} = \sum_{k=1}^{n} a_{k} - i \sum_{k=1}^{n} b_{k} = \sum_{k=1}^{n} \overline{z_{k}}
\]
With these facts, the complex conjugate of the left side simply yields that \(\overline{s}\) is too a solution of the real equation.

\vspace{12pt}

% QUESTION 25
\qs{2.2.25}{
\begin{enumerate}[label=(\alph*)]
	\item If two complex numbers add to \(s + S = 6\) and multiply to \(sS = 10\), what are \(s\) and \(S\)? (They are complex conjugates.)
	\item If two numbers add to \(s + S = 6\) and multiply to \(sS = -16\), what are \(s\) and \(S\)? (Now they are real.)
\end{enumerate}}

\begin{enumerate}[label=(\alph*)]
	\item \(s = 3 + i\), \(S = 3 - i\)
	\item \(s = 8\), \(S = -2\)
\end{enumerate}

% QUESTION 26
\qs{2.2.26}{If two numbers \(s\) and \(S\) add to \(s + S = -B\) and multiply to \(sS = C\), show that \(s\) and \(S\) solve the quadratic equation \(s^{2} + Bs + C = 0\).}

This is an interesting question because we relax the condition that \(S = \overline{s}\). Substituting for \(B\) and \(C\), we have
\[
	s^{2} - \!\left( s + S \right) s + sS = 0, \qquad S^{2} - \!\left( s + S \right)S + sS = 0
\]
% QUESETION 27
\qs{2.2.27}{Find three solutions to \(s^{3} = -8i\) and plot the three points in the complex plane. What is the sum of the three solutions?}

In polar form, \(s^{3} = 8e^{-i \pi / 2}\). Its cube roots are \(2e^{-i \pi / 6}\), \(2e^{i \pi / 2}\), and \(2 e^{i 7 \pi / 6}\). By vector superposition, the roots sum to zero.

\begin{center}
\includegraphics[width=1.0\textwidth]{../chap2/sec2.2/chap2sec2.2ex27.eps}
\end{center}

\vspace{12pt}

% QUESTION 28
\qs{2.2.28}{
\begin{enumerate}[label=(\alph*)]
	\item For which complex numbers \(s = a + i \omega \) does \(e^{st}\) approach \(0\) as \(t \rightarrow \infty\)? Those numbers \(s\) fill which ``half-plane'' in the complex plane?
	\item For which complex numbers \(s = a + i \omega \) does \(s^{n}\) approach \(0\) as \(n \rightarrow \infty\)? Those numbers \(s\) fill which part of the complex plane? Not a half-plane!
\end{enumerate}}

\begin{enumerate}[label=(\alph*)]
	\item For \(a < 0\), \(e^{st} = e^{at}e^{i \omega t}\) approaches zero.
	\item In polar form, \(s = \sqrt{a^{2} + \omega^{2}} e^{i \tan^{-1}\left( \omega / a \right)}\). We must require the modulus to be less than zero, or \(\sqrt{a^{2} + \omega^{2}} < 0\), for \(s^{n}\) to tend towards zero as \(n\) increases.
\end{enumerate}

\newpage

\subsection{\textsf{2.3 - Constant Coefficients \texorpdfstring{\(\pmb{A}\), \(\pmb{B}\), \(\pmb{C}\)}{A, B, C}}}

% QUESTION 1
\qs{2.3.1}{Substitute \(y = e^{st}\) and solve the characteristic equation for \(s\):
\begin{enumerate}[label=(\alph*)]
	\item \(2y'' + 8y' + 6y = 0\)
	\item \(y'''' - 2y'' + y = 0\)
\end{enumerate}}

\begin{enumerate}[label=(\alph*)]
	\item The characteristic equation is \(2s^{2} + 8s + 6 = 0\) which factorizes into \(2 \!\left( s + 3 \right) \!\left( s + 1 \right) = 0\). Then \(s_{1} = -3\) and \(s_{2} = -1\).
	\item The characteristic equation is \(s^{4} - 2s^{2} + 1 = 0\) which factorizes into \(\!\left( s^{2} - 1 \right)^{2} = \!\left( s - 1 \right)^{2}\!\left( s + 1 \right)^{2}\). It has repeated solutions: \(s_{1}, s_{2} = 1\) and \(s_{3}, s_{4} = -1\). 
\end{enumerate}

% QUESTION 2
\qs{2.3.2}{Substitute \(y = e^{st}\) and solve the characteristic equation for \(s = a + i \omega \):
\begin{enumerate}[label=(\alph*)]
	\item \(y'' + 2y' + 5y = 0\)
	\item \(y'''' + 2y'' + y = 0\)
\end{enumerate}}

\begin{enumerate}[label=(\alph*)]
	\item The characteristic equation is \(s^{2} + 2s + 5 = 0\), which by the quadratic formula has roots
	\[
		s_{1}, s_{2} = \frac{-2 \pm \sqrt{-16}}{2} = -1 \pm 2i
	\]
	\item The characteristic equation is \(s^{4} + 2s^{2} + 1 = 0\), which by the quadratic formula has double root
	\[
		s^{2}= \frac{-2}{2} = -1
	\]
	Rooting once more, we find repeated roots
	\[
		s_{1}, s_{2} = i, \quad s_{3}, s_{4} = -i
	\]
\end{enumerate}

% QUESTION 3
\qs{2.3.3}{Which second order equation is solved by \(y = c_{1}e^{-2t} + c_{2}e^{-4t}\)? Or \(y = te^{5t}\)?}

In the first instance, the roots are \(s_{1} = -2\) and \(s_{2} = -4\). Then the characteristic polynomial is \(\!\left( s + 2 \right) \!\left( s + 4 \right) = s^{2} + 6s + 8\), corresponding to second order equation \(y'' + 6y' + 8y = 0\). In the second case, we have repeated roots \(s_{1}, s_{2} = 5\), which has characteristic polynomial \(\!\left( s - 5 \right)^{2} = s^{2} - 10s + 25\). This corresponds to second order equation \(y'' - 10y' + 25y = 0\).

\vspace{12pt}

% QUESTION 4
\qs{2.3.4}{Which second order equation has solutions \(y = c_{1}e^{-2t} \cos\left( 3t \right) + c_{2}e^{-2t}\sin\left( 3t \right)\)?}

The complex roots are \(s_{1} = -2 + 3i\) and \(s_{2} = -2 - 3i\), with characteristic polynomial \(s^{2} - 4s + 13\). This corresponds to second order equation 
\[
	y'' - 4y' + 13 = 0
\]

% QUESTION 5
\qs{2.3.5}{Which numbers \(B\) give (under)(critical)(over) damping in \(4y'' + By' + 16y = 0\)?}

\textbf{Underdamping}: We must have \(B^{2} < 4AC\), or in this case, \(B^{2} < 256\). Then we must have \(\abs{B} < 16\).

\textbf{Critical damping}: We have \(B^{2} = 4AC\), or \(B^{2} = 256\), implying \(\abs{B} = 16\).

\textbf{Overdamping}: We have \(B^{2} > 4AC\), or \(B^{2} > 256\), implying \(\abs{B} > 16\).

\vspace{12pt}

% QUESTION 6
\qs{2.3.6}{If you want oscillation from \(my'' + by' + ky = 0\), then \(b\) must stay below \underline{\qquad}.}

We must have \(b^{2} < 4mk\) or \(\abs{b} < 2 \sqrt{mk}\).

\vspace{12pt}

% QUESTION 7
\qs{2.3.7}{The roots \(s_{1}\) and \(s_{2}\) satisfy \(s_{1} + s_{2} = -2p = -B/A\) and \(s_{1}s_{2} = p^{2} + \omega^{2}_{n} = C/A\). Show this two ways:

\begin{enumerate}[label=(\alph*)]
	\item Start from \(A^{2} + Bs + C = A \!\left( s - s_{1} \right) \!\left( s - s_{2} \right) \). Multiply to see \(s_{1}s_{2}\) and \(s_{1} + s_{2}\).
	\item Start from \(s_{1} = -p + i \omega_{d}\), \(s_{2} = -p - i \omega_{d}\)
\end{enumerate}}

\begin{enumerate}[label=(\alph*)]
	\item We write
	\begin{align*}
		A^{2} + Bs + C & = A \!\left( s - s_{1} \right) \!\left( s - s_{2} \right)  \\
		 & = A \!\left( s^{2} - s \!\left( s_{1} + s_{2} \right) + s_{1}s_{2} \right)  
	\end{align*}
	Concluding that \(B = -A\!\left( s_{1} + s_{2} \right) \) and \(C = As_{1}s_{2}\). Then
	\[
		-\frac{B}{A} = s_{1} + s_{2}, \quad \frac{C}{A} = s_{1}s_{2}
	\]
	\item Let \(s_{1} = -p + i \omega_{d}\) and \(s_{2} = -p - i \omega_{d}\). Then \(s_{1} + s_{2} = -2p\) and
	\[
		s_{1}s_{2} = \!\left( -p + i \omega_{d} \right) \!\left( -p - i \omega_{d} \right) = p^{2} + \omega^{2}_{d}
	\]
\end{enumerate}
\textbf{Note}: There are two errors in the problem statement. The textbook erroneously has the equalities \(s_{1} + s_{2} = -B/2A\) and \(s_{1}s_{2} = p^{2} + \omega^{2}_{n}\). The corrections have been made in my writing of the problem above.

\vspace{12pt}

% QUESTION 8
\qs{2.3.8}{Find \(s\) and \(y\) at the bottom point of the graph of \(y = As^{2} + Bs + C\). At that minimum point \(s = s_{\text{min}}\) and \(y = y_{\text{min}}\), the slope is \(dy/ds = 0\).}

Differentiate with respect to \(s\) to find
\[
	\frac{d y}{d s} = 2As + B = 0 \quad \implies \quad s_{\text{min}} = -\frac{B}{2A}
\]
The corresponding \(y_{\text{min}}\) is
\begin{align*}
	y_{\text{min}} & = A \!\left( - \frac{B}{2A} \right)^{2} + B \!\left( -\frac{B}{2A} \right) + C \\
		   & = \frac{B^{2}}{4A} - \frac{B^{2}}{2A} + C \\
		   & = -\frac{B^{2}}{4A} + C
\end{align*}

% QUESTION 9
\qs{2.3.9}{The parabolas in Figure 2.10 show how the graph of \(y = As^{2} + Bs + C\) is raised by increasing \(B\). Using Problem 8, show that the bottom point of the graph moves left (change in \(s_{\text{min}}\)) and down (change in \(y_{\text{min}}\)) when \(B\) is increased by \(\Delta B\).}

Assume \(\Delta B > 0\). By Problem 8, the new \(s_{\text{min}}\) becomes
\[
	s_{\text{min}} = -\frac{\!\left( B + \Delta B \right) }{2A} < -\frac{B}{2A}
\]
and the new \(y_{\text{min}}\)
\[
	y_{\text{min}} = -\frac{\!\left( B + \Delta B \right)^{2}}{4A} + C = -\frac{\!\left( B^{2} + 2B \Delta B + \Delta B^{2} \right) }{4A} + C < -\frac{B^{2}}{4A} + C
\]
both of which correspond to a leftward and downward shift, respectively.

\vspace{12pt}

\newpage
% QUESTION 10
\qs{2.3.10}{(recommended) Draw a picture to show the paths of \(s_{1}\) and \(s_{2}\) when \(s^{2} + Bs + 1 = 0\) and the damping increases from \(B = 0\) to \(B = \infty\). At \(B = 0\), the roots are on the \underline{\quad} axis. As \(B\) increases, the roots travel on a circle (why?). At \(B = 2\), the roots meet on the real axis. For \(B > 2\) the roots separate to approach \(0\) and \(-\infty\). \textit{Why is their product \(s_{1}s_{2}\) always equal to \(1\)}?}

The roots are
\[
	s = \frac{-B \pm \sqrt{B^{2} - 4}}{2}
\]
When \(B = 0\), the roots are \(s_{1,2} = \pm i\), which lie on the imaginary axis. The roots travel on a circle as \( B  \) grows because they are complex conjugates, and given that the modulus of the roots remains constant, a quarter-circle traced out by the complex vector for one root will be reflected over the real axis for the other root, conforming to the geometry of a circle.

\begin{center}
\includegraphics[width=1.0\textwidth]{../chap2/sec2.3/chap2sec2.3ex10.pdf}
\end{center}

Recall from problem 7 that \(s_{1}s_{2} = C / A = 1\). Another, more intuitive approach is to realize that, \( s_{1} = \overline{s_{2}} \), so they are complex conjugates of one another, and their product is the modulus (in this case, unity).

% QUESTION 11
\qs{2.3.11}{(this too if possible) Draw the paths of \( s_{1} \) and \( s_{2} \) when \( s^{2} + 2s + k = 0\) and the stiffness increases from \( k = 0 \) to \( k = \infty \). When \( k = 0 \), the roots are \underline{\quad}. At \( k = 1 \), the roots meet at \( s =  \) \underline{\qquad}. For \( k \rightarrow \infty \) the two roots travel up/down on a \underline{\qquad} in the complex plane. \textit{Why is their sum \( s_{1} + s_{2} \) always equal to \( -2 \)}?}

The roots are
\[
	s = \frac{-2 \pm \sqrt{2^{2} - 4k}}{2} = -1 \pm \sqrt{1 - k}
\]
When \( k = 0 \), the roots are \( s_{1,2} = 0, -2 \). Once we reach \( k = 1 \), we have double roots with \( s_{1,2} = -1 \). As \( k \) increases to infinity, the real component remains fixed, while the imaginary component grows in both directions on the vertical axis. Altogether, the trajectory of the roots forms a cross.

\begin{center}
\includegraphics[width=1.0\textwidth]{../chap2/sec2.3/chap2sec2.3ex11.pdf}
\end{center}

As for their sum, observe that the imaginary components of \( s_{1,2} \) cancel, and the real parts add to \( -2 \).

\vspace{12pt}

% QUESTION 12
\qs{2.3.12}{If a polynomial \( P \!\left( s  \right) \) has a double root at \( s = s_{1} \), then \( \!\left( s - s_{1} \right) \) is a double factor and \( P \!\left( s  \right) = \!\left( s - s_{1} \right)^{2} Q \!\left( s  \right) \). Certainly \( P = 0 \) at \( s = s_{1} \). Show that also \( dP/ds = 0 \) at \( s = s_{1} \). use the product rule to find \( dP/ds  \).}

By the product rule, we have
\[
	\frac{d P }{d s }_{\!\left( s = 0 \right)} = 2 \!\left( s - s_{1} \right) Q \!\left( s  \right) + \!\left( s - s_{1} \right)^{2} \frac{d Q }{d s } = 0
\]

% QUESTION 13
\qs{2.3.13}{Show that \( y'' = 2ay' - \!\left( a^{2} + \omega^{2} \right)y \) leads to \( s = a \pm i \omega \). Solve \( y'' - 2y' + 10y = 0 \).}

The characteristic equation is given by
\[
	s^{2} - 2as + \!\left( \alpha^{2} + \omega^{2} \right) = 0 
\]
which has roots
\[
	s = \frac{2a \pm \sqrt{4a^{2} - 4 \!\left( \alpha^{2} + \omega^{2} \right)}}{2} = a \pm i \omega 
\]
For the given equation, we have \( a = 1 \) and \( \omega = 3 \). Then we have solution
\begin{alignat*}{1}
y \!\left( t  \right) & = e^{\!\left( 1 + 3i \right)t} + e^{\!\left( 1 - 3i \right)t} \\
	 & = 2e^{t} \cos\!\left( 3t \right)  \\ 
\end{alignat*}

% QUESTION 14
\qs{2.3.14}{The undamped \textit{natural frequency} is \( \omega_{n } = \sqrt{k / m } \). The two roots of \( ms^{2} + k = 0 \) are \( s = \pm i \omega_{n } \) (pure imaginary). With \( p = b / 2m \), the roots of \( ms^{2} + bs + k = 0 \) are \( \pmb{s_{1}, s_{2} = -p \pm \sqrt{p^{2} - \omega^{2}_{n }}} \). The coefficient \( p = b / 2m  \) has the units of 1 / time.

Solve \( s^{2} + 0.1s + 1 = 0 \) and \( s^{2} + 10s + 1 = 0 \) with numbers correct to two decimals.}

In the first equation, we have \( m = 1 \), \( b = 0.1 \), and \( k = 1 \). Then \( p = 0.05 \) and \( \omega_{n } = 1 \). This yields roots \( s_{1,2} = -0.05 \pm 0.99i \).

The second equation gives us \( p = 5 \) and \( \omega_{n } = 1 \), which has roots \( s_{1,2} = -5 \pm \sqrt{24} = -0.10, -9.89 \).

\vspace{12pt}

% QUESTION 15
\qs{2.3.15}{With large overdamping \( \pmb{p >> \omega_{n }} \), the square root \( \sqrt{p^{2} - \omega^{2}_{n }} \) is close to \( p - \omega^{2}_{n } / 2p \). Show that the roots of \( ms^{2} + bs + k \) are \( s_{1} \approx \pmb{-\omega^{2}_{n } / 2p} =\) (small) and \( s_{2} \approx -2p = \pmb{-b/m } \) (large).}

From question 14, the roots are \( s_{1}, s_{2} = -p \pm \sqrt{p^{2} - \omega^{2}_{n}} \). By the overdamping assumption, the roots can then be approximated by \( s_{1} \approx -\omega^{2}_{n } / 2p \) and \( s_{2} \approx -2p - \omega^{2}_{n } / 2p \approx -2p = -b / m  \), appealing to the fact that \( p >> \omega_{n } \).

\vspace{12pt}

% QUESTION 16
\qs{2.3.16}{With small underdamping \( \pmb{p << \omega_{n }} \), the square root of \( p^{2} - \omega^{2}_{n } \) is approximately \( i \omega_{n } - ip^{2} / 2 \omega_{n } \). Square that to come close to \( p^{2} - \omega^{2}_{n } \). Then the frequency for small underdamping is reduced to \( \omega_{d } \approx \omega_{n } - p^{2} / 2 \omega_{n } \).}

Write
\[
	\!\left( i \omega_{n } - i p^{2} / 2 \omega_{n } \right)^{2} = p^{2} - \omega^{2}_{n } + \frac{p^{4}}{4 \omega^{2}_{n }} \approx p^{2} - \omega^{2}_{n }
\]
Since we have underdamping, the magnitude of the square root of \( p^{2} - \omega^{2}_{n } \) is approximated as \( \omega_{d } \approx \omega_{n } - p^{2} / 2 \omega_{n } \).

\vspace{12pt}

% QUESTION 17
\qs{2.3.17}{Here is an 8th order equation with eight choices for solutions \( y = e^{st } \):
\[
	\frac{d^{8}y }{dt^{8}} = y \quad \text{becomes} \quad s^{8}e^{st } = e^{st } \quad \text{and} \quad \pmb{s^{8} = 1} : 
\]
\[
	\text{Eight roots in Figure 2.6} 
\]
Find two solutions \( e^{st } \) that don't oscillate ( \( s  \) is real). Find two solutions that only oscillate ( \( s  \) is imaginary). Find two that spiral into zero and two that spiral out.}

The eighth roots of unity are:
\[
	1, -1, i, -i, e^{i \pi / 4}, e^{i 3 \pi / 4}, e^{i 5 \pi / 4}, e^{i 7 \pi / 4} 
\]
The real solutions are \( e^{t } \) and \( e^{-t } \) and imaginary are \( e^{it } \) and \( e^{-it } \). In order for the solutions to spiral into zero, the real component of \( s  \) must be negative, ensuring that the outer exponential decays with increasing \( t  \). This corresponds to the solutions involving \( e^{i 3 \pi / 4} \) and \( e^{i 5 \pi / 4} \):
\begin{align*}
	& e^{t \cos\!\left( 3 \pi / 4 \right)} \!\left[ \cos\!\left( t \sin\!\left( \frac{3 \pi }{4} \right) \right) + \sin\!\left( t \sin\!\left( \frac{3 \pi }{4} \right) \right) \right] \\[12pt]
	& e^{t \cos\!\left( 5 \pi / 4 \right)} \!\left[ \cos\!\left( t \sin\!\left( \frac{5 \pi }{4} \right) \right) + \sin\!\left( t \sin\!\left( \frac{5 \pi }{4} \right) \right) \right] 
\end{align*}
Lastly, the solutions that spiral away from zero involve \( e^{i \pi / 4} \) and \( e^{i 7 \pi / 4} \), and they are
\begin{align*}
	& e^{t \cos\!\left( \pi / 4 \right)} \!\left[ \cos\!\left( t \sin\!\left( \frac{\pi }{4} \right) \right) + \sin\!\left( t \sin\!\left( \frac{\pi }{4} \right) \right) \right] \\[12pt]
	& e^{t \cos\!\left( 7 \pi / 4 \right)} \!\left[ \cos\!\left( t \sin\!\left( \frac{7 \pi }{4} \right) \right) + \sin\!\left( t \sin\!\left( \frac{7 \pi }{4} \right) \right) \right] 
\end{align*}
Observe here that the outer exponentials have positive arguments, ensuring growth as \( t  \) grows.

\vspace{12pt}

% QUESTION 18
\qs{2.3.18}{
\[
	A_{n } \frac{d^{n }y}{dt^{n }} + \cdots + A_{1}\frac{d y}{d t } + A_{0}y = 0 \text{ leads to } \pmb{A_{n }s^{n } + \cdots + A_{1}s + A_{0} = 0.} 
\]
The \( n  \) roots \( s_{1}, ..., s_{n } \) produce \( n  \) solutions \( y \!\left( t  \right) = e^{st }\) (if those roots are distinct). Write down \( n  \) equations for the constants \( c_{1} \) to \( c_{n } \) in \( y = c_{1}e^{s_{1}t } + \cdots + c_{n }e^{s_{n }t } \) by matching the \( n  \) initial conditions for \( y \!\left( 0 \right) \), \( y'\!\left( 0 \right) \), ..., \( D^{n - 1}y \!\left( 0 \right) \).}

Matching the initial conditions, we write
\begin{alignat*}{1}
	 y \!\left( 0 \right) & = c_{1} + \cdots c_{n }  \\
	 y'\!\left( 0 \right) & = c_{1}s_{1} + \cdots + c_{n }s_{n } \\ 
			      & \quad \vdots \\
	 D^{n - 1}y \!\left( 0 \right) & = c_{1}s^{n - 1}_{1} + \cdots + c_{n }s^{n - 1}_{n}
\end{alignat*}

% QUESTION 19
\qs{2.3.19}{\textbf{Find two solutions to } \( \pmb{d^{2015}y / dt^{2015} = dy/dt } \). Describe all solutions to \( s^{2015} = s  \).}

Two solutions are \( y = e^{t } \) and \( y = e^{-t } \). The set of all solutions to \( s^{2015} = s \) is described by 0 and 
\[
	e^{i 2 k \pi / 2014} \text{ for } k \in \!\left\{ 0, ..., 2013 \right\}
\]

% QUESTION 20
\qs{2.3.20}{The solution to \( y'' = 1 \) starting from \( y \!\left( 0 \right) = y'\!\left( 0 \right) = 0 \) is \( y \!\left( t  \right) = t^{2} / 2 \). The fundamental solution to \( g'' = \delta \!\left( t  \right) \) is \( g \!\left( t  \right) = t  \) by Example 5. Does the integral \( \int g \!\left( t - s  \right) f \!\left( s  \right) ds = \int \!\left( t - s  \right) ds  \) from 0 to \( t  \) give the correct solution \( y = t^{2} / 2 \)?}

The forcing function \( f \!\left( t  \right) \) here simply equals 1, hence the particular solution reduces to solving
\[
	\int_{0}^{t } \!\left( t - s  \right) \, ds = -\frac{\!\left( t - s  \right)^{2}}{2} \Big|^{t }_{0} = \frac{t^{2}}{2}
\]

\newpage

% QUESTION 21
\qs{2.3.21}{The solution to \( y'' + y = 1 \) starting from \( y \!\left( 0 \right) = y'\!\left( 0 \right) = 0 \) is \( y = 1 - \cos t  \). The solution to \( g'' + g = \delta \!\left( t  \right) \) is \( \pmb{g \!\left( t  \right) = \sin t } \) by equation (13) with \( \omega = 1 \) and \( A = 1 \). Show that \( 1 - \cos t  \) agrees with the integral \( \int g \!\left( t - s  \right) f \!\left( s  \right)ds = \int \sin\!\left( t - s \right)ds \).}

We have
\[
	\int_{0 }^{t} \sin\!\left( t - s  \right) \, ds = \cos\!\left( t - s \right) \Big|^{t }_{0} = 1 - \sin\!\left( t \right)
\]

% QUESTION 22
\qs{2.3.22}{The step function \( H \!\left( t  \right) = 1\) for \( t \ge 0 \) is the integral of the delta function. \textbf{So the step response } \( \pmb{r \!\left( t  \right)} \) \textbf{is the integral of the impulse response.} This fact must also come from our basic solution formula:
\[
	Ar'' + Br' + Cr = 1 \text{ with } r \!\left( 0 \right) = r'\!\left( 0 \right) = 0 \text{ has } \pmb{r \!\left( t  \right) = \int_{0}^{t } g \!\left( t - s  \right) \, ds  } 
\]
Change \( t - s  \) to \( \tau  \) and change \( ds  \) to \( -d \tau  \) to confirm that \( r \!\left( t  \right) = \int_{0}^{t } g \!\left( \tau \right) \, d \tau   \).

Section 2.5 will find two good formulas for the step response \( r \!\left( t  \right) \).}

Let \( \tau = t - s  \). Then \( d \tau / ds = -1 \), implying \( ds = -d \tau  \). Lastly, the change of variables implies the change of integration bounds: \( s_{\text{upper}} = t  \) goes to \( \tau_{\text{upper}} = t - t = 0 \) and \( s_{\text{lower}} = 0 \) goes to \( \tau_{\text{lower}} = t - 0 = t  \). Thus we have
\[
	r \!\left( t  \right) = -\int_{t }^{0} g \!\left( \tau  \right) \, d \tau = \int_{0}^{t } g \!\left( \tau  \right) \, d \tau   
\]

\newpage
\subsection{\textsf{2.4 - Forced Oscillations and Exponential Response}}

\textbf{Problems 1-4 use the exponential response } \( \pmb{y_{p } = e^{ct } / P \!\left( c  \right)} \) \textbf{ to solve } \( \pmb{P \!\left( D  \right) y = e^{ct }} \).

\vspace{12pt}

% QUESTION 1
\qs{2.4.1}{Solve these constant coefficient equations with exponential driving force:
\begin{enumerate}[label=(\alph*)]
	\item \( y''_{p } + 3y'_{p } + 5y_{p } = e^{t } \)
	\item \( 2y''_{p } + 4y_{p } = e^{it } \)
	\item \( y'''' = e^{t } \)
\end{enumerate}}

\begin{enumerate}[label=(\alph*)]
	\item Let \( y_{p } = Y e^{t } \). Without using \( P \!\left( D  \right) y = e^{ct } \), we have
	\begin{alignat*}{2}
		 && Ye^{t } + 3Ye^{t } + 5Ye^{t } & = e^{t } \\
		 \implies && 9Y & = 1 \\ 
		 \implies && Y & = 1 / 9
	\end{alignat*}
	Thus \( \displaystyle{y_{p } = \frac{1}{9} e^{t }} \). Note that \( P \!\left( 1 \right) = 1^{2} + 3 \cdot 1 + 5 = 9 \).
	\item Using \( P \!\left( D  \right)y = e^{ct} \), we have
	\[
		y_{p } = \frac{e^{it }}{P \!\left( i  \right)} = \frac{e^{it }}{-2 + 4} = \frac{e^{it }}{2}
	\]
	\item Using \( P \!\left( D  \right)y = e^{ct } \), we have
	\[
		y_{p } = \frac{e^{t }}{P \!\left( 1  \right)} = e^{t}
	\]
\end{enumerate}

% QUESTION 2
\qs{2.4.2}{These equations \( P \!\left( D  \right)y = e^{ct } \) use the symbol \( D  \) for \( d / dt  \). Solve for \( y_{p }\!\left( t  \right) \):
\begin{enumerate}[label=(\alph*)]
	\item \( \!\left( D^{2} + 1 \right)y_{p }\!\left( t  \right) = 10e^{-3t } \)
	\item \( \!\left( D^{2} + 2D + 1 \right)y_{p }\!\left( t  \right) = e^{i \omega t } \)
	\item \( \!\left( D^{4} + D^{2} + 1 \right)y_{p } \!\left( t  \right) = e^{i \omega t} \)
\end{enumerate}}

\begin{enumerate}[label=(\alph*)]
	\item \( \displaystyle{y_{p } \!\left( t  \right) = \frac{10}{\!\left( -3 \right)^{2} + 1} e^{-3t } = e^{-3t}} \)
	\item \( \displaystyle{y_{p } \!\left( t  \right) = \frac{1}{-\omega^{2} + 2\omega i + 1}e^{i \omega t } = \frac{1}{1 - \omega^{2} + 2 \omega i } e^{i \omega t} } \)
	\item \( \displaystyle{y_{p }\!\left( t  \right) = \frac{1}{1 - \omega^{2} + \omega^{4}} e^{i \omega t }} \)
\end{enumerate}

% QUESTION 3
\qs{2.4.3}{How could \( y_{p } = e^{ct } / P \!\left( c  \right) \) solve \( y'' + y = e^{t }e^{it } \) and then \( y'' + y = e^{t } \cos t \)?}

Observe that \( y_{p } \) solves the first equation when \( c = \!\left( 1 + i  \right)t  \), and then the real part serves as the solution to the second.
\begin{align*}
	y_{p } = \frac{e^{\!\left( 1 + i  \right)t }}{P \!\left( 1 + i  \right)} = \frac{1}{1 + 2i}e^{t }e^{it} & = \frac{1 - 2i}{5}e^{t } \!\left[ \cos\!\left( t \right) + i \sin\!\left( t \right) \right] \\
														& = \underbrace{\frac{1}{5}e^{t }\!\left[ \cos\!\left( t  \right) + 2\sin\!\left( t \right) \right]}_{\text{Solution to second equation}} + \frac{i}{5} e^{t } \!\left[ -2 \cos\!\left( t \right) + \sin\!\left( t \right)\right]
\end{align*}

% QUESTION 4
\qs{2.4.4}{\begin{enumerate}[label=(\alph*)]
	\item What are the roots \( s_{1} \) to \( s_{3} \) and the null solutions to \( y_{n }''' - y_{n } = 0 \)?
	\item Find particular solutions to \( y_{p }''' - y_{p } = e^{it } \) and to \( y_{p }''' - y_{p } = e^{t } - e^{i \omega t } \).
\end{enumerate}}

\begin{enumerate}[label=(\alph*)]
	\item The roots are \( s_{1} = 1 \), \( s_{2} = e^{i 2 \pi / 3} = \cos\!\left(2 \pi / 3\right) + i \sin\!\left(2 \pi / 3\right)  \), and \( s_{3} = e^{i 4 \pi / 3} = \cos\!\left(4 \pi / 3 \right) + i \sin\!\left(4 \pi / 3 \right) \). The null solution is
	\begin{align*}
		y_{n } & = c_{1}e^{t} + c_{2}e^{\!\left( -1/2 + i \sqrt{3}/2 \right)t } + c_{3}e^{\!\left( -1/2 - i \sqrt{3}/2 \right)t } \\
		       & = c_{1}e^{t } + \!\left( c_{2} + c_{3} \right)e^{-t/2}\cos\!\left( \frac{\sqrt{3}}{2}t \right) + i \!\left( c_{2} - c_{3} \right)e^{-t/2}\sin\!\left( \frac{\sqrt{3}}{2}t \right)
        \end{align*}	
\item For the first equation, we have \( \displaystyle{y_{p} = \frac{1}{-1-i}}e^{it} \). In the second case, appeal to the superposition principle to separately calculate the particular solution for each forcing term to find
	\[
		y_{p } = \frac{te^{t }}{3} + \frac{1}{1 + i \omega^{3}} e^{i \omega t} 
	\]
	Decomposed into real and imaginary components, we can rewrite as:
	\[
		y_{p } = \frac{te^{t }}{3} + \frac{\cos\!\left(\omega t \right) + \omega^{3} \sin\!\left(\omega t \right) }{1 + \omega^{6}} + i \!\left[ \frac{\sin\!\left(\omega t \right)  - \omega^{3} \cos\!\left(\omega t \right) }{1 + \omega^{6}} \right] 
	\]
\end{enumerate}

% QUESTION 5
\qs{2.4.5}{Which value of \( C  \) gives resonance in \( y'' + Cy = e^{i \omega t } \)? Why do we never get resonance in \( y'' + 5y' + Cy = e^{i \omega t } \)?}

We must have \( P \!\left( i \omega  \right) = \!\left( i \omega \right)^{2} + C = 0\) for some value of \( C  \). It immediately follows that we must have \( C = \omega^{2} \). In the second case, observe that the roots of \( P \!\left( s  \right) = s^{2} + 5s + C  \) can never be pure imaginary. Thus it is impossible for \( s = i \omega \) to ever be a solution to \( P \!\left( s  \right) = 0 \).

% QUESTION 6
\qs{2.4.6}{Suppose the third order equation \( P \!\left( D  \right) y_{n } = 0 \) has solutions \( y = c_{1}e^{t } + c_{2}e^{2t } + c_{3}e^{3t } \). What are the null solutions to the sixth order equation \( P \!\left( D  \right) P \!\left( D  \right)y_{n } = 0 \)?}

We must have repeated roots, as we can think of \( P \!\left( D  \right) \) as repeated `factors` to the polynomial corresponding to the sixth order equation. With repeated roots, we must have
\[
	y = c_{1}e^{t} + c_{2}e^{2t} + c_{3}e^{3t} + c_{4}te^{t} + c_{5}te^{2t} + c_{6}te^{3t} 
\]

% QUESTION 7
\qs{2.4.7}{Complete this table with equations for \( s_{1} \) and \( s_{2} \) and \( y_{n } \) and \( y_{p } \):
\begin{alignat*}{3}
	& \textbf{Undamped free oscillation} && \quad my'' + ky = 0 && \quad \pmb{y_{n }} = \underline{\qquad} \\
	& \textbf{Undamped forced oscillation} && \quad my'' + ky = e^{i \omega t } && \quad \pmb{y_{p }} = \underline{\qquad} \\
	& \textbf{Damped free motion} && \quad my'' + by' + ky = 0 && \quad \pmb{y_{n }} = \underline{\qquad} \\
	& \textbf{Damped forced motion} && \quad my'' + by' + ky = e^{ct } && \quad \pmb{y_{p }} = \underline{\qquad}
\end{alignat*}}

Let \( \omega_{n } = \sqrt{k / m } \). In the order of the table:
\begin{enumerate}[label=(\alph*)]
	\item Solve \( s^{2} + \omega_{n}^{2} = 0 \) to find
	\[ 
		s_{1,2} = \pm i \omega_{n}, \quad \displaystyle{ y_{n } = c_{1} \cos\!\left( \omega_{n } t \right) + ic_{2}\sin\!\left( \omega_{n } t \right) } 
	\]
	\item Divide through by \( m  \) and \( P \!\left( i \omega  \right) = \!\left( i \omega  \right)^{2} + \omega_{n} = \omega_{n} - \omega^{2} \). Then we have
	\[
		y_{p } = \frac{1}{m \!\left( \omega_{n}^{2} - \omega^{2} \right)}e^{i \omega t}
	\]
	\item Solve \( s^{2} + \!\left( b / m  \right)s + \omega_{n}^{2} = 0 \) to find
	\[
		s = \frac{-b/m \pm \sqrt{\displaystyle{\frac{b^{2} }{m^{2}}} - 4 \omega^{2}_{n} }}{2} = \frac{-b \pm \sqrt{b^{2} - 4mk}}{2m}
	\]
	This leads us to null solution
	\[
		y_{n} = e^{-b / 2m} \!\left( c_{1} e^{\!\left( \sqrt{b^{2} - 4mk} \right) t} + c_{2} e^{-\!\left( \sqrt{b^{2} - 4mk} \right) t} \right) 
	\]
	\item Divide through by \( m  \) and \( P \!\left( c  \right) = c^{2} + \!\left( b/m  \right)c + k / m \). Then we have
	\[
		y_{p } = \frac{1}{mc^{2} + bc + k} e^{ct}
	\]
\end{enumerate}

% QUESTION 8
\qs{2.4.8}{Complete the same table when the coefficients are 1 and \( 2 Z_{\omega_{n }} \) and \( \omega^{2}_{n } \) with \( Z < 1 \).
\begin{alignat*}{3}
	& \textbf{Undamped and free} && \quad y'' + \omega^{2}_{n }y = 0 && \quad \pmb{y_{n }} = \underline{\qquad} \\
	& \textbf{Undamped and forced} && \quad y'' + \omega^{2}_{n } y = e^{i \omega t } && \quad \pmb{y_{p }} = \underline{\qquad} \\ 
	& \textbf{Underdamped and free} && \quad y'' + 2Z \omega_{n } y' + \omega^{2}_{n } y = 0 && \quad \pmb{y_{n}} = \underline{\qquad} \\
	& \textbf{Underdamped and forced} && \quad y'' + 2Z \omega_{n } y' + \omega^{2}_{n } y = e^{ct } && \quad \pmb{y_{p }} = \underline{\qquad}
\end{alignat*}
}

Let \( Z = b / \sqrt{4mk} \) and \( \omega_{n } \) be as before. These are similar equations as in the previous question, but using different variable definitions.

\begin{enumerate}[label=(\alph*)]
	\item Solve \( s^{2} + \omega^{2}_{n } = 0 \) to find roots \( s = \pm i \omega_{n } \). The null solution is
	\[
		y_{n } = c_{1} e^{i \omega_{n } t } + c_{2} e^{-i \omega_{n } t} = C_{1} \cos\!\left( \omega_{n } t \right) + i C_{2} \sin\!\left( \omega_{n } t \right)  
	\]
	\item Calculate \( P \!\left( i \omega \right) = \omega^{2} + \omega^{2}_{n } \) and divide the right-hand side by \( P \!\left( i \omega \right) \)to find
	\[
		y_{p } = \frac{1}{\omega^{2} + \omega^{2}_{n }} e^{i \omega t} 
	\]
	\item Solve \( s^{2} + 2Z \omega_{n } s + \omega^{2}_{n } = 0 \) to find roots
	\[
		s = \frac{-2Z \omega_{n } \pm \sqrt{4 Z^{2} \omega^{2}_{n } - 4 \omega^{2}_{n }}}{2} = -Z \omega_{n } \pm i \omega_{n } \sqrt{1 - Z^{2}}
	\]
	with the imaginary term ensuing from the premise that \( Z < 1 \). This leads us to the null solution
	\[
		y_{n } = e^{-Z \omega_{n}} \!\left[ c_{1} \cos\!\left( \omega_{n } \!\left( \sqrt{1 - Z^{2}} \right) t \right) + ic_{2} \sin\!\left( \omega_{n } \!\left( \sqrt{1 - Z^{2}} \right) t \right)  \right] 
	\]
	\item Calculate \( P \!\left( c  \right) = c^{2} + 2 Z \omega_{n } c + \omega^{2}_{n } \) and divide the right-hand side by \( P \!\left( c  \right) \) to get
	\[
		y_{p } = \frac{1}{c^{2} + 2 Z \omega_{n } c + \omega^{2}_{n }} e^{ct} 
	\]
\end{enumerate}

% QUESTION 9
\qs{2.4.9}{What equations \( y'' + By' + Cy = f  \) have these solutions?
\begin{enumerate}[label=(\alph*)]
	\item \( y = c_{1} \cos\!\left( 2t \right) + c_{2} \sin\!\left( 2t \right) + \cos\!\left( 3t \right) \)
	\item \( y = c_{1}e^{-t } \cos\!\left( 4t \right) + c_{2} e^{-t } \sin\!\left( 4t \right) + \cos\!\left( 5t \right)  \)
	\item \( y = c_{1} e^{-t } + c_{2} t e^{-t } + e^{i \omega t}  \)
\end{enumerate}}

\begin{enumerate}[label=(\alph*)]
	\item The first two terms constitute the null solution. One solution is to substitute the null solution in the equation, and deducing that in collecting the sine and cosine terms, \( B  \) and \( C  \) must be such that these equations vanish:
	\begin{align*}
		-4 c_{1} + 2B c_{2} + Cc_{1} & = 0 \\
		-4 c_{2} - 2B c_{1} + Cc_{2} & = 0 
	\end{align*}
	We can conclude that \( B = 0 \) and \( C = 4 \). 

	Alternatively, recall from question 2.3.7 that \( s_{1} + s_{2} = B/A \) and \( s_{1}s_{2} = C/A  \) (we call these Vieta's formulas, which can be generalized to any polynomial of degree \( n  \)). From the absence of the damping term, we can surmise that the null solution is undamped and freely oscillates, with roots \( s_{1,2} = \pm 2i \). Using Vieta's formulas, we have \( s_{1} + s_{2} = B = 0 \) and \( s_{1}s_{2} = C = 4 \).

	The homogeneous equation is
	\[
		y'' + 4y = 0
	\]
	For the forcing term \( f \), substitute the particular solution to derive
	\[
		 -9 \cos\!\left( 3t \right) + 4 \cos\!\left( 3t \right) = f = -5 \cos\!\left( 3t \right) 
	\]
	\item The roots are \( s_{1,2} = -1 \pm 4i \), which lead to polynomial \( s^{2} + 2s + 17 \). Intuitively, the damping term tracks with the exponential decay. Our homogeneous equation is
	\[
		 y'' + 2y' + 17y = 0
	\]
	For the forcing term, substitute the particular solution to find
	\[
		-25 \cos\!\left( 5t \right) - 10 \sin\!\left( 5t \right) + 17 \cos\!\left( 5t \right) = f = -8 \cos\!\left( 5t \right) - 10 \sin\!\left( 5t \right)  
	\]
	\item We have repeated real roots: \( s_{1,2} = 1 \). This suggests polynomial
	\[
		\!\left( s - 1 \right)^{2} = s^{2} - 2s + 1 
	\]
	The corresponding homogeneous differential equation, which has critical damping, is
	\[
		y'' - 2y' + y = 0 
	\]
	The particular solution arises from forcing term
	\[
		\!\left( i \omega  \right)^{2} e^{i \omega t } - 2 i \omega e^{i \omega t } + e^{i \omega t } = f = e^{i \omega t } \!\left( 1 - \omega^{2} - 2i \right)
	\]
\end{enumerate}

% QUESTION 10
\qs{2.4.10}{If \( y_{p } = te^{-6t } \cos\!\left( 7t \right)  \) solves a second order equation \( Ay'' + By' + Cy = f  \), what does that tell you about \( A  \), \( B  \), \( C  \), and \( f  \)?}

The particular solution suggests that we simultaneously have resonance and complex roots \( s = -6 \pm 7i \): thus \( B^{2} < 4AC \) and if \( P \!\left( s  \right) = As^{2} + Bs + C  \), then \( P \!\left( -6 \pm 7i  \right) = 0 \). Now, the forcing function must be a linear combination of real terms involving \( e^{-6t } \cos\!\left( 7t \right)  \) or \( e^{-6t} \sin\!\left( 7t \right)  \); the \( t  \) arises because of resonance, and is not a part of the forcing function. If we wish to determine exactly \( A, B, C  \), and \( f  \), we must substitute the particular solution into our equation (this is tedious).

Sparing the reader from the work, it turns out \( A, B, C  \) are \( 1, 12, 85 \), respectively. The homogeneous equation is
\[
	 y'' + 12y' + 85 = 0
\]
Incorporating these into the quadratic formula returns the desired roots. Using these values, the forcing function is then
\[
	f = -14e^{-6t } \sin\!\left( 7t \right) 
\]
The insights:
\begin{itemize}[label=\(\circ\)]
	\item The forcing function, generally, has form 
	\[ 
		f \!\left( t  \right) = Y \!\left( s = -6 + 7i \right) e^{-6t} \!\left( \cos\!\left( 7t \right) + i \sin\!\left( 7t \right) \right) 
	\]
	\item Multiplying through by the complex coefficient \( Y \!\left( s  \right) \) will yield the form
	\[
		f \!\left( t  \right) = e^{-6t } \!\left( c_{1} \cos\!\left( 7t \right) + c_{2} \sin\!\left( 7t \right)  \right) 
	\]
	where \( c_{1} \) and \( c_{2} \) are real coefficients.
	\item Since the forcing function oscillates at the natural frequency (i.e. has the same roots as the characteristic polynomial implied by the differential equation), this implies that the particular solution \( y_{p } \) will include a \( t  \) term. This is resonance: intuitively, it suggests that since the forcing frequency is equal to the natural frequency, there will be an amplification in the response as evidenced by the \( t  \) term.
\end{itemize}

% QUESTION 11
\qs{2.4.11}{\begin{enumerate}[label=(\alph*)]
	\item Find the steady oscillation \( y_{p }\!\left( t  \right) \) that solves \( y'' + 4y' + 3y = 5 \cos\!\left( \omega t \right)  \).
	\item Find the amplitude \( A  \) of \( y_{p }\!\left( t  \right) \) and its phase lag \( \alpha  \).
	\item Which frequency \( \omega  \) gives maximum amplitude (maximum gain)?
\end{enumerate}}

\begin{enumerate}[label=(\alph*)]
	\item In rectangular form, let \( y_{p } \!\left( t  \right) = M \cos\!\left( \omega t \right) + N \sin\!\left( \omega t \right) \). Deriving \( M  \) and \( N  \) in the same fashion as equations (20) and (21) in Strang lead us to
	\begin{align*}
		M  & = \frac{5 \!\left( 3 - \omega^{2} \right)}{\!\left( 3 - \omega^{2} \right)^{2} + \!\left( 4 \omega  \right)^{2}} \\[12pt]
		N  & = \frac{5 \!\left( 4 \omega  \right)}{\!\left( 3 - \omega^{2} \right)^{2} + \!\left( 4 \omega  \right)^{2}} \\ 
	\end{align*}
	\item The amplitude \( A  \) is
	\[
		A = \frac{5}{\sqrt{\!\left( 3 - \omega^{2} \right)^{2} + \!\left( 4 \omega  \right)^{2}}} 
	\]
	and its phase lag is
	\[
		\alpha = \tan^{-1}\!\left( \frac{4 \omega }{3 - \omega^{2}} \right)
	\]
	\item When the denominator is smallest, the gain is largest. Differentiate the argument of the square root and optimize to find:
	\[
		\frac{d }{d \omega } \!\left( \!\left( 3 - \omega^{2} \right)^{2} + \!\left( 4 \omega \right)^{2} \right) = -4 \omega \!\left( 3 - \omega^{2} \right) + 8 \!\left( 4 \omega \right) = 0 \quad \implies \omega = 0, \pm \sqrt{5}i
	\]
	Taking the second derivative and evaluating at \( \omega = 0 \) (we cannot have an imaginary frequency in this instance) ascertains that this is a local minimum.
	\[
		\frac{d^{2}}{d \omega^{2}}_{\!\left( \omega = 0 \right)} \!\left( \!\left( 3 - \omega^{2} \right)^{2} + \!\left( 4 \omega  \right)^{2} \right) = -12 + 12 \omega^{2} + 32 = 20
	\]
	Thus the maximum amplitude is \( A = 5/3 \). 
\end{enumerate}
Conceptually, we can conclude that a higher oscillating frequency contributes to increasing damping of the response function. So much so, that the forcing ought not to oscillate at all: \( \omega = 0 \).

\vspace{12pt}

% QUESTION 12
\qs{2.4.12}{Solve \( y'' + y = \sin\!\left( \omega t \right)  \) starting from \( y \!\left( 0 \right) = 0 \) and \( y'\!\left( 0 \right) = 0 \). Find the limit of \( y \!\left( t  \right) \) as \( \omega  \) approaches 1, and the problem approaches resonance.}

We have \( y_{n }\!\left( t  \right) = c_{1}\cos\!\left( t \right) + c_{2}\sin\!\left( t \right) \). For the particular solution, let \( y_{p }\!\left( t  \right) = Y \sin\!\left( \omega t \right)  \). Solving for \( Y \), we find
\[
	Y = \frac{1}{1 - \omega^{2}} 
\]
Imposing the initial conditions, we must have
\begin{align*}
	y \!\left( 0 \right) & = c_{1} = 0 \\
	y'\!\left( 0 \right) & = c_{2} + \frac{\omega}{1 - \omega^{2}} = 0 \quad \implies c_{2} = -\frac{\omega}{1 - \omega^{2}} \\ 
\end{align*}
Therefore, our solution is
\[
	y \!\left( t  \right) = \frac{-\omega \sin\!\left( t \right) + \sin\!\left(\omega t \right) }{1 - \omega^{2}}
\]
Now consider the limit as \( \omega \rightarrow 1 \). Since we have the indeterminate form \( 0 / 0 \), we apply l'H\^opital's rule to find
\[
	\lim\limits_{\omega \to 1} y \!\left( t \right) = \lim\limits_{\omega \to 1} \frac{-\sin\!\left( t \right) + t \cos\!\left( \omega t \right) }{-2 \omega} = \frac{\sin\!\left( t \right)  - t \cos\!\left( t \right) }{2}
\]
As expected, resonance happens.

\vspace{12pt}

% QUESTION 13
\qs{2.4.13}{Does critical damping and a double root \( s = -1 \) in \( y'' + 2y' + y = e^{ct } \) produce an extra factor \( t  \) in the null solution \( y_{n } \) or in the particular solution \( y_{p } \) (proportional to \( e^{ct } \))? What is \( y_{n } \) with constants \( c_{1} \), \( c_{2} \)? What is \( y_{p } = Ye^{ct } \)?}

An extra factor \( t  \) appears in the null solution, which is
\[
	y_{n } = c_{1} e^{-t } + c_{2}te^{-t}
\]
The particular solution is
\[
	Yc^{2}e^{ct } + 2Yce^{ct } + Ye^{ct } = e^{ct } \quad \implies Y = \frac{1}{c^{2} + 2c + 1} 
\]
\[
	y_{p } = \frac{e^{ct }}{c^{2} + 2c + 1} 
\]

% QUESTION 14
\qs{2.4.14}{If \( c = i \omega  \) in Problem 13, the solution \( y_{p } \) to \( y'' + 2y' + y = e^{i \omega t } \) is \underline{\qquad}. That fraction \( Y \) is the transfer function at \( i \omega  \). What are the magnitude and phase in \( Y = Ge^{-i \alpha} \)?

\textbf{By rescaling both }\( \pmb{t} \) \textbf{and} \( \pmb{y} \),\textbf{ we can reach} \( \pmb{A = C = 1} \). \textbf{Then } \( \pmb{\omega_{n } = 1} \) \textbf{and} \( \pmb{B = 2Z} \). \textbf{The model problem is} \( \pmb{y'' + 2Zy' + y = f \!\left( t  \right)} \).}

The particular solution becomes
\[
	y_{p } = \frac{e^{i \omega t }}{1 - \omega^{2} + 2i \omega} 
\]
The gain and phase are calculated by
\[
	G = \frac{1}{\sqrt{\!\left( 1 - \omega^{2} \right)^{2} + 4 \omega^{2}}}, \quad \alpha = \tan^{-1}\!\left( \frac{2 \omega}{1 - \omega^{2}} \right)
\]

% QUESTION 15
\qs{2.4.15}{What are the roots of \( s^{2} + 2Zs + 1 = 0 \)? Find two roots for \( Z = 0, \frac{1}{2}, 1, 2 \) and identify each type of damping. The natural frequency is now \( \omega_{n } = 1 \).}

The roots are
\[
	s = \frac{-2Z \pm \sqrt{4Z^{2} - 4}}{2} = -Z \pm \sqrt{Z^{2} - 1}
\]
For each value of \( Z \), we have
\begin{center}
	\renewcommand{\arraystretch}{2.5}
	\begin{tabular}{|c|c|c|}
		\hline
		\( \pmb{Z} \) & \textbf{Roots} & \textbf{Damping} \\
		\hline
		\hline
		\( 0 \) & \( \pm i \) & No damping\\
		\hline
		\( \displaystyle{\frac{1}{2}} \) & \( \displaystyle{-\frac{1}{2} \pm i \frac{\sqrt{3}}{2}} \) & Underdamping \\
		\hline
		\( 1 \) & \( -1 \) (double root) & Critical damping\\
		\hline
		\( 2 \) & \( -2 \pm \sqrt{3} \) & Overdamping \\
		\hline
	\end{tabular}
\end{center}

% QUESTION 16
\qs{2.4.16}{Find two solutions to \( y'' + 2Zy' + y = 0 \) for every \( Z \) except \( Z = 1 \) and \( Z = -1 \). Which solution \( g \!\left( t  \right) \) starts from \( g \!\left( 0 \right) = 0 \) and \( g'\!\left( 0 \right) = 1 \)? What is different about \( Z = 1 \)?}

The characteristic polynomial is \( s^{2} + 2Zs + 1 = 0 \), which has roots \( s = -Z \pm \sqrt{Z^{2} - 1} \), as in the previous question. For all \( Z  \) except \( Z = 1, -1 \) (which has repeated roots), we have
\[
	y_{n }\!\left( t  \right) = c_{1} e^{\!\left( -Z + \sqrt{Z^{2} - 1} \right) t} + c_{2} e^{\!\left( -Z - \sqrt{Z^{2} - 1} \right) t } 
\]
Under the initial conditions, we must have
\begin{align*}
	y_{n }\!\left( 0 \right) & = c_{1} + c_{2} = 0 \implies \boxed{c_{1} = - c_{2}} \\[12pt]
	y_{n }'\!\left( 0 \right) & = c_{1} \!\left( -Z + \sqrt{Z^{2} - 1} \right) + c_{2} \!\left( -Z - \sqrt{Z^{2} - 1} \right) \\[12pt]
				  & = 2c_{1} \sqrt{Z^{2} - 1} = 1 \implies \boxed{c_{1} = \frac{1}{2 \sqrt{Z^{2} - 1}}}
\end{align*}
At \( Z = 1 \), observe that \( c_{1}, c_{2} \) are undefined, as we have resonance and the null solution must feature a \( te^{st } \) term.

\vspace{12pt}

% QUESTION 17
\qs{2.4.17}{The equation \( my'' + ky = \cos\!\left( \omega_{n } t \right)  \) is exactly at resonance. The driving frequency on the right side equals the natural frequency \( \omega_{n } = \sqrt{k / m } \) on the left side. Substitute \( y = Rt \sin\!\left( \sqrt{k / m } t \right) \) to find \( R  \). This resonant solution grows in time because of the factor \( t  \).}

Substituting, we find
\[
	2R \sqrt{mk} \cos\!\left( \sqrt{k / m } t \right) = \cos\!\left( \sqrt{k / m } t \right)  
\]
which leads us to conclude that
\[
	R = \frac{1}{2 \sqrt{mk}} 
\]

% QUESTION 18
\qs{2.4.18}{Comparing the equations \( Ay'' + By' + Cy = f \!\left( t  \right) \) and \( 4Az'' + Bz' + \!\left( C/4 \right)z = f \!\left( t / 4  \right) \), what is the difference in their solutions?}

Note that there is a typo in the book; the argument in the forcing term in the \( z \)-equation needs to be scaled by \( 1/4 \). Let \( y \!\left( t  \right) \) be a solution to the \( y \)-equation. The relationship between the roots of the \( y \)- and \( z \)-equations is \( s_{z} = s_{y}/4 \). This is analogous to ``scaling`` time by \( 1/4 \), which is consistent with the z-forcing function.

Now, we have \( y \!\left( t / 4 \right) \). Its derivatives are
\[
	y'\!\left( \frac{t }{4} \right) = \frac{1}{4}y \!\left( \frac{t}{4} \right), \quad y''\!\left( \frac{t }{4} \right) = \frac{1}{16}y \!\left( \frac{t }{4} \right) 
\]
Substituting in the \( z \)-equation gives us
\[
	\!\left( \frac{A }{4} + \frac{B }{4} + \frac{C }{4} \right) y \!\left( \frac{t }{4} \right) = f \!\left( \frac{t }{4} \right)
\]
But since the forcing function here is not itself scaled (only time is scaled), we can throw a 4 in front of \( y \!\left( t/4 \right) \) to conclude that
\[
	z \!\left( t  \right) = 4 y \!\left( \frac{t }{4} \right) 
\]

% QUESTION 19
\qs{2.4.19}{Find the fundamental solution to the equation \( g'' - 3g' + 2g = \delta \!\left( t  \right) \).}

The characteristic polynomial is \( s^{2} -3s + 2 = 0 \) which has roots \( s_{1,2} = 1, 2 \). Then the equation is overdamped (real roots), and the fundamental solution is given by
\[
	g \!\left( t  \right) = e^{2t } - e^{t } 
\]

% QUESTION 20
\qs{2.4.20}{(Challenge problem) Find the solution to \( y'' + By' + y = \cos\!\left( t \right)  \) that starts from \( y \!\left( 0 \right) = 0 \) and \( y'\!\left( 0 \right) = 0 \). Then let the damping constant \( B  \) approach zero, to reach the resonant equation \( y'' + y = \cos\!\left( t \right)  \) in Problem 17, with \( m = k = 1 \).

Show that your solution \( y \!\left( t  \right) \) is approaching the resonant solution \( \frac{1}{2}t \sin\!\left( t \right)  \).}

The characteristic polynomial is \( s^{2} + Bs + 1 = 0 \) with roots
\[
	s_{1,2} = \frac{-B \pm \sqrt{B^{2} - 4}}{2} 
\]
Thus the null solution is
\[
	y_{n } \!\left( t  \right) = c_{1} e^{s_{1}t } + c_{2} e^{s_{2} t } 
\]
The particular solution is
\[
	y_{p } \!\left( t  \right) = \frac{1}{B } \sin\!\left( t \right)  
\]
which can be deduced by inspection: notice how the \( y'' \) and \( y \) terms will annihilate each other (made convenient by the fact that \( \omega = 1 \) here), so the forcing has a response offset by 90 degrees (cosine to sine) and scaled in amplitude by \( 1 / B  \). Imposing the initial conditions forces
\begin{align*}
	y \!\left( 0 \right) & = c_{1} + c_{2} = 0 \implies \boxed{c_{1} = -c_{2}} \\[12pt]
	y'\!\left( 0 \right) & = c_{1}s_{1} + c_{2}s_{2} + \frac{1}{B } = 0 \implies \boxed{c_{1} = \frac{1}{B \!\left( s_{2} - s_{1} \right)}} \\ 
\end{align*}
Our full solution is
\[
	y \!\left( t  \right) = \frac{e^{s_{1}t } - e^{s_{2} t }}{B \!\left( s_{2} - s_{1} \right)} + \frac{1}{B } \sin\!\left( t \right) = \frac{e^{s_{2} t } - e^{s_{1} t }}{B \sqrt{B^{2} - 4}} + \frac{1}{B } \sin\!\left( t \right) 
\]
There are many terms so it is mildly confusing, but if you set \( B = 0 \) and algebraically manipulate everything to have common denominators, we end up with \( y \!\left( t  \right) = 0/0 \). The indeterminate form enables us to move forward with l'H\^opital's rule. Unfortunately the algebra will be tedious. Use the following identities:
\begin{align*}
	s_{1,2, \!\left( B = 0 \right)} & = \boxed{\pm i} \\[12pt]
	\frac{d s_{1,2}t}{d B }_{\!\left( B = 0 \right)} & = t \!\left( -\frac{1}{2} \pm \frac{B }{2 \sqrt{B^{2} - 4}} \right) = \boxed{-\frac{t}{2}} \\[12pt]
	\frac{d }{d B }_{\!\left( B = 0 \right)} \!\left( B \sqrt{B^{2} - 4} \right) & = \sqrt{B^{2} - 4} + \frac{B^{2}}{\sqrt{B^{2} - 4}} = \boxed{2i} \\[12pt]
	\frac{d }{d B }_{\!\left( B = 0 \right)} \!\left( \sqrt{B^{2} - 4} \right) & = \frac{B }{\sqrt{B^{2} - 4}} = \boxed{0}
\end{align*}

Now, applying l'H\^opital's rule:
\begin{align*}
	\lim\limits_{B  \to 0} y \!\left( t  \right) & = \lim\limits_{B  \to 0} \frac{e^{s_{2} t } \frac{d s_{2}t}{d B } - e^{s_{1} t } \frac{d s_{1}t}{d B} + \frac{d }{d B } \!\left( \sqrt{B^{2} - 4} \right) \sin\!\left( t \right) }{\frac{d }{d B } \!\left( B \sqrt{B^{2} - 4} \right)} \\[12pt]
						     & = \frac{\frac{t}{2}\!\left( -e^{-it } + e^{it}\right) }{2i} \\[12pt]
						     & = \frac{1}{2}t \sin\!\left(t\right) 
\end{align*}

Tiresome, but a deeply satisfying conclusion.

\vspace{12pt}

% QUESTION 21
\qs{2.4.21}{Suppose you know three solutions \( y_{1} \), \( y_{2} \), \( y_{3} \) to \( y'' + B \!\left( t  \right) y' + C \!\left( t  \right)y = f \!\left( t  \right) \). How could you find \( B \!\left( t  \right) \) and \( C \!\left( t  \right) \) and \( f \!\left( t  \right) \)? }

First determine that \( y_{1} - y_{2} \) and \( y_{2} - y_{3} \) are in the null space of the equation, and one leading to the forcing term. Define each as \( u  \) and \( v  \) respectively. Analogize to solving a system of equations:
\begin{alignat*}{4}
	& u'' & + B \!\left( t  \right) u' & + C \!\left( t  \right) u && = 0 \\[6pt]
	& v'' & + B \!\left( t  \right) v' & + C \!\left( t  \right) v && = 0 \\[6pt]
	& y_{3}'' & + B \!\left( t  \right) y_{3}' & + C \!\left( t  \right) y_{3} && = f \!\left( t \right) 
\end{alignat*}

Isolate the \( B \!\left( t  \right) \) and \( C \!\left( t  \right) \) terms. My solutions for each are
\begin{align*}
	B \!\left( t  \right) & = \frac{u''v - v''u }{v'u - u'v} \\[12pt]
	C \!\left( t \right) & = \frac{u''v' - v''u'}{vu' - uv'} \\ 
\end{align*}
Now incorporate both into the third equation to derive \( f \!\left( t  \right) \).

\newpage
\subsection{\textsf{2.5 - Electrical Networks and Mechanical Systems}}

\qs{2.5.1}{(Resistors in parallel) Two parallel resistors \( R_{1} \) and \( R_{2} \) connect a node at voltage \( V  \) to a node at voltage zero. The currents are \( V / R_{1} \) and \( V / R_{2} \). What is the total current \( I \) between the nodes? Writing \( R_{12} \) for the ratio \( V / I  \), what is \( R_{12} \) in terms of \( R_{1} \) and \( R_{2} \)?}

Since the resistors are in parallel, we sum the currents to find
\[
	I = I_{1} + I_{2} = \frac{V }{R_{1}} + \frac{V }{R_{2}} = V \!\left( \frac{R_{1} + R_{2}}{R_{1}R_{2}} \right) 
\]
Substitute into \( R_{12} = V / I  \) to find
\[
	R_{12} = \frac{R_{1}R_{2}}{R_{1} + R_{2}} 
\]

% QUESTION 2
\qs{2.5.2}{(Inductor and capacitor in parallel) Those elements connect a node at voltage \( Ve^{i \omega t } \) to a node at voltage zero (grounded node). The currents are \( \!\left( V / i \omega L \right) e^{i \omega t } \) and \( V \!\left( i \omega C  \right)e^{i \omega t } \). The total current \( I e^{i \omega t } \) between the nodes is their sum. Writing \( Z_{12}e^{i \omega t } \) for the ratio \( Ve^{i \omega t } / I e^{i \omega t } \), what is \( Z_{12} \) in terms of \( i \omega L \) and \( i \omega C  \)?}

Deriving the total current, we get
\[
	Ie^{i \omega t } = \frac{V }{i \omega L }e^{i \omega t } + V \!\left( i \omega C  \right) e^{i \omega t } = V e^{i \omega t } \!\left( \frac{1}{i \omega L } + i \omega C  \right) 
\]
Substituting into the equation for \( Z_{12} \) yields
\[
	Z_{12}e^{i \omega t } = \frac{V e^{i \omega t }}{I e^{i \omega t }} = \frac{i \omega L }{1 - \omega^{2} CL } 
\]

% QUESTION 3
\qs{2.5.3}{The impedance of an RLC loop is \( Z = i \omega L + R + 1/i \omega C  \). This impedance \( Z  \) is real when \( \omega =  \) \underline{\qquad}. This impedance is pure imaginary when \underline{\qquad}. This impedance is zero when \underline{\qquad}.}

The impedance is real when \( i \omega L = -1/i \omega C \) (assuming there is a signal, i.e. non-zero \( \omega  \)). Solving for \( \omega  \), we find \( \omega = 1 / \sqrt{LC } \). It is pure imaginary when \( R = 0 \). The impedance is zero when both of the aforementioned conditions hold.

\vspace{12pt}

% QUESTION 4
\qs{2.5.4}{What is the impedance \( Z \) of an RLC loop when \( R = L = C = 1 \)? Draw a graph that shows the magnitude \( \abs{Z} \) as a function of \( \omega  \).}

We get
\[
	Z = 1 + i \omega + \frac{1}{i \omega } = 1 + i \!\left( \omega - \frac{1}{\omega } \right) 
\]
The magnitude of the impedance is
\[
	\abs{Z} = \sqrt{1 + \!\left( \omega - \frac{1}{\omega } \right)^{2}} 
\]
Observe that as \( \omega \rightarrow \infty \), only the \( \omega  \) dominates and washes out the others, so we can approximate \( \abs{Z} \approx \omega  \). Conversely, if \( \omega \rightarrow 0 \), then \( 1 / \omega  \) dominates, and we have \( \abs{Z} \approx 1 / \omega  \). We can represent this graphically:

\begin{center}
\includegraphics[width=1.0\textwidth]{../chap2/sec2.5/chap2sec2.5ex4.pdf}
\end{center}

\vspace{12pt}

% QUESTION 5
\qs{2.5.5}{Why does an LC loop with no resistor produce a \( 90^{\circ} \) phase shift between current and voltage? Current goes around the loop from a battery of voltage \( V  \) in the loop.}

Conceptually, think of the impedance \( Z \) as the amount by which current is opposed in a circuit. This opposition can come in two forms: resistance or reactance, the real and imaginary components of \( Z \), respectively. Depending on the values of both, the phase difference between current and voltage changes:
\[
	I \!\left( t  \right) = \frac{V }{\abs{Z}} e^{i \!\left( \omega t - \alpha \right)} 
\]
Now, under the premise that \( R \) is zero, we have
\[
	R \rightarrow 0 \quad \implies \quad \tan\!\left(\alpha\right) = \frac{\omega L - 1 / \omega C }{R} \rightarrow \infty 
\]
When \( \tan\!\left(\alpha\right)  \) diverges in this manner, we can conclude that \( \alpha = \pi / 2 \), ascertaining that the phase shift between current and voltage is indeed \( 90^{\circ} \).

\vspace{12pt}

% QUESTION 6
\qs{2.5.6}{The mechanical equivalent of zero resistance is zero damping: \( my'' + ky = \cos\!\left(\omega t \right)  \). Find \( c_{1} \) and \( Y  \) starting from \( y \!\left( 0 \right) = 0 \) and \( y'\!\left( 0 \right) = 0 \) with \( \omega^{2}_{n } = k/m  \).
\[
	y \!\left( t  \right) = c_{1} \cos\!\left(\omega_{n }t \right) + Y \cos\!\left(\omega t \right)  
\]
That answer can be written in two equivalent ways:
\[
	y = Y \!\left( \cos\!\left(\omega t \right) - \cos\!\left( \omega_{n }t \right)  \right) = 2Y \sin\!\left( \frac{\!\left( \omega_{n } - \omega  \right) t }{2} \right) \sin\!\left( \frac{ \!\left( \omega_{n } + \omega \right) t}{2}\right)  
\]}
Substitute \( y_{p } \) into the equation to find
\[
	-m \omega^{2}Y + kY = 1 \quad \implies \boxed{Y = \frac{1}{k - m \omega^{2}}}
\]
From the first initial condition, we deduce
\[
	y \!\left( 0 \right) = c_{1} + Y = 0 \quad \implies \boxed{c_{1} = -Y}
\]
which gives us complete solution
\[
	y \!\left( t  \right) = \frac{\cos\!\left(\omega_{n }t \right) - \cos\!\left(\omega t \right) }{k - m \omega^{2}} 
\]

% QUESTION 7
\qs{2.5.7}{Suppose the driving frequency \( \omega  \) is close to \( \omega_{n } \) in Problem 6. A fast oscillation \( \sin \!\left[ \!\left( \omega_{n } + \omega  \right) t/2 \right] \) is multiplying a very slow oscillation \( 2Y \sin \!\left[ \!\left( \omega_{n } - \omega  \right)t/2 \right] \). By hand or by computer, draw the graph of \( y = \sin\!\left(t \right) \sin\!\left(9t \right)  \) from \( 0 \) to \( 2 \pi  \).

You should see a fast sine curve inside a slow sine curve. This is a \textbf{beat}.}

Suppose that \( \omega \rightarrow \omega_{n } \). Then we can write for the fast oscillation
\[
	\sin \!\left[ \frac{\!\left( \omega_{n } + \omega \right)t }{2} \right] \approx \sin\!\left(\omega_{n }t\right) 
\]
For the slow oscillation, the argument to the sinusoid becomes very small. We can then write the following approximation
\[
	2Y \sin \!\left[ \frac{\!\left( \omega_{n } - \omega \right)t }{2} \right] \approx Y \!\left( \omega_{n } - \omega \right)t 
\]
Taking the product of the approximations, we get
\[
	y \approx Y \!\left( \omega_{n } - \omega \right) t \sin\!\left(\omega_{n } t \right) 
\]
In the numerical example, by inspection we can see that \( \omega_{n } = 10 \) and \( \omega = 8 \), which generates the following graph of the product of oscillations:

\begin{center}
\includegraphics[width=1.0\textwidth]{../chap2/sec2.5/chap2sec2.5ex7.pdf}
\end{center}

But to see the resonance effects as \( \omega \rightarrow \omega_{n } \), consider \( \omega_{n } = 10 \) and \( \omega = 9.9 \). As the delta between them diminishes, the convergence to the resonance solution (\( y_{\text{approx}} \)) becomes apparent:

\begin{center}
\includegraphics[width=1.0\textwidth]{../chap2/sec2.5/chap2sec2.5ex7a.pdf}
\end{center}

% QUESTION 8
\qs{2.5.8}{What \( m  \), \( b  \), \( k  \), \( F  \) equation for a mass-dashpot-string-force corresponds to Kirchhoff's Voltage Law around a loop? What force balance equation on a mass corresponds to Kirchhoff's Current Law?}

Mass \( m  \) is akin to inductance \( L  \), as both are inertial quantities that resist changes in velocity or electric current, respectively. The damping constant \( b  \) is analogous to circuit resistance \( R  \), as both are measures of the subtractive forces acting on a system's total energy. Lastly, the spring constant \( k  \) serves the function as the reciprocal capacitance \( 1/C  \) -- where \( k  \) quantifies the stiffness of a spring (how hard or easy it is to stretch or compress it), reciprocal capacitance (sometimes called elastance) is a measure of how hard or easy it is to store charge.

In Kirchhoff's Voltage Law, the sum of voltage drops around a closed loop must vanish. It is also the Current Law integrated. The case of the Current Law is perhaps easier to conceptualize: the net sum of current flow at an ode must be zero. In the mechanical analogy, for a multinodal system, every node must have balanced forces.

If we analogize \( y \!\left( t  \right) \) (displacement) to the integral of the current (total distance the charge propagates):
\[
	y \!\left( t  \right) \sim \int I \!\left( t  \right) \, dt
\]
Then the Voltage Law corresponds to the mechanical equation
\[
	my'' + by' + ky = \textbf{applied force} 
\]

% QUESTION 9
\qs{2.5.9}{If you only know the natural frequency \( \omega_{n } \) and the damping coefficient \( b  \) for one mass and one spring, why is that \textit{not enough} to find the damped frequency \( \omega_{d } \)? If you know all of \( m  \), \( b  \), \( k \) what is \( \omega_{d } \)?}

The damped frequency is reliant on the damping ratio \( Z = b / 2 \sqrt{mk} \), namely
\[
	\omega_{d } = \omega_{n } \sqrt{1- Z^{2}} 
\]
Without \( k \), it is impossible to know \( \omega_{d } \).

\vspace{12pt}

% QUESTION 10
\qs{2.5.10}{Varying the number \( a  \) in a first order equation \( y' - ay = 1 \) changes the \textit{speed} of the response. Varying \( B  \) and \( C  \) in a second order equation \( y'' + By' + Cy = 1 \) changes the \textit{form} of the response. Explain the difference.}

The solution to the first order equation is
\[
	y \!\left( t  \right) =\frac{1 - e^{-at }}{a } 
\]
where varying \( a \) alters the rate at which \( y \!\left( t  \right) \) grows while maintaining the specific form of the response. In the case of the second order equation, altering \( B  \) and \( C  \) potentially changes the relationship between \( B^{2} \) and \( 4AC \), which criticaly determines the existence and degree of damping in the equation. Moreover, the natural frequency \( \omega_{n } = \sqrt{C/A } = \sqrt{C } \) is altered.

\vspace{12pt}

% QUESTION 11
\qs{2.5.11}{Find the step response \( r \!\left( t \right) = y_{p } + y_{n } \) for this overdamped system:
\[
	r'' + 2.5r' + r = 1 \text{ with } r \!\left( 0 \right) = 0 \text{ and } r'\!\left( 0 \right) = 0 
\]}

We find \( y_{n } \) and \( y_{p } \) to be
\[
	 r \!\left( t  \right) = 1 + c_{1}e^{-2t } + c_{2}e^{-t/2}
\]
Applying the initial conditions, the coefficients evaluate to
\begin{align*}
	r \!\left( 0 \right) & = 1 + c_{1} + c_{2} = 0 \implies \boxed{c_{1} = -\!\left( 1 + c_{2} \right)} \\
	r'\!\left( 0 \right) & = -2c_{1} - \frac{c_{2}}{2} = 0 \implies \boxed{c_{1} = -\frac{c_{2}}{4}} \\ 
\end{align*}

Leading to \( c_{1} = 1/3 \) and \( c_{2} = -4/3 \), and response
\[
	r \!\left( t  \right) = 1 + \frac{e^{-2t }}{3} - \frac{4e^{-t/2}}{3}
\]

% QUESTION 12
\qs{2.5.12}{Find the step response \( r \!\left( t  \right) = y_{p } + y_{n } \) for this critically damped system. The double root \( s = -1 \) produces what form for the null solution?
\[
	r'' + 2r' + r = 1 \text{ with } r \!\left( 0 \right) = 0 \text{ and } r'\!\left( 0 \right) = 0 
\]}

We have response
\[
	r \!\left( t  \right) = 1 + c_{1}e^{-t } + c_{2} te^{-t } 
\]
which features resonance from the double root. Applying the initial conditions gives us
\begin{align*}
	r \!\left( 0 \right) & = 1 + c_{1} = 0 \implies \boxed{c_{1} = -1} \\
	r'\!\left( 0 \right) & = 1 - c_{1} + c_{2} = 0 \implies \boxed{c_{2} = -2} \\ 
\end{align*}
This gives us final response
\[
	r \!\left( t  \right) = 1 -e^{-t } - 2te^{-t} 
\]

% QUESTION 13
\qs{2.5.13}{Find the step response \( r \!\left( t  \right) \) for this underdamped system using equation (22):
\[
	r'' + r' + r = 1 \text{ with } r \!\left( 0 \right) = 0 \text{ and } r'\!\left( 0 \right) = 0 
\]}

Using equation (22), we first identify \( p  \), \( \omega^{2}_{n } \), and \( \omega_{d } \). Matching the coefficients in the equation, we have
\[
	p = \frac{1}{2}, \quad \omega^{2}_{n } = 1, \quad \omega_{d } = \frac{\sqrt{3}}{2} 
\]
which corresponds to response
\[
	r \!\left( t  \right) = 1 - \frac{2}{\sqrt{3}}e^{-t/2} \sin\!\left( \frac{\sqrt{3}}{2} t + \frac{\pi }{3} \right)  
\]
Note that the phase shift \( \phi = \pi / 3 \) is calculated from \( \phi = \sin^{-1}\!\left( \sqrt{3}/2 \right) = \pi / 3 \).

\vspace{12pt}

% QUESTION 14
\qs{2.5.14}{Find the step response \( r \!\left( t  \right) \) for this undamped system and compare with (22):
\[
	r'' + r = 1 \text{ with } r \!\left( 0 \right) = 0 \text{ and } r'\!\left( 0 \right) = 0 
\]}

Without using equation (22), we identify the null and particular solutions as 
\[
	r \!\left( t  \right) = 1 + c_{1}e^{it } + c_{2}e^{-it } 
\]
Applying the initial conditions gives us
\begin{align*}
	r \!\left( 0 \right) & = 1 + c_{1} + c_{2} \implies \boxed{c_{1} = -\!\left( 1  + c_{2} \right)} \\
	r'\!\left( 0 \right) & = c_{1}i - c_{2}i = 0 \implies \boxed{c_{1} = c_{2}}
\end{align*}
Combined, we have \( c_{1} = c_{2} = -1/2 \), and solution
\[
	r \!\left( t  \right) = 1 - \cos\!\left( t \right)  
\]
Now, using equation (22) gives us parameters \( p = 0 \), \( \omega_{n } = 1 \), \( \omega_{d } = 1 \). From this, we calculate \( \sin\!\left(\phi \right) = 1 \implies \phi = \pi / 2 \), giving us the same solution:
\[
	r \!\left( t  \right) = 1 - \sin\!\left(t + \pi / 2 \right) = 1 - \cos\!\left( t \right) 
\]

% QUESTION 15
\qs{2.5.15}{For \( b^{2} < 4mk \) (underdamping), what parameter decides the speed at which the step response \( r \!\left( t  \right) \) rises to \( r \!\left( \infty \right) = 1 \)? Show that the \textbf{peak time} is \( T = \pi / \omega_{d } \) when \( r \!\left( t  \right) \) reaches its maximum before settling back to \( r = 1 \) at peak time \( r'\!\left( T \right) = 0 \).}

Since we have underdamping (namely, that damping exists), that establishes that \( p \neq 0 \). Then differentiating equation (22) and optimizing the derivative by setting equal to zero, we end up with the following condition:
\[
	\omega_{d } \cos\!\left(\omega_{d } t + \phi \right) = p \sin\!\left(\omega_{d }t + \phi \right)  
\]
Using the fact that \( \omega_{d } / p = \tan\!\left(\phi \right)  \), divide both sides by \( p  \) and the cosine term to derive the clean relation
\[
	\tan\!\left( \phi \right) = \tan\!\left( \omega_{d }t + \phi \right) 
\]
which is true if and only if we have
\[ 
	t = \frac{k \pi }{\omega_{d }} 
\]
For positive integer \( k  \), which presents to us an infinite number of solutions (technically, the oscillation persists forever, even if the amplitude is killed from the exponential decay with time). However, we can only have one solution for the maximum. So let's go with \( k = 1 \), which will be the earliest time as the outer exponential function decays. The peak time, as expected, is derived as 
\[
	T = \frac{\pi }{\omega_{d}} 
\]
The degree to which the response ``stabilizes'' around \( r \!\left( \infty \right) = 1 \) is contingent on \( p  \), which is the decay parameter in \( e^{-pt } \).

\vspace{12pt}

% QUESTION 16
\qs{2.5.16}{If the voltage source \( V \!\left( t  \right) \) in an RLC loop is a unit step function, what resistance \( R  \) will produce an overshoot to \( r_{\text{max}} = 1.2 \) if \( C = 10^{-6} \) Farads and \( L = 1 \) Henry? (Problem 15 found the peak time \( T  \) when \( r \!\left( T  \right) = r_{\text{max}} \)).

Sketch two graphs of \( r \!\left( t  \right) \) for \( p_{1} < p_{2} \). Sketch two graphs as \( \omega_{d } \) increases.}

From the Voltage Law, the differential equation we want to solve is
\[
	y'' + Ry' + 10^{6}y = 1 
\]
and we want \( r_{\text{max}} = 1.2 \). From the previous exercise, we determined that the corresponding time for this maximum response is \( T = \pi / \omega_{d } \). Writing the response in polar form, we have parameters
\[
	p = \frac{R }{2}, \quad \omega^{2}_{n } = 10^{6}, \quad \omega^{2}_{d } = 10^{6} - \frac{R^{2}}{4} \quad \!\left( \frac{1}{\omega^{2}_{d }} = \frac{4}{4 \cdot 10^{6} - R^{2}} \right) 
\]
Now, we substitute \( T  \) in our response function and isolate \( R  \):
\begin{alignat*}{4}
	&& r \!\left( T  \right) && = 1 - \frac{\omega_{n }}{\omega_{d }}e^{-R \pi / 2 \omega_{d }} \sin\!\left( \pi + \phi \right)  & = 1 + \frac{\omega_{n }}{\omega_{d }} e^{-R \pi / 2 \omega_{d }} \frac{\omega_{d }}{\omega_{n }} & \quad \!\left(\text{since } \sin\!\left(\phi \right) = \frac{\omega_{d }}{\omega_{n}} \right)  \\
	&& \implies && r_{\text{max}} = 1.2 & = 1 + e^{-R \pi / 2 \omega_{d}} & \\ 
	&& \implies && \ln\!\left(0.2\right) & = -\frac{R \pi }{2} \!\left( \frac{4}{4 \cdot 10^{6} - R^{2}} \right)^{1/2} & \\
	&& \implies && R^{2} & = \frac{ \ln\!\left(0.2\right)^{2} \cdot 4 \cdot 10^{6}}{\pi^{2} + \ln\!\left(0.2\right)^{2}} & \\
	&& \implies && R & \approx \boxed{912 \ \Omega}
\end{alignat*}

Graphically, \( R_{1} < R_{2} \) implies \( p_{1} < p_{2} \), but \( \omega_{n,1} \) and \( \omega_{n,2} \) are inversely inequal. The lower resistance \( R  \) features higher amplitudes in the oscillation of the response -- less energy is removed from the circuit system than in the case of higher resistance.

\vspace{12pt}

\begin{center}
\includegraphics[width=1.0\textwidth]{../chap2/sec2.5/chap2sec2.5ex16.pdf}
\end{center}

\vspace{12pt}

% QUESTION 17
\qs{2.5.17}{What values of \( m  \), \( b  \), \( k  \) will give the step response \( r \!\left( t  \right) = 1 - \sqrt{2}e^{-t } \sin\!\left(t + \frac{\pi }{4}\right) \) ?}

We have
\[
	\frac{\omega_{n }}{\omega_{d }} = \sqrt{2}, \quad p = 1, \quad \omega_{d } = 1 
\]
These relations in turn imply
\[
	\omega^{2}_{n} = \frac{k}{m } = 2, \quad \frac{b }{m } = 2
\]
One choice of parameters that satisfy these relations is \( m = 1 \) and \( b = k = 2 \).

\vspace{12pt}

% QUESTION 18
\qs{2.5.18}{What happens to the \( p - \omega_{d } - \omega_{n } \) right triangle as the damping ratio \( \omega_{n } / p  \) increases to 1 (critical damping)? At that point the damped frequency \( \omega_{d } \) becomes \underline{\qquad}. The step response becomes \( r \!\left( t  \right) = \underline{\qquad} \).}

We have damped frequency \( \omega_{d } = 0 \), which implies that we have repeated roots \( s_{1,2} = p  \) and \( p = \omega_{n } \). Because of resonance, our solution changes to be 
\[
	r \!\left( t  \right) = 1 + c_{1}e^{-pt } + c_{2}te^{-pt } 
\]
Applying the initial conditions \( r \!\left( 0 \right) \) and \( r'\!\left( 0 \right) \) give us coefficients \( c_{1} = -1 \) and \( c_{2} = -p  \). Assembling it all together, our solution is
\[
	r \!\left( t  \right) = 1 - e^{-pt } \!\left( 1 + pt \right) 
\]

% QUESTION 19
\qs{2.5.19}{\textbf{The roots \( \pmb{s_{1}, s_{2} = -p \pm i \omega_{d }} \) are poles of the transfer function \( \pmb{1 / \!\left( As^{2} + Bs + C  \right)} \)}

Show directly that the product of the roots \( s_{1} = -p + i \omega_{d } \) and \( s_{2} = -p - i \omega_{d } \) is \( s_{1}s_{2} = \omega^{2}_{n } \). The sum of the roots is \( -2p \). The quadratic equation with those roots is \( s^{2} + 2ps + \omega^{2}_{n } = 0 \).}

The product is
\[
	s_{1}s_{2} = \!\left( -p + i \omega_{d } \right)\!\left( -p - i \omega_{d } \right) = p^{2} + \omega^{2}_{d} = \omega^{2}_{n}
\]
The imaginary components cancel in the sum, giving us \( s_{1} + s_{2} = -2p \). These relations are Vieta's formulas for a quadratic, where the sum of the roots equals \( -B/A  \) and their product \( C/A  \).

\vspace{12pt}

% QUESTION 20
\qs{2.5.20}{Suppose \( p  \) is increased while \( \omega_{n } \) is held constant. How do the roots \( s_{1} \) and \( s_{2} \) move?}

The imaginary component \( \omega_{d } \) decreases in magnitude, meaning that the roots move along the circle of radius \( \omega_{n } \) on the complex plane towards the negative real axis at \(-p \rightarrow -\omega_{n } \).

\vspace{12pt}

% QUESTION 21
\qs{2.5.21}{Suppose the mass \( m  \) is increased while the coefficients \( b  \) and \( k \) are unchanged. What happens to the roots \( s_{1} \) and \( s_{2} \)?}

All else equal, increasing the mass \( m  \) means that the roots move inward onto a complex circle of smaller radius, approaching zero in the limit. Since the roots are expressed as 
\[
	s = -\frac{b }{2m} \pm \frac{\sqrt{b^{2} - 4mk}}{2m}
\]
We can see that increasing \( m  \) will eventually force the second term to be imaginary, even as the real and imaginary components in magnitude diminish. Importantly, \( \omega^{2}_{n} = k/m \) 

Intuitively, we imagine a heavy mass-dashpot-spring system. The natural frequency \( \omega_{n } \) will drastically decrease, indicating the increasingly slow oscillation of the mass. Since the damping coefficient is unchanged, the damping per unit mass is increasingly washed out with a heavier mass, indicating a proportionally lower amount of energy loss given an input forcing.

\vspace{12pt}

\newpage
% QUESTION 22
\qs{2.5.22}{\textbf{Ramp response} How could you find \( y \!\left( t  \right) \) when \( F = t  \) is a ramp function?

\[
	y'' + 2py' + \omega^{2}_{n }y = \omega^{2}_{n }t \text{ starting from } y \!\left( 0 \right) = 0 \text{ and } y'\!\left( 0 \right) = 0 
\]
A particular solution (straight line) is \( y_{p } =  \) \underline{\qquad}. The null solution still has the form \( y_{n} =  \) \underline{\qquad}. Find the coefficients \( c_{1} \) and \( c_{2} \) in the null solution from the two conditions at \( t = 0 \).

This ramp response \( y \!\left( t \right) \) can also be seen as the integral of \underline{\qquad}.}

The particular solution is \( y_{p } = t - 2p / \omega^{2}_{n} \). The null solution has form \( y_{n } = c_{1}e^{s_{1}t } + c_{2}e^{s_{2}t } \), where the roots are
\[
	s_{1,2} = -p \pm \sqrt{p^{2} - \omega^{2}_{n }} = -p \pm i \omega_{d} 
\]
This gives us complete solution
\[
	y \!\left( t  \right) = c_{1}e^{s_{1}t } + c_{2}e^{s_{2}t } + t - \frac{2p}{\omega^{2}_{n}}
\]
To find the coefficients, apply the initial conditions and derive
\begin{align*}
	y \!\left( 0 \right) & = c_{1} + c_{2} - \frac{2p}{\omega^{2}_{n}} = 0 \implies \boxed{c_{1} = \frac{2p}{\omega^{2}_{n}} - c_{2}} \\[12pt]
	y'\!\left( 0 \right) & = c_{1}s_{1} + c_{2}s_{2} + 1 = 0 \implies \boxed{c_{2} = \frac{1 + \frac{2p}{\omega^{2}_{n}}s_{1}}{s_{1} - s_{2}}}
\end{align*}
which then implies
\[
	\boxed{c_{1} = \frac{1 + \frac{2p}{\omega^{2}_{n}}s_{2}}{s_{2} - s_{1}} }
\]
The ramp response (slanted line) is the integral of the step response (horizontal line). 

\newpage
\subsection{\textsf{2.6 - Solutions to Second Order Equations}}

\textbf{Find a particular solution by inspection} (or the method of undetermined coefficients)

\vspace{12pt}

% QUESTION 1
\qs{2.6.1}{
\begin{enumerate}[label=(\alph*)]
	\item \( y'' + y = 4 \)
	\item \( y'' + y' = 4 \)
	\item \( y'' = 4 \)
\end{enumerate}}

\begin{enumerate}[label=(\alph*)]
	\item \( y_{p } = 4 \)
	\item \( y_{p } = 4t \)
	\item \( y_{p } = 2t^{2} \)
\end{enumerate}

% QUESTION 2
\qs{2.6.2}{
\begin{enumerate}[label=(\alph*)]
	\item \( y'' + y' + y = e^{t } \)
	\item \( y'' + y' + y = e^{ct } \)
\end{enumerate}}

\begin{enumerate}[label=(\alph*)]
	\item \( y_{p } = \frac{1}{3}e^{t } \)
	\item Substitute \( y_{p } = Ye^{ct } \) to find
	\[
	      c^{2}Y + cY + Y = 1 \implies Y = \frac{1}{c^{2} + c + 1}
	\]
\end{enumerate}

% QUESTION 3
\qs{2.6.3}{
\begin{enumerate}[label=(\alph*)]
	\item \( y'' - y  =\cos\!\left(t \right)  \)
	\item \( y'' + y = \cos\!\left(2t \right)  \)
	\item \( y'' + y = t + e^{t } \)
\end{enumerate}}

\begin{enumerate}[label=(\alph*)]
	\item Substitute \( y_{p } = Y\cos\!\left(t \right)  \) to find
	\[
		 -2Y \cos\!\left(t \right) = \cos\!\left(t \right) \implies Y = -\frac{1}{2}
	\]
	\item Substitute \( y_{p } = Y \cos\!\left(2t\right)  \) to find
	\[
		-3Y \cos\!\left(2t\right)  = \cos\!\left(2t\right) \implies Y = -\frac{1}{3}
	\]
	\item \( y_{p } = t + \frac{1}{2}e^{t} \)
\end{enumerate}

% QUESTION 4
\qs{2.6.4}{For these \( f \!\left( t  \right) \), predict the form of \( y \!\left( t  \right) \) with undetermined coefficients:
\begin{enumerate}[label=(\alph*)]
	\item \( f \!\left( t  \right) = t^{3} \)
	\item \( f \!\left( t  \right) = \cos\!\left(2t\right)  \)
	\item \( f \!\left( t  \right) = t \cos\!\left(t \right)  \)
\end{enumerate}}

\begin{enumerate}[label=(\alph*)]
	\item \( y_{p } \!\left( t  \right) = at^{3} + bt^{2} + ct + d  \)
	\item \( y_{p } \!\left( t  \right) = a \cos\!\left(2t\right) + b \sin\!\left(2t\right)  \)
	\item \( y_{p } \!\left( t  \right) = \!\left( at + b  \right) \cos\!\left(t \right) + \!\left( ct + d  \right) \sin\!\left(t \right) \)
\end{enumerate}

% QUESTION 5
\qs{2.6.5}{Predict the form for \( y \!\left( t  \right) \) when the right hand side is
\begin{enumerate}[label=(\alph*)]
	\item \( f \!\left( t  \right) = e^{ct } \)
	\item \( f \!\left( t  \right) = te^{ct } \)
	\item \( f \!\left( t  \right) = e^{t }\cos\!\left(t \right) \)
\end{enumerate}}

\begin{enumerate}[label=(\alph*)]
	\item \( y_{p } \!\left( t  \right) = Ye^{ct } \)
	\item \( y_{p } \!\left( t  \right) = \!\left( at + b  \right)e^{ct } \)
	\item \( y_{p } \!\left( t  \right) = e^{t } \!\left( a \cos\!\left(t \right) + b \sin\!\left(t \right)  \right) \)
\end{enumerate}

% QUESTION 6
\qs{2.6.6}{For \( f \!\left( t  \right) = e^{ct } \) when is the prediction for \( y \!\left( t  \right) \) different from \( Ye^{ct } \)?}

If \( e^{ct } \) is one of the null solutions, we must have a \( Yte^{ct } \) term to account for the resonance.

\vspace{12pt}

\textbf{Use the method of undetermined coefficients to find a solution \( \pmb{y_{p } \!\left( t  \right)} \).}

\vspace{12pt}

% QUESTION 7
\qs{2.6.7}{
\begin{enumerate}[label=(\alph*)]
	\item \( y'' + 9y = e^{2t} \)
	\item \( y'' + 9y = te^{2t} \)
\end{enumerate}}

\begin{enumerate}[label=(\alph*)]
	\item Let \( y_{p } \!\left( t  \right) = Ye^{2t} \). Then
	\[
		4Ye^{2t} + 9Ye^{2t} = e^{2t} \implies Y = \frac{1}{13} \implies y_{p } \!\left( t  \right) = \frac{e^{2t}}{13}
	\]
	\item Let \( y_{p } \!\left( t  \right) = \!\left( at + b  \right)e^{2t} \). Then 
	\begin{align*}
		y_{p }' \!\left( t  \right) & = ae^{2t} + 2 \!\left( at + b  \right)e^{2t} \\
					    & = \!\left( 2at + a + 2b \right)e^{2t} \\ 
		y_{p }''\!\left( t  \right) & = 2ae^{2t} + 2 \!\left( 2at + a + 2b \right)e^{2t} \\
					    & = \!\left( 4at + 4a + 4b \right)e^{2t}
	\end{align*}
	Substituting into the equation, we derive
	\[
		\!\left( 4at + 4a + 4b \right)e^{2t} + 9 \!\left( at + b  \right)e^{2t} = \!\left( 13at  + 4a + 13b \right)e^{2t} = te^{2t}
	\]
	Here, we can deduce that \( a = \displaystyle{\frac{1}{13}} \) and \( b = \displaystyle{-\frac{4}{13}\frac{1}{13}} \). Our particular solution is
	\[
		y_{p } \!\left( t  \right) = \frac{1}{13}\!\left( t - \frac{4}{13} \right)e^{2t} 
	\]
\end{enumerate}

% QUESTION 8
\qs{2.6.8}{
\begin{enumerate}[label=(\alph*)]
	\item \( y'' + y' = t + 1 \)
	\item \( y'' + y' = t^{2} + 1 \)
\end{enumerate}}

\begin{enumerate}[label=(\alph*)]
	\item Observe that the roots of the characteristic polynomial are \( s = 0, -1 \). Then one of the null solutions is simply a constant, so that cannot be present in our particular solution. Let \( y_{p } \!\left( t  \right) = at^{2} + bt \) (no constant term). Then
		\begin{align*}
			y'\!\left( t  \right) & = 2at + b  \\
			y''\!\left( t  \right) & = 2a
		\end{align*}
	Substituting, we get
	\[
		2at + \!\left( 2a + b  \right) = t + 1 
	\]
	Where we can conclude \( a = 1/2 \) and \( b = 0 \), and
	\[
		y_{p } \!\left( t  \right) = \frac{1}{2}t^{2} 
	\]
	\item The equation is the same but the forcing changes, so we have ansatz \( y_{p } \!\left( t  \right) = at^{3} + bt^{2} + ct  \) (again, no constant term). Then
		\begin{align*}
			y'\!\left( t  \right) & = 3at^{2} + 2bt + c  \\[12pt]
			y''\!\left( t  \right) & = 6at + 2b
		\end{align*}
	We substitute to get
	\[
		3at^{2} + \!\left( 6a + 2b \right)t + 2b + c = t^{2} + 1 
	\]
	Concluding that \( a = 1/3 \), \( b = -1 \), and \( c = 2 \). This corresponds to particular solution
	\[
		y_{p }\!\left( t  \right) = \frac{1}{3}t^{3} - t^{2} + 2t 
	\]
\end{enumerate}

% QUESTION 9
\qs{2.6.9}{
\begin{enumerate}[label=(\alph*)]
	\item \( y'' + 3y = \cos\!\left(t \right)  \)
	\item \( y'' + 3y = t \cos\!\left(t \right)  \)
\end{enumerate}}

\begin{enumerate}[label=(\alph*)]
	\item Let \( y_{p } \!\left( t  \right) = a \cos\!\left(t \right) + b \sin\!\left(t \right)  \). Substituting gives us
	\[
		2a \cos\!\left(t \right) + 2b \sin\!\left(t \right) = \cos\!\left(t \right)  
	\]
	From which we can deduce \( a = 1/2 \) and \( b = 0 \), yielding solution
	\[
		y_{p } \!\left( t  \right) = \frac{1}{2}\cos\!\left(t \right)  
	\]
	\item Let \( y_{p } \!\left( t  \right) = \!\left( at + b  \right)\cos\!\left(t \right) + \!\left( ct + d  \right) \sin\!\left(t \right)   \). Then
	\begin{align*}
		y'_{p} \!\left( t  \right) & = a \cos\!\left(t \right) - \!\left( at + b  \right) \sin\!\left(t \right) +c \sin\!\left(t \right) + \!\left( ct + d  \right) \cos\!\left(t \right) \\
					   & = \!\left( ct + a + d  \right) \cos\!\left(t \right) + \!\left( -at - b + c  \right) \sin\!\left(t \right) \\
		y''_{p } \!\left( t  \right) & = c \cos\!\left(t \right) - \!\left( ct + a + d  \right) \sin\!\left(t \right) - a \sin\!\left(t \right) + \!\left( -at - b + c  \right) \cos\!\left(t \right) \\
					     & = \!\left( -at - b + 2c \right) \cos\!\left(t \right) - \!\left( ct + 2a + d \right) \sin\!\left(t \right) 
	\end{align*}
	Substituting gives us
	\[
		 \!\left( 2at + 2b + 2c \right) \cos\!\left(t \right) + \!\left( 2ct - 2a + 2d \right) \sin\!\left(t \right) = t \cos\!\left(t \right) 
	\]
	We must have \( a = d - 1/2 \) and \( b = c = 0 \), which corresponds to particular solution
	\[
		y_{p } \!\left( t  \right) = \frac{1}{2}t \cos\!\left(t \right) + \frac{1}{2} \sin\!\left(t \right)  
	\]
\end{enumerate}

% QUESTION 10
\qs{2.6.10}{
\begin{enumerate}[label=(\alph*)]
	\item \( y'' + y' + y = t^{2} \)
	\item \( y'' + y' + y = t^{3} \)
\end{enumerate}}

\begin{enumerate}[label=(\alph*)]
	\item Let \( y_{p } \!\left( t  \right) = at^{2} + bt + c  \). Then \( y_{p}'\!\left( t \right) = 2at + b  \) and \( y_{p }''\!\left( t  \right) = 2a \). Substituting yields
	\[
		at^{2} + \!\left( 2a + b  \right)t + 2a + b + c = t^{2} 
	\]
	where we can deduce that \( a = 1 \), \( b = -2 \), and \( c = 0 \), and solution
	\[
		y_{p } \!\left( t  \right) = t^{2} - 2t
	\]
	\item Let \( y_{p } \!\left( t  \right) = at^{3} + bt^{2} + ct + d  \). Then \( y_{p }'\!\left( t  \right) = 3at^{2} + 2bt + c  \) and \( y_{p }''\!\left( t  \right) = 6at + 2b \). Substituting yields
	\[
		 at^{3} + \!\left( 3a + b  \right)t^{2} + \!\left( 6a + 2b + c  \right)t + 2b + c + d = t^{3}
	\]
	which gives us coefficients \( a = 1 \), \( b = -3 \), \( c = 0 \), and \( d = 6 \). We have particular solution
	\[
		y_{p } \!\left( t  \right) = t^{3} - 3t^{2} + 6 
	\]
\end{enumerate}

% QUESTION 11
\qs{2.6.11}{
\begin{enumerate}[label=(\alph*)]
	\item \( y'' + y' + y = \cos\!\left(t \right)  \)
	\item \( y'' + y' + y = t \sin\!\left(t \right)  \)
\end{enumerate}}

\begin{enumerate}[label=(\alph*)]
	\item Let \( y_{p } \!\left( t  \right) = a \cos\!\left(t \right) + b \sin\!\left(t \right)  \). Then \( y_{p }'\!\left( t  \right) = -a \sin\!\left(t \right) + b \cos\!\left(t \right)  \) and \( y_{p }''\!\left( t  \right) = -a \cos\!\left(t \right) - b \sin\!\left(t \right)  \). Substituting gives us
	\[
		-a \sin\!\left(t \right) + b \cos\!\left(t \right) = \cos\!\left(t \right)  
	\]
	which has \( a = 0 \) and \( b = 1 \), with solution \( y_{p } \!\left( t  \right) = \sin\!\left(t \right)  \).
	\item Let \( y_{p } \!\left( t  \right) = \!\left( at + b  \right)\cos\!\left(t \right) + \!\left( ct + d  \right)\sin\!\left(t \right)  \). Then
	\begin{align*}
		y_{p }'\!\left( t  \right) & = a \cos\!\left(t \right) - \!\left( at + b  \right)\sin\!\left(t \right) + c \sin\!\left(t \right) + \!\left( ct + d  \right)\cos\!\left(t \right)  \\
					   & = \!\left( ct + a + d  \right)\cos\!\left(t \right) + \!\left( -at - b + c  \right)\sin\!\left(t \right) \\
		y_{p }''\!\left( t  \right) & = c \cos\!\left(t \right) - \!\left( ct + a + d  \right) \sin\!\left(t \right) - a \sin\!\left(t \right) + \!\left( -at - b + c  \right) \cos\!\left(t \right)  \\ 
					    & = \!\left( -at - b + 2c \right) \cos\!\left(t \right) + \!\left( -ct - 2a - d  \right) \sin\!\left(t \right) 
	\end{align*}
	Substituting gives us
	\[
		\!\left( ct + a + 2c + d  \right) \cos\!\left(t \right) + \!\left( -at - b - 2a + c  \right) \sin\!\left(t \right) = t \sin\!\left(t \right)  
	\]
	This forces \( c = 0 \), \( a = -1 \), \( d = 1 \), and \( b = 2 \), which corresponds to solution
	\[
		y_{p } \!\left( t  \right) = \!\left( -t + 2 \right)\cos\!\left(t \right) + \sin\!\left(t \right)  
	\]
\end{enumerate}

\textbf{Problems 12-14 involve resonance. Multiply the usual form of \( \pmb{y_{p}} \) by \( \pmb{t} \).}

\vspace{12pt}

% QUESTION 12
\qs{2.6.12}{
\begin{enumerate}[label=(\alph*)]
	\item \( y'' + y = e^{it } \)
	\item \( y'' + y = \cos\!\left(t \right)  \)
\end{enumerate}}

\begin{enumerate}[label=(\alph*)]
	\item Let \( y_{p } \!\left( t  \right) = Yte^{it } \). Then \( y_{p }'\!\left( t  \right) = Y \!\left( e^{it } + ite^{it } \right) = Y \!\left( 1 + it  \right)e^{it } \) and \( y_{p }''\!\left( t  \right) = Y \!\left( ie^{it } + \!\left( i - t  \right)e^{it } \right) = Y \!\left( 2i - t  \right)e^{it} \). Substituting gives us
	\[
		2iYe^{it } = e^{it } \implies Y = -\frac{i }{2} 
	\]
	which corresponds to solution
	\[
		y_{p } \!\left( t  \right) = -\frac{i }{2}te^{it } 
	\]
	\item Let \( y_{p } \!\left( t  \right) = t \!\left( a \cos\!\left(t \right) + b \sin\!\left(t \right)  \right) \). Then
	\begin{align*}
		y_{p }'\!\left( t  \right) & = a \cos\!\left(t \right) + b \sin\!\left(t \right) + t \!\left( -a \sin\!\left(t \right) + b \cos\!\left( t \right)  \right) \\
					   & = \!\left( bt + a  \right)\cos\!\left(t \right) + \!\left( -at + b  \right)\sin\!\left(t \right) \\
		y_{p }''\!\left( t  \right) & = b \cos\!\left(t \right) - \!\left( bt + a  \right) \sin\!\left(t \right) - a \sin\!\left(t \right) + \!\left( -at + b  \right) \cos\!\left(t \right) \\ 
					    & = \!\left( -at + 2b \right) \cos\!\left(t \right) - \!\left( bt + 2a \right) \sin\!\left( t \right)
	\end{align*}
	Substituting gives us
	\[
		2b \cos\!\left(t \right) - 2a \sin\!\left(t \right) = \cos\!\left(t \right)  
	\]
	Forcing \( a = 0 \) and \( b = 1/2 \), and solution
	\[
		y_{p } \!\left( t  \right) = \frac{1}{2}t \sin\!\left(t\right) 
	\]
\end{enumerate}

% QUESTION 13
\qs{2.6.13}{
\begin{enumerate}[label=(\alph*)]
	\item \( y'' - 4y' + 3y = e^{t } \)
	\item \( y'' - 4y' + 3y = e^{3t} \)
\end{enumerate}}

\begin{enumerate}[label=(\alph*)]
	\item Let \( y_{p } \!\left( t  \right) = Yte^{t } \). Then \( y_{p }'\!\left( t  \right) = Y \!\left( e^{t } + te^{t } \right) \) and \( y_{p }''\!\left( t  \right) = Y \!\left( 2e^{t } + te^{t} \right) \). Substituting gives us
	\[
		-2Ye^{t } = e^{t } \implies Y = -\frac{1}{2}
	\]
	which corresponds to solution
	\[
		y_{p } \!\left( t  \right) = -\frac{1}{2}te^{t} 
	\]
	\item Let \( y_{p } \!\left( t  \right) = Yte^{3t} \). Then \( y_{p }'\!\left( t  \right) = Y \!\left( e^{3t} + 3te^{3t} \right) \) and \( y_{p }''\!\left( t  \right) = Y \!\left( 6e^{3t} + 9te^{3t} \right) \). Substituting gives us
	\[
		2Ye^{3t} = e^{3t} \implies Y = \frac{1}{2}
	\]
	which corresponds to solution
	\[
		y_{p } \!\left( t  \right) = \frac{1}{2}te^{3t} 
	\]
\end{enumerate}

% QUESTION 14
\qs{2.6.14}{
\begin{enumerate}[label=(\alph*)]
	\item \( y' - y = e^{t } \)
	\item \( y' - y = te^{t } \)
	\item \( y' - y = e^{t } \cos\!\left(t\right)  \)
\end{enumerate}}

\begin{enumerate}[label=(\alph*)]
	\item Let \( y_{p } \!\left( t  \right) = Yte^{t } \). Then \( y_{p }'\!\left( t  \right) = Y \!\left( e^{t } + te^{t } \right) \). Then we substitute to derive
	\[
		Ye^{t } = e^{t } \implies Y = 1 
	\]
	Then we have \( y_{p } \!\left( t  \right) = te^{t } \).
	\item Let \( y_{p } \!\left( t  \right) = Yt^{2}e^{t } \). Then \( y_{p }'\!\left( t  \right) = Y \!\left( t^{2}e^{t } + 2te^{t } \right) \). We substitute to get
	\[
		2Yte^{t } = te^{t } \implies Y = \frac{1}{2} 
	\]
	Our solution is
	\[
		y_{p } \!\left( t  \right) = \frac{1}{2}t^{2}e^{t }
	\]
	\item Let \( y_{p }\!\left( t  \right) = e^{t } \!\left( a \cos\!\left(t \right) + b \sin\!\left(t \right)  \right) \). Then 
	\begin{align*}
		y_{p}'\!\left( t \right) & = e^{t}\!\left[ \!\left( a \cos\!\left(t \right) + b \sin\!\left(t \right)  \right) + \!\left( -a \sin\!\left(t \right) + b \cos\!\left(t \right)  \right) \right] \\
					 & = e^{t } \!\left[ \!\left( a + b  \right)\cos\!\left(t \right) + \!\left( b - a  \right) \sin\!\left(t \right)  \right]
	\end{align*}
	We substitute to get
	\[
		e^{t } \!\left( b \cos\!\left(t \right) - a \sin\!\left(t \right)  \right) = e^{t } \cos\!\left(t \right)  \implies a = 0, b = 1
	\]
	Therefore our solution is
	\[
		y_{p }\!\left( t  \right) = e^{t }\sin\!\left(t\right)  
	\]
\end{enumerate}

% QUESTION 15
\qs{2.6.15}{For \( y'' - 4y = e^{t }\sin\!\left(t \right)  \) (exponential times sinusoidal) we have two choices:
\begin{alignat*}{2}
	\textbf{1} \quad & \text{(Real)} && \quad \text{Substitute } y_{p } = Me^{t } \cos\!\left(t \right) + Ne^{t }\sin\!\left(t \right): \\
		   & 		   && \quad \text{determine } M \text{ and } N \\
	\textbf{2} \quad & \text{(Complex)} && \quad \text{Solve } z'' + 4z = e^{\!\left( 1 + i  \right)t }. \\
		   &                  && \quad \text{Then } y \text{ is the imaginary part of } z. 
\end{alignat*}
Use both methods to find the same \( y \!\left( t  \right) \) -- which do you prefer?}

\begin{enumerate}[label=\arabic*.]
	\item Let \( y_{p } \) be as given. Then
	\begin{align*}
		y'_{p } & = Me^{t} \!\left( \cos\!\left(t \right) - \sin\!\left(t \right)  \right) + Ne^{t} \!\left( \sin\!\left(t \right) + \cos\!\left(t\right)  \right)  \\
		y''_{p} & = -2Me^{t} \sin\!\left(t \right) + 2Ne^{t} \cos\!\left(t\right) 
	\end{align*}
	Substituting gives us
	\[
		e^{t } \!\left[ \!\left( 2N - 4M \right)e^{t }\cos\!\left(t \right) - \!\left( 2M + 4N \right)\sin\!\left(t\right) \right] = e^{t }\sin\!\left(t\right) 
	\]
	We must have \( -4M + 2N = 0 \) and \( -\!\left( 2M + 4N \right) = 1 \). Solve the matrix equation:
	\[
		\begin{bmatrix}
			-4 & 2\\
			-2 & -4\\
		\end{bmatrix}
		\begin{bmatrix}
			M \\
			N \\
		\end{bmatrix} = 
		\begin{bmatrix}
			1 \\
			0 \\
		\end{bmatrix}
	\]
	Which gives us \( M = -1/5 \) and \( N = 1/10 \).
	\item In the complex case, let \( y_{p } = Ye^{\!\left( 1 + i  \right)t} \). Then \( y'_{p } = Y \!\left( 1 + i  \right)e^{\!\left( 1 + i  \right)t } \) and \( y''_{p } = Y \!\left( 1 + i  \right)^{2}e^{\!\left( 1 + i  \right)t } \), and our task reduces to isolating \( Y  \) in the equation
	\[
		Y \!\left( 1 + i  \right)^{2} - 4Y \!\left( 1 + i  \right) = 1 
	\]
	Doing so yields
	\[
		Y = -\frac{1}{4 + 2i} = \frac{-4 + 2i}{20} = \frac{-2 + i }{10}
	\]
	Our particular solution is
	\begin{align*}
		y_{p } & = \!\left( \frac{-2 + i}{10} \right) e^{t} \!\left( \cos\!\left(t \right) + i \sin\!\left(t\right)  \right) \\
		       & = -\frac{1}{5} e^{t }\cos\!\left(t \right) + \frac{1}{10}e^{t }\sin\!\left(t \right) + i \!\left[ \frac{1}{10}e^{t }\cos\!\left(t \right) - \frac{1}{5}e^{t }\sin\!\left(t\right)  \right]
	\end{align*}
	Taking the real component, we see that it matches with what we had in part 1.
\end{enumerate}

% QUESTION 16
\qs{2.6.16}{
\begin{enumerate}[label=(\alph*)]
	\item Which values of \( c  \) give resonance for \( y'' + 3y' - 4y = te^{ct } \)?
	\item What form would you substitute for \( y \!\left( t  \right) \) if there is no resonance?
	\item What form would you use when \( c  \) produces resonance?
\end{enumerate}}

\begin{enumerate}[label=(\alph*)]
	\item The characteristic polynomial is \( s^{2} + 3s - 4 \) which has roots \( s_{1} = -4 \) and \( s_{2} = 1 \), so those values of \( c  \) will give resonance. This is because our ansatz, when of the form \( y_{p } = \!\left( at + b  \right)e^{ct } \), includes the form of the null solutions.
	\item If no resonance, then we can safely put forth our ansatz with the form \( y_{p }= \!\left( at + b  \right)e^{ct } \).
	\item Otherwise, use form \( y_{p } = \!\left( at^{2} + bt  \right)e^{ct } \).
\end{enumerate}

% QUESTION 17
\qs{2.6.17}{This is the rule for equations \( P \!\left( D  \right)y = e^{ct } \) with resonance \( P \!\left( c  \right) = 0 \):

If \( P \!\left( c  \right) = 0 \) and \( P'\!\left( c  \right) \neq 0 \), look for a solution \( y_{p } = Cte^{ct } \) (\( m = 1 \))

If \( c  \) is a root of multiplicity \( m  \), then \( y_{p } \) has the form \underline{\qquad}.}

Then \( y_{p } \) has form \( Ct^{m }e^{ct } \).

\vspace{12pt}

\newpage
% QUESTION 18
\qs{2.6.18}{
\begin{enumerate}[label=(\alph*)]
	\item To solve \( d^{4}y / dt^{4} - y = t^{3}e^{5t} \), what form do you expect for \( y \!\left( t  \right) \)?
	\item If the right side becomes \( t^{3}\cos\!\left(5t\right)  \), which 8 coefficients are to be determined?
\end{enumerate}}

\begin{enumerate}[label=(\alph*)]
	\item The characteristic polynomial is \( s^{4} - 1 \), which gives us roots \( s = \pm 1, \pm i  \). Since \( c = 5 \), we need not concern ourselves with resonance, and we guess \( y_{p } = \!\left( at^{3} + bt^{2} + ct + d  \right)e^{5t} \).
	\item In this case, we then determine the number of combinations between sine and cosine and the four terms of a third-degree polynomial:
	\begin{alignat*}{3}
		t^{3}\cos\!\left(5t \right) & \quad t^{2}\cos\!\left(5t \right) & \quad t\cos\!\left(5t \right) & \quad \cos\!\left(5t \right)  \\
		t^{3}\sin\!\left(5t \right) & \quad t^{2}\sin\!\left(5t \right) & \quad t \sin\!\left(5t \right) & \quad \sin\!\left(5t \right)  \\ 
	\end{alignat*}
\end{enumerate}

% QUESTION 19
\qs{2.6.19}{For \( y' - ay = f \!\left( t  \right) \), the method of undetermined coefficients is looking for all \( f \!\left( t  \right) \) so that the usual formula \( y_{p } = e^{at } \int e^{-as } f \!\left( s  \right) \, ds  \) is easy to integrate. Find these integrals for the ``nice functions'' \( f = e^{ct } \), \( f = e^{i \omega t } \), and \( f = t  \):
\[
	\int e^{-as }e^{cs}  \, ds \qquad \int e^{-as } e^{i \omega s } \, ds \qquad \int e^{-as }s \, ds
\]}

\begin{enumerate}[label=(\alph*)]
	\item 
	\[
		\int e^{-as }e^{cs } \, ds = \int e^{\!\left( -a + c  \right)s } \, ds = \frac{e^{\!\left( -a + c  \right)s }}{-a + c } + C
	\]
	\item 
	\[
		\int e^{-as }e^{i \omega s } \, ds = \int e^{\!\left( -a + i \omega  \right)s } \, ds = \frac{e^{\!\left( -a + i \omega  \right)s }}{-a + i \omega } + C
	\]
	\item 
	Resort to integration by parts. Let \( u = s  \) and \( dv = e^{-as } ds  \). Then \( du = ds  \) and \( v = -e^{-as } / a \).
	\[
		\int e^{-as }s \, ds = -\frac{e^{-as }s }{a } + \frac{1}{a } \int e^{-as } \, ds = -e^{-as } \!\left( \frac{s }{a } + \frac{1}{a^{2}} \right)
	\]
\end{enumerate}

\vspace{12pt}

\newpage
\textbf{Problems 20-27 develop the method of variation of parameters.}

\vspace{12pt}

% QUESTION 20
\qs{2.16.20}{Find two solutions \( y_{1} \), \( y_{2} \) to \( y'' + 3y' + 2y = 0 \). Use those in formula (13) to solve
\begin{enumerate}[label=(\alph*)]
	\item \( y'' + 3y' + 2y = e^{t } \)
	\item \( y'' + 3y' + 2y = e^{-t } \)
\end{enumerate}}

\begin{enumerate}[label=(\alph*)]
	\item The roots to the characteristic polynomial are \( s_{1,2} = -2, -1 \). Then we have null solutions \( y_{1} = e^{-2t} \) and \( y_{2} = e^{-t } \), and correspondingly \( y'_{1} = -2e^{-2t} \) and \( y'_{2} = -e^{-t } \). The Wronskian is calculated as 
\[
	W \!\left( t  \right) = -e^{-3t} + 2e^{-3t} = e^{-3t}
\]
By the variation of parameters, the particular solution is
\begin{align*}
	y_{p } \!\left( t  \right) & = - e^{-2t} \int e^{3t} \, dt + e^{-t } \int e^{2t } \, dt \\
	 & = -\frac{e^{t }}{3} + \frac{e^{t }}{2} \\ 
	 & = \frac{e^{t }}{6}
\end{align*}
	\item In the second case, we have resonance, since \( c = -1 \) which is one of our roots. The particular solution becomes
	\begin{align*}
		y_{p }\!\left( t  \right) & = -e^{-2t} \int e^{t } \, dt + e^{-t } \int \, dt  \\
		 & = e^{-t } \!\left( t - 1 \right)
	\end{align*}
\end{enumerate}

% QUESTION 21
\qs{2.6.21}{Find two solutions to \( y'' + 4y' = 0 \) and use variation of parameters for 
\begin{enumerate}[label=(\alph*)]
	\item \( y'' + 4y' = e^{2t} \)
	\item \( y'' + 4y' = e^{-4t} \)
\end{enumerate}}

\begin{enumerate}[label=(\alph*)]
	\item The characteristic polynomial has roots \( s_{1,2} = 0, -4 \). Then we have null solutions \( y_{1} = 1  \) and \( y_{2} = e^{-4t} \), and correspondingly \( y'_{1} = 0 \) and \( y'_{2} = -4e^{-4t} \). The Wronksian is
\[
	W \!\left( t  \right) = -4e^{-4t} 
\]
Variation of parameters gives us particular solution
\begin{align*}
	y_{p }\!\left( t  \right) & = \frac{1}{4}\int e^{2t} \, dt - \frac{e^{-4t}}{4} \int e^{6t} \, dt \\
				  & = \!\left( \frac{1}{8} - \frac{1}{24} \right)e^{2t} \\
				  & = \frac{e^{2t}}{12}
\end{align*}
	\item Again, resonance presents itself. Calculate
	\[
		y_{p }\!\left( t  \right) = \frac{1}{4} \int e^{-4t} \, dt - \frac{e^{-4t}}{4} \int dt = -e^{-4t} \!\left( \frac{t }{4} + \frac{1}{16} \right)
	\]
\end{enumerate}

% QUESTION 22
\qs{2.6.22}{Find an equation \( y'' + By' + Cy = 0 \) that is solved by \( y_{1} = e^{t } \) and \( y_{2} = te^{t } \). If the right side is \( f \!\left( t  \right) = 1 \), what solution comes from the \( VP \) formula (13)?}

The null solutions imply a double root \( s = 1 \), so we have \( B = 2 \) and \( C = 1 \) for 
\[
	 y'' + 2y' + 1 = 0
\]
The derivatives of the null solutions are \( y'_{1} = e^{t } \) and \( y'_{2} = e^{t } \!\left( 1 + t \right) \). The Wronskian is
\[
	W \!\left( t  \right) = e^{2t}\!\left( 1 + t  \right) - te^{2t} = e^{2t}
\]
and the particular solution is
\begin{align*}
	y_{p }\!\left( t  \right) & = -e^{t } \int te^{-t } \, dt + te^{t } \int e^{-t } \, dt \\
				  & = t + 1 - t \\
				  & = 1
\end{align*}
We could have come to this conclusion on inspection, but good to see that variation of parameters brings us to this result nonetheless.

\vspace{12pt}

% QUESTION 23
\qs{2.6.23}{\( y'' - 5y' + 6y = 0 \) is solved by \( y_{1} = e^{2t} \) and \( y_{2} = e^{3t} \), because \( s = 2 \) and \( s = 3 \) come from \( s^{2} - 5s + 6 = 0 \). Now solve \( y'' - 5y' + 6y = 12 \) in two ways:

\textbf{1.} Undetermined coefficients (or inspection)

\textbf{2.} Variation of parameters using (13)

The answers are different. Are the initial conditions different?}

\begin{enumerate}[label=\arabic*.]
	\item By inspection, \( y_{p } = 2 \). Our complete solution is
	\[
		y \!\left( t  \right) = c_{1}e^{2t} + c_{2}e^{3t} + 2 
	\]
	\item The Wronskian is calculated as 
	\[
		W \!\left( t  \right) = 3e^{5t} - 2e^{5t} = e^{5t} 
	\]
	Our particular solution is
	\[
		y_{p } \!\left( t  \right) = -e^{2t} \int 12 e^{-2t} \, dt + e^{3t} \int 12e^{-3t} \, dt = 2
	\]
	Both deriving the particular solution by inspection and variation of parameters lead to the same result, but if we integrate from \( 0 \) to \( t  \) to incorporate initial conditions, we see
	\begin{align*}
		y \!\left( t \right) & = -e^{2t} \int_{0}^{t} 12 e^{-2s} \, ds + e^{3t} \int_{0}^{t} 12e^{-3s} \, ds \\
		 & = 6e^{2t} \!\left( e^{-2t} - 1 \right) - 4e^{3t} \!\left( e^{-3t} - 1 \right) \\ 
		 & = 2 - 6e^{2t} + 4e^{3t}
	\end{align*}
	Which is the full solution -- including the null solutions -- implying \( y \!\left( 0 \right) = y'\!\left( 0 \right) = 0 \).
\end{enumerate}

% QUESTION 24
\qs{2.6.24}{What are the initial conditions \( y \!\left( 0 \right) \) and \( y'\!\left( 0 \right) \) for the solution (13) coming from the variation of parameters, starting from any \( y_{1} \) and \( y_{2} \)?}

We can see that \( y \!\left( 0 \right) = 0 \), as both terms cancel each other out at \( t = 0 \). To find the time derivative, use the dummy variable \( s  \) inside the integral and apply Leibniz's rule:
\begin{align*}
	y'\!\left( t  \right) & = \frac{d }{d t } \!\left[ -y_{1}\!\left( t  \right) \int_{0}^{t } \frac{y_{2}\!\left( s  \right)f \!\left( s  \right)}{W \!\left( s  \right)} \, ds + y_{2}\!\left( t  \right) \int_{0}^{t } \frac{y_{1}\!\left( s  \right) f \!\left( s  \right)}{W \!\left( s  \right)} \, ds \right] \\
	 & = -y'_{1}\!\left( t  \right) \int_{0}^{t } \frac{y_{2} \!\left( s  \right) f \!\left( s  \right)}{W \!\left( s  \right)} \, ds - \frac{y_{1}\!\left( t  \right) y_{2}\!\left( t  \right) f \!\left( t  \right)}{W \!\left( t  \right)}  \\ 
	 & \qquad + y'_{2} \!\left( t  \right) \int_{0}^{t } \frac{y_{1}\!\left( s  \right) f \!\left( s  \right)}{W \!\left( s  \right)} \, ds + \frac{y_{1} \!\left( t  \right) y_{2} \!\left( t  \right) f \!\left( t  \right)}{W \!\left( t  \right)} \\
	\implies y'\!\left( 0 \right) & = 0
\end{align*}
The initial position and velocity are assumed to be zero.

\vspace{12pt}

% QUESTION 25
\qs{2.6.25}{The equation \( y'' = 0 \) is solved by \( y_{1} = 1 \) and \( y_{2} = t  \). Use variation of parameters to solve \( y'' = t  \) and also \( y'' = t^{2} \).}

We have derivatives \( y'_{1} = 0 \) and \( y'_{2} = 1 \), and Wronskian
\[
	W \!\left( t  \right) = 1 
\]
The particular solution for \( y'' = t  \) is derived as 
\[
	y_{p }\!\left( t  \right) = -\int t^{2} \, dt + t \int t \, dt = -\frac{t^{3}}{3} + \frac{t^{3}}{2} = \frac{t^{3}}{6}
\]
And for \( y'' = t^{2} \) we have
\[
	y_{p }\!\left( t  \right) = - \int t^{3} \, dt + t \int t^{2} \, dt = -\frac{t^{4}}{4}+ \frac{t^{4}}{3} = \frac{t^{4}}{12}
\]

% QUESTION 26
\qs{2.6.26}{Solve \( y''_{s} + y_{s } = 1 \) for the step response using variation of parameters, starting from the null solutions \( y_{1} = \cos\!\left(t \right)  \) and \( y_{2} = \sin\!\left(t \right)  \).}

We have \( y'_{1} = -\sin\!\left(t \right)  \) and \( y'_{2} = \cos\!\left(t \right)  \), and Wronskian
\[
	W \!\left( t  \right) = \sin^{2}\!\left(t\right) + \cos^{2}\!\left(t\right) = 1
\]
The particular solution is
\[
	y_{p }\!\left( t  \right) = -\cos\!\left(t \right) \int \sin\!\left(t \right) \, dt + \sin\!\left(t \right) \int \cos\!\left(t \right) \, dt = \cos^{2}\!\left(t\right) + \sin^{2}\!\left(t\right) = 1
\]
If we impose initial conditions \( y \!\left( 0 \right) = y'\!\left( 0 \right) = 0 \), then we have
\begin{align*}
	y \!\left( t  \right) & = -\cos\!\left(t \right) \int_{0}^{t } \sin\!\left(t \right)  \, dt + \sin\!\left(t \right) \int_{0}^{t } \cos\!\left(t \right)  \, dt \\
			      & = \cos^{2}\!\left(t\right) - \cos\!\left(t \right) + \sin^{2}\!\left(t\right) \\
			      & = 1 - \cos^{2}\!\left(t\right) 
\end{align*}

% QUESTION 27
\qs{2.6.27}{Solve \( y''_{s } + 3y'_{s } + 2y_{s } = 1 \) for the step response starting from the null solutions \( y_{1} = e^{-t } \) and \( y_{2} = e^{-2t} \).}

We have \( y'_{1} = -e^{-t } \) and \( y'_{2} = -2e^{-2t} \), with Wronskian
\[
	W \!\left( t  \right) = -2e^{-3t} + e^{-3t} = -e^{-3t} 
\]
The particular solution is
\[
	y_{p }\!\left( t  \right) = e^{-t } \int e^{t } \, dt - e^{-2t} \int e^{2t} \, dt = 1 - \frac{1}{2} = \frac{1}{2}
\]
If we impose initial conditions \( y \!\left( 0 \right) = y'\!\left( 0 \right) = 0 \), then we have
\begin{align*}
	y \!\left( t  \right) & = e^{-t } \int_{0}^{t } e^{t } \, dt - e^{-2t} \int_{0}^{t } e^{2t} \, dt   \\
	 & = \frac{1}{2} - e^{-t } + \frac{1}{2}e^{-2t} \\ 
\end{align*}

% QUESTION 28
\qs{2.6.28}{Solve \( Ay'' + Cy = \cos\!\left(\omega t \right) \) when \( A \omega^{2} = C  \) (the case of resonance). Example 4 suggests to substitute \( y = Mt \cos\!\left(\omega t \right) + Nt \sin\!\left(\omega t \right)  \). Find \( M  \) and \( N  \).}

We have 
\begin{align*}
	y' & = M \cos\!\left(\omega t \right) + N \sin\!\left(\omega t \right) + t \!\left( -M \omega \sin\!\left(\omega t \right) + N \omega \cos\!\left(\omega t \right)  \right) \\
	   & = \!\left( M + N \omega t  \right) \cos\!\left(\omega t \right) + \!\left( N - M \omega t  \right) \sin\!\left(\omega t \right) \\
       y'' & = N \omega \cos\!\left( \omega t \right) - \!\left( M \omega + N \omega^{2} t  \right) \sin\!\left(\omega t \right) \\
	   & \qquad - M \omega \sin\!\left( \omega t \right) + \!\left( N \omega - M \omega^{2} t  \right) \cos\!\left(\omega t \right) \\
	   & = \!\left( 2N \omega - M \omega^{2} t  \right) \cos\!\left(\omega t \right) - \!\left( 2M \omega + N \omega^{2} t  \right) \sin\!\left(\omega t \right) 
\end{align*}
Substituting and letting \( A = C / \omega^{2} \), we find
\begin{alignat*}{2}
	&& C \!\left[ \!\left( \frac{2N}{\omega } - Mt  \right)\cos\!\left(\omega t \right) - \!\left( \frac{2M}{\omega } + Nt  \right)\sin\!\left(\omega t \right)  \right] \quad &  \\
	&& \quad + Ct \!\left[ M \cos\!\left(\omega t \right) + N \sin\!\left(\omega t \right)  \right] & =  \\ 
	\implies && 2C \!\left[ \frac{N }{\omega } \cos\!\left(\omega t \right) - \frac{M }{\omega }\sin\!\left(\omega t \right)  \right] & = \cos\!\left(\omega t \right) 
\end{alignat*}
We can conclude that \( M = 0 \) and \( N = \omega / 2C \), and our particular solution is
\[
	y_{p } \!\left( t  \right) = \frac{\omega}{2C} t \sin\!\left(\omega t \right)  
\]

% QUESTION 29
\qs{2.6.29}{Put \( g \!\left( t  \right) \) into the great formulas (17)-(18) to see the equations above them.}

(17): We define \( g \!\left( t  \right) \) as 
\[
	g \!\left( t  \right) = \frac{e^{s_{1}t } - e^{s_{2}t }}{s_{1} - s_{2}} 
\]
Substituting gives us:

\textbf{Particular solution, constant coefficients}
\begin{align*}
	y_{p } \!\left( t  \right) & = \int_{0}^{t } \!\left( \frac{e^{s_{1}\!\left( t - T  \right)} - e^{s_{2}\!\left( t - T  \right)}}{s_{1} - s_{2}} \right)f \!\left( T  \right) \, dT  \\
	 & = \frac{e^{s_{1}t }}{s_{1} - s_{2}} \int_{0}^{t } e^{s_{1}T } f \!\left( T  \right) \, dT - \frac{e^{s_{2}t }}{s_{1} - s_{2}} \int_{0}^{t } e^{s_{2}T } f \!\left( T  \right) \, dT   \\ 
\end{align*}
(18): We define \( g \!\left( t  \right) \) as 
\[
	g \!\left( t  \right) = te^{st } 
\]
Substituting gives us:

Particular solution \( y_{p } \), null solutions \( e^{st } \), \( te^{st } \)
\begin{align*}
	y_{p }\!\left( t  \right) & = \int_{0}^{t } \!\left( t - T  \right)e^{s \!\left( t - T  \right)} f \!\left( T  \right) \, dT \\
	 & = \int_{0}^{t } te^{s \!\left( t - T  \right)} f \!\left( T  \right) \, dT - \int_{0}^{t } Te^{s \!\left( t - T  \right)} f \!\left( T  \right) \, dT \\ 
	 & = te^{st} \int_{0}^{t } e^{-sT } f \!\left( T  \right) \, dT - e^{st } \int_{0}^{t } Te^{-sT } f \!\left( T  \right) \, dT  
\end{align*}

\newpage
\subsection{\textsf{2.7 - Laplace Transforms \texorpdfstring{\( Y \!\left( s  \right) \) and \( F \!\left( s \right) \)}{Y(s) and F(s)}}}

\end{document}
