\documentclass[a4paper]{article}

\def\nbook {Differential Equations and Linear Algebra}
\def\nbookshort {DeqLA}

\input{headerdeqla}

\begin{document}
\maketitle

\newpage
\tableofcontents

\newpage
\section{\textsf{2 - Second Order Equations}}

\subsection{\textsf{2.1 - Second Derivatives in Science and Engineering}}

% QUESTION 1
\qs{2.1.1}{Find a cosine and a sine that solve \(d^{2}y / dt^{2} = -9y\). This is a second order equation so we expect \textit{two constants} \(C\) and \(D\) (from integrating twice):
\[
	\textbf{Simple harmonic motion} \quad y \!\left( t \right) = C \cos\left( \omega t \right) + D \sin\left( \omega t \right) 
\]

What is \(\omega\)? If the system starts from rest (this means \(dy/dt = 0\) at \(t = 0\)), which constant \(C\) or \(D\) will be zero?
}

Differentiating \(y \!\left( t \right) \) twice, we get a \(\omega^{2}\) term. Making the necessary substitutions, we can see that \(\omega^{2} = 9\), implying \(\omega = 3\). Thus we have \(y = \sin\left( 3t \right)\) and \(y = \cos\left( 3t \right)\). The constants \(C\) and \(D\) are determined by the initial conditions. Assuming the system starts at rest, we must have

\[
	\frac{d y}{d t} = - 3C \sin\left( 3t \right) + 3D \cos\left( 3t \right) = 0
\]
which implies that \(dy / dt_{t = 0} = 3D = 0\), thus \(D = 0\).

\vspace{12pt}

% QUESTION 2
\qs{2.1.2}{In Problem 1, which \(C\) and \(D\) will give the starting values \(y \!\left( 0 \right) = 0\) and \(y' \!\left( 0 \right) = 1\)?}

We have \(y \!\left( 0 \right) = C = 0\) and \(y' \!\left( 0 \right) = 3D = 1\), or \(D = 1/3\).

\newpage

% QUESTION 3
\qs{2.1.3}{Draw Figure 2.3 to show simple harmonic motion \(y = A \cos\left( \omega t - \alpha  \right)\) with phases \(\alpha = \pi / 3\) and \(\alpha = -\pi /2\).}

\begin{center}
\includegraphics[width=1.0\textwidth]{../chap2/sec2.1/chap2sec2.1ex3.eps}
\end{center}

% QUESTION 4
\qs{2.1.4}{Suppose the circle in Figure 2.4 has radius 3 and circular frequency \(f = 60\) Hertz. If the moving point starts at the angle \(-45^{\circ}\), find its \(x\)-coordinate \(A \cos\left( \omega t - \alpha  \right)\). The phase lag is \(\alpha = 45^{\circ}\). When does the point first hit the \(x\)-axis?}

The circular motion of the point is expressed by the sinusoidal
\[
	3 \cos\left( 120 \pi t - \frac{\pi }{4} \right)
\]
Note that since the frequency is \(f = 60\) Hertz, the angular frequency is \(2 \pi \cdot 60 = 120 \pi \) radians s\(^{-1}\). The point hits the \(x\)-axis when the argument of the cosine is zero, namely at
\[
	120 \pi t - \frac{\pi }{4} = 0 \quad \implies \quad t = \frac{1}{480} \text{s}
\]

% QUESTION 5
\qs{2.1.5}{If you drive at 60 miles per hour on a circular track with radius \(R = 3\) miles, what is the time \(T\) for one complete circuit? Your circular frequency is \(f = \) \underline{\qquad} and your angular frequency is \(\omega = \) \underline{\qquad} (with what units?). The period is \(T\).}

Using dimensional analysis, the period is
\[
	T = \frac{1 \text{ hr}}{60 \text{ mi}} \!\left( 3 \text{ mi} \right) = \frac{1}{20} \text{ hr}
\]
The circular frequency is
\[
	f = \frac{60 \text{ mi}}{\text{ hr}} \frac{1 \text{ hr}}{3600 \text{ s}} \frac{1 \text{ cycle}}{3 \text{ mi}} = \frac{1}{180} \text{ s}^{-1}
\]
with angular frequency
\[
	\omega = 2 \pi f = \frac{\pi }{90} \text{ rad s}^{-1}
\]

% QUESTION 6
\qs{2.1.6}{The total energy \(E\) in the oscillating spring-mass system is
\begin{alignat*}{2}
	E & = \textbf{kinetic} \text{ energy in mass} + \textbf{potential} \text{ energy in spring} \\
	  & = \frac{m}{2} \!\left( \frac{d y}{d t}  \right)^{2} + \frac{k}{2}y^{2}
\end{alignat*}
Compute \(E\) when \(y = C \cos\left( \omega t \right) + D \sin\left( \omega t \right)\). The energy is constant!}

Given \(y \!\left( t \right) \), we have first time-derivative
\[
	\frac{d y}{d t} = - \omega C \sin\left( \omega t \right) + \omega D \cos\left( \omega t \right)
\]
squaring gives us
\[
	\!\left( \frac{d y}{d t}  \right)^{2} = \omega^{2} \!\left( C^{2} \sin^{2}\left( \omega t \right) - 2 CD \sin\left( \omega t \right) \cos\left( \omega t \right) + D^{2}\cos^{2}\left( \omega t \right) \right)
\]
Lastly, the \(y^{2}\) term is
\[
	y^{2} = C^{2} \cos^{2}\left( \omega t \right) + 2CD \sin\left( \omega t \right)\cos\left( \omega t \right) + D^{2}\sin^{2}\left( \omega t \right)
\]
Combining our ingredients, with \(\omega = \sqrt{k / m}\), observe that energy \(E\) reduces to a constant:
\begin{alignat*}{2}
	E & = \frac{m}{2} \!\left( \frac{d y}{d t}  \right)^{2} + \frac{k}{2} y^{2} \\
	  & = \frac{m}{2} \!\left( \frac{k}{m} \right) \!\left( C^{2}\sin^{2}\left( \omega t \right) - 2CD \sin\left( \omega t \right) \cos\left( \omega t \right) + D^{2} \cos^{2}\left( \omega t \right)\right)  \\ 
	  & \qquad + \frac{k}{2} \!\left( C^{2} \cos^{2}\left( \omega t \right) + 2CD \sin\left( \omega t \right)\cos\left( \omega t \right) + D^{2}\sin^{2}\left( \omega t \right) \right) \\
	  & = C^{2} + D^{2}
\end{alignat*}

\newpage

% QUESTION 7
\qs{2.1.7}{Another way to show that the total energy \(E\) is constant:

Multiply \(\bm{my'' + ky = 0}\) by \(\bm{y'}\). Then integrate \(my'y''\) and \(kyy'\).}

Take the first term and integrate:
\[
	m \int y'y'' \, dt
\]
By integration by parts, let
\[
	u = \frac{d y}{d t}, \quad \frac{d u}{d t} = \frac{d^{2}y}{d t^{2}} \quad \implies \quad du = \frac{d^{2}y}{d t^{2}} dt 
\]
Then we can rewrite the first term as
\[
	m \int u \, du = \frac{m}{2}u^{2} + C = \frac{m}{2} \!\left( y' \right)^{2} + C 
\]
as for the second term:
\[
	k \int yy' \, dt = k \int y \frac{d y}{d t}  \, dt = k \int y \, dy = \frac{k}{2}y^{2} + C 
\]
Summing the two pieces (sans the constants) restores the original energy function:
\[
	E = \frac{m}{2} \!\left( \frac{d y}{d t}  \right)^{2} + \frac{k}{2}y^{2}
\]
and since the derivative of a constant is zero, it must be the case that \(E\) is a constant.

\vspace{12pt}

% QUESTION 8
\qs{2.1.8}{A \textbf{forced oscillation} has another term in the equation and \(A \cos\left( \omega t \right)\) in the solution:
\[
	\frac{d^{2}y}{dt^{2}} + 4y = F \cos\left( \omega t \right) \quad \text{has} \quad y = C \cos\left( 2 t \right) + D \sin\left( 2 t \right) + A \cos\left( \omega t \right)
\]
\begin{enumerate}[label=(\alph*)]
	\item Substitute \(y\) into the equation to see how \(C\) and \(D\) disappear (they give \(y_{n}\). Find the forced amplitude \(A\) in the particular solution \(y_{p} = A \cos\left( \omega t \right)\).
	\item In case \(\omega = 2\) (forcing frequency \(=\) natural frequency), what answer does your formula give for \(A\)? The solution formula for \(y\) breaks down in this case.
\end{enumerate}}

The second time-derivative is
\[
	\frac{d^{2}y}{dt^{2}} = -4C \cos\left( 2t \right) - 4D \sin\left( 2t \right) - \omega^{2}A \cos\left( \omega t \right)
\]
\begin{enumerate}[label=(\alph*)]
	\item We have
		\[
			\frac{d^{2}y}{dt^{2}} + 4y = \!\left( 4 - \omega^{2} \right) A \cos\left( \omega t \right) = F \cos\left( \omega t \right)
		\]
	implying \(A = \dfrac{F}{4 - \omega^{2}}\).
	\item When \(\omega = 2\), \(A\) is undefined. 
\end{enumerate}

% QUESTION 9
\qs{2.1.9}{Following Problem 8, write down the complete solution \(y_{n} + y_{p}\) to the equation
\[
	m \frac{d^{2}y}{dt^{2}} + ky = F \cos\left( \omega t \right) \quad \text{with} \quad \omega \neq \omega_{n} = \sqrt{k / m} \quad \text{(no resonance)}
\]
The answer \(y\) has free constants \(C\) and \(D\) to match \(y \!\left( 0 \right) \) and \(y'\!\left( 0 \right) \) (\(A\) \textit{is fixed} by \(F\).}

Per Problem 8, we have solution
\[
	y = \underbrace{C \cos\left( \sqrt{\frac{k}{m}} t \right) + D \sin\left( \sqrt{\frac{k}{m}} t \right)}_{y_{n}} + \underbrace{\frac{F}{k - m \omega^{2}} \cos\left( \omega t \right)}_{y_{p}}
\]

All this involves is dividing the equation through by \(m\), understanding that the angular frequency of the sinusoidals in the null solution is the square root of \(k / m\), and making the changes to our formula for \(A\) accordingly.

\vspace{12pt}

% QUESTION 10
\qs{2.1.10}{Suppose Newton's Law \(F = ma\) has the force \(F\) in the \textit{same} direction as \(a\):
\[
	my'' = + ky \quad \text{including} \quad y'' = 4y
\]
Find two possible choices of \(s\) in the exponential solutions \(y = e^{st}\). The solution is not sinusoidal and \(s\) is real and the oscillations are gone. Now \(y\) is unstable.}

Substituting in \(y\), we find
\[
	ms^{2}e^{st} = ke^{st}
\]
This forces
\[
	s = \pm \sqrt{\frac{k}{m}}
\]
\newpage

% QUESTION 11
\qs{2.1.11}{Here is a \textit{fourth} order equation: \(d^{4}y / dt^{4} = 16y\). Find \textit{four} values of \(s\) that give exponential solutions \(y = e^{st}\). You could expect four initial conditions on \(y\): \(y \!\left( 0 \right) \) is given along with what three other conditions?}

Equivalently, we find the four complex roots of \(16\):
\[
	s^{4} = 16
\]
which are \(s = \pm 2, \pm 2i\).

\vspace{12pt}

% QUESTION 12
\qs{2.1.12}{To find a particular solution to \(y'' + 9y = e^{ct}\), I would look for a multiple \(y_{p}\!\left( t \right) = Ye^{ct}\) of the forcing function. What is that number \(Y\)? When does your formula give \(Y = \infty\)? (Resonance needs a new formula for \(Y\).)}

Let \(y_{p} = Ye^{ct}\). Substituting, we find
\[
	Yc^{2} e^{ct} + 9Ye^{ct} = e^{ct}
\]
Solving for \(Y\) yields \(Y = \dfrac{1}{c^{2} + 9}\). When \(c \rightarrow \pm3\), we have resonance, and \(Y \rightarrow \infty\).

\vspace{12pt}

% QUESTION 13
\qs{2.1.13}{In a particular solution \(y = Ae^{i \omega t}\) to \(y'' + 9y = e^{i \omega t}\), what is the amplitude \(A\)? The formula blows up when the forcing frequency \(\omega =\) what natural frequency?}

Substituting, we derive
\[
	-\omega^{2}A e^{i \omega t} + 9A e^{i \omega t} = e^{i \omega t}
\]
which gives us \(A = \dfrac{1}{9 - \omega^{2}}\). Resonance is when \(\omega = 3\).

\vspace{12pt}

\newpage
% QUESTION 14
\qs{2.1.14}{Equation (10) says that the tangent of the phase angle is \(\tan\left( \alpha  \right) = y' \!\left( 0 \right) / \omega y \!\left( 0 \right) \). First, check that \(\tan\left( \alpha  \right)\) is dimensionless when \(y\) is in meters and time is in seconds. Next, if that ratio is \(\tan\left( \alpha  \right) = 1\), should you choose \(\alpha = \pi / 4\) or \(\alpha = 5 \pi / 4\)? Answer:
\[
	\text{Separately you want } R \cos \left( \alpha  \right) = y \!\left( 0 \right)  \text{ and } R \sin \left( \alpha  \right) = y' \!\left( 0 \right) / \omega 
\]
If those right hand sides are positive, choose the angle \(\alpha \) between \(0\) and \(\pi / 2\). 

If those right hand sides are negative, add \(\pi \) and choose \(\alpha = 5 \pi / 4\).

\vspace{12pt}

\textit{Question}: If \(y \!\left( 0 \right) > 0\) and \(y' \!\left( 0 \right) < 0\), does \(\alpha \) fall between \(\pi / 2\) and \(\pi \) or between \(3 \pi / 2\) and \(2 \pi \)? If you plot the vector from \( \!\left( 0, 0 \right) \) to \(\!\left( y \!\left( 0 \right) , y' \!\left( 0 \right) / \omega  \right) \), its angle is \(\alpha \).}

As \(y \!\left( 0 \right) > 0\) and \(y'\!\left( 0 \right) < 0\) requires positive cosine and negative sine, \(\alpha \) falls between \(3 \pi / 2\) and \(2 \pi \).

\vspace{12pt}

% QUESTION 15
\qs{2.1.15}{Find a point on the sine curve in Figure 2.1 where \(y > 0\) but \(v = y' < 0\) and also \(a = y'' < 0\). The curve is sloping down and bending down.

\vspace{12pt}

Find a point where \(y < 0\) but \(y' > 0\) and \(y'' > 0\). The point is below the \(x\)-axis but the curve is sloping \underline{\qquad} and bending \underline{\qquad}.}

One area corresponding to the first set of conditions is \(\pi / 2 < t < \pi \). As for the second, we have \(3 \pi / 2 < t < 2 \pi \).

\vspace{12pt}

% QUESTION 16
\qs{2.1.16}{
\begin{enumerate}[label=(\alph*)]
	\item Solve \(y'' + 100y = 0\) starting from \(y \!\left( 0 \right) = 1\) and \(y' \!\left( 0 \right) = 10\). (\( \textbf{This is } \bm{y_{n}} \).)
	\item Solve \(y'' + 100y = \cos\left( \omega t \right)\) with \(y \!\left( 0 \right) = 0\) and \(y' \!\left( 0 \right) = 0\). (\( \textbf{This can be }\bm{y_{p}} \).)
\end{enumerate}}

\begin{enumerate}[label=(\alph*)]
	\item Let \(y = c_{1} \cos\left( 10 t \right) + c_{2} \sin\left( 10 t \right)\). Then \(y \!\left( 0 \right) = c_{1} = 1\) and \(y' \!\left( 0 \right) = 10 c_{2} = 10\) implies \(c_{2} = 1\). Then the null solution is
	\[
		y_{n} = \cos\left( 10 t \right) + \sin\left( 10 t \right)
	\]
\item Let \(y_{p} = R \cos\left( \omega t \right)\). Substitute to find
	\[
		-\omega^{2} R \cos\left( \omega t \right) + 100 R \cos\left( \omega t \right) = \cos\left( \omega t \right)
	\]
	Isolate \(R\) to derive
	\[
		R = \frac{1}{100 - \omega^{2}}
	\]
	The solution to this set of initial conditions requires \(y \!\left( 0 \right) = 0 \) and \(y' \!\left( 0 \right) = 0\). Begin with
	\[
		y \!\left( t \right) = c_{1} \cos\left( 10 t \right) + c_{2} \sin\left( 10 t \right) + \frac{1}{100 - \omega^{2}} \cos\left( \omega t \right)	
	\]
	From the first condition, we have \(c_{1} = -\dfrac{1}{100 - \omega^{2}}\). From the second, we have \(c_{2} = 0\). The full solution is
	\[
		y \!\left( t \right) = \frac{1}{100 - \omega^{2}} \!\left( \cos\left( \omega t \right) - \cos\left( 10t \right) \right) 
	\]
\end{enumerate}

% QUESTION 17
\qs{2.1.17}{Find a particular solution \(y_{p} = R \cos\left( \omega t - \alpha  \right)\) to \(y'' + 100y = \cos\left( \omega t \right) - \sin\left( \omega t \right)\).}

Substituting \(y_{p}\) gives us

\begin{alignat*}{2}
	& -\omega^{2} R \cos\left( \omega t - \alpha  \right) + 100 R \cos\left( \omega t - \alpha  \right) \\
		 & \qquad = \!\left( 100R - \omega^{2}R \right) \!\left[ \cos\left( \omega t \right)\cos\left( \alpha  \right) + \sin\left( \omega t \right)\sin\left( \alpha  \right) \right]  \\
\end{alignat*}

This implies that
\begin{alignat*}{2}
	R \cos\left( \alpha  \right) \!\left( 100 - \omega^{2} \right) & = 1 \\
	R \sin\left( \alpha  \right) \!\left( 100 - \omega^{2} \right) & = -1 \\ 
\end{alignat*}
enabling us to conclude that \(\alpha = 7 \pi / 4\), and amplitude
\[
	R = \frac{\sqrt{2}}{100 - \omega^{2}}
\]
Ergo, the particular solution is
\[
	y_{p} = \frac{\sqrt{2}}{100 - \omega^{2}} \cos\left( \omega t - \frac{7 \pi }{4} \right)
\]

% QUESTION 18
\qs{2.1.18}{Simple harmonic motion also comes from a linear pendulum (like a grandfather clock). At time \(t\), the height is \(A \cos\left( \omega t \right)\). What is the frequency \(\omega \) if the pendulum comes back to the start after 1 second? The period does not depend on the amplitude (a large clock or a small metronome or the movement in a watch can all have \(T = 1\)).}

The angular frequency is \(2 \pi \cdot f = 2 \pi \) radians s\(^{-1}\).

\vspace{12pt}

% QUESTION 19
\qs{2.1.19}{If the phase lag is \(\alpha \), what is the time lag in graphing \(\cos\left( \omega t - \alpha \right)\)?}

Put differently, we want to find the value of \(t'\) such that we are able to restore \(\omega t\) as the cosine's argument. If we have
\[
	t' = t + \alpha / \omega 
\]
then we get
\[
	\cos\left( \omega t' - \alpha  \right) = \cos\left( \omega \!\left( t + \alpha / \omega  \right) - \alpha  \right) = \cos\left( \omega t \right)
\]
Thus the time lag term is \(\alpha / \omega \).

\vspace{12pt}

% QUESTION 20
\qs{2.1.20}{What is the response \(y \!\left( t \right) \) to a delayed impulse if \(my'' + ky = \delta \!\left( t - T \right) \)?}

The full solution will be a factor of the step function, given by:
\[
	y \!\left( t \right) = \int^{t - T}_{0} \frac{\sin\left( \omega_{n} \!\left( t - T - s \right)  \right)}{m \omega_{n}} \delta \!\left( s \right)  \, ds = \frac{\sin\left( \omega_{n} \!\left( t - T \right)  \right)}{m \omega_{n}} H \!\left( t - T \right) 
\]
Intuitively, when \(t \le T\), the right-hand side vanishes. No impulse is imparted, and thus there is no response. But when we are at time \(t \ge T\) -- after the threshold -- the response kicks in.
\vspace{12pt}

% QUESTION 21
\qs{2.1.21}{(Good challenge) Show that \(\displaystyle{y = \int^{t}_{0} g \!\left( t - s \right) f \!\left( s \right)  \, ds }\) has \(my'' + ky = f \!\left( t \right) \).

\begin{enumerate}[label=\arabic*.]
	\item Why is \(\displaystyle{y' = \int^{t}_{0} g'\!\left( t - s \right) f \!\left( s \right)  \, ds + g \!\left( 0 \right) f \!\left( t \right)  }\)? Notice the two \(t\)'s in \(y\).
	\item Using \(g \!\left( 0 \right) = 0\), explain why \(\displaystyle{y'' = \int^{t}_{0} g''\!\left( t - s \right) f \!\left( s \right)  \, ds + g'\!\left( 0 \right) f \!\left( t \right)  }\).
	\item Now use \(g'\!\left( 0 \right) = 1 / m\) and \(mg'' + kg = 0\) to confirm \(my'' + ky = f \!\left( t \right) \).
\end{enumerate}}

Use the Leibniz integral rule:
\[
	\frac{d }{d t} \!\left( \int^{t}_{0} g \!\left( t - s \right) f \!\left( s \right)  \, ds \right) = g\!\left( 0 \right) f \!\left( t \right) + \int^{t}_{0} \frac{\partial }{\partial t} g \!\left( t - s \right) f \!\left( s \right)  \, ds 
\]

\begin{enumerate}[label=(\arabic*)]
	\item By above, applying the partial derivative inside the second term yields the desired result.
	\item One more application of the Leibniz rule gives us
	\[
		y'' = \int^{t}_{0} g''\!\left( t - s \right) f \!\left( s \right)  \, ds + g \!\left( 0 \right) f'\!\left( t \right) + g'\!\left( 0 \right) f \!\left( t \right) 
	\]
	With the premise \(g \!\left( 0 \right) = 0\), the second term vanishes and gives us the expected result.
	\item Derive
	\[
		my'' + ky = \int^{t}_{0} \!\left[ mg''\!\left( t - s \right) + kg \!\left( t - s \right)  \right] f \!\left( s \right)  \, ds + f \!\left( t \right) = f \!\left( t \right)   \\
	\]
	where the last equality follows by appealing to the nullity of \(g \!\left( t \right) \).
\end{enumerate}

% QUESTION 22
\qs{2.1.22}{With \(f = 1\) (direct current has \(\omega = 0\)) verify that \(my'' + ky = 1\) for this \(y\):
\[
	\textbf{Step response}
\]
\[
	y \!\left( t \right) = \int^{t}_{0} \frac{\sin\left( \omega_{n} \!\left( t - s \right)  \right)}{m \omega_{n}} \cdot 1 \, ds = y_{p} + y_{n} = \pmb{\frac{1}{k} - \frac{1}{k} \cos\left( \omega_{n}t \right)}
\]}

We have second derivative
\[
	y''\!\left( t \right) = \frac{\omega_{n}^{2}}{k}\cos\left( \omega_{n}t \right)	
\]
Since \(\omega_{n} = \sqrt{k / m}\), \(my''\!\left( t \right) = \cos\left( \omega_{n}t \right)\). Ergo, we have
\[
	my'' + ky = \cos\left( \omega_{n}t \right) + k \!\left[ \frac{1}{k} - \frac{1}{k}\cos\left( \omega_{n}t \right) \right] = 1
\]

% QUESTION 23
\qs{2.1.23}{(Recommended) For the equation \(d^{2}y / dt^{2} = 0\) find the null solution. Then for \(d^{2}g / dt^{2} = \delta \!\left( t \right) \) find the fundamental solution (start the null solution with \(g \!\left( 0 \right) = 0\) and \(g'\!\left( 0 \right) = 1\)). For \(y'' = f \!\left( t \right) \) find the particular solution using formula (16).}

Integrating twice, the null solution is
\[
	y \!\left( t \right) = C_{1} t + C_{2}
\]
To find the fundamental solution \(g \!\left( t \right) \), imposing the initial conditions \(g \!\left( 0 \right) = 0\) and \(g'\!\left( 0 \right) = 1\) forces \(C_{1} = 1\) and \(C_{2} = 0\), thus
\[
	g \!\left( t \right) = t
\]
Lastly, if \(y'' = f \!\left( t \right) \), then
\[
	y_{p}\!\left( t \right) = \int^{t}_{0} \!\left( t - s \right) f \!\left( s \right)  \, ds 
\]
% QUESTION 24
\qs{2.1.24}{For the equation \(d^{2}y/ dt^{2} = e^{i \omega t}\) find a particular solution \(y = Y \!\left( \omega  \right) e^{i \omega t}\). Then \(Y \!\left( \omega  \right) \) is the frequency response. Note the ``resonance'' when \(\omega = 0\) with the null solution \(y_{n} = 1\).}

One particular solution is
\[
	y \!\left( t \right) = -\frac{e^{i \omega t}}{\omega^{2}}
\]
meaning that \(Y \!\left( \omega  \right) = - 1 / \omega^{2}\). When \(\omega = 0\), we have resonance, and this particular solution breaks down.

\vspace{12pt}

% QUESTION 25
\qs{2.1.25}{Find a particular solution \(Ye^{i \omega t}\) to \(my'' - ky = e^{i \omega t}\). The equation has \(-ky\) instead of \(ky\). What is the frequency response \(Y \!\left( \omega  \right) \)? For which \(\omega \) is \(Y\) infinite?}

We have
\[
	\frac{d^{2} }{d t^{2}} \!\left( Y e^{i \omega t} \right) = - Y \omega^{2} e^{i \omega t}
\]
Substituting and canceling the \(e^{i \omega t}\) terms, we get
\[
	-mY \omega^{2} - kY = 1
\]
Implying
\[
	Y \!\left( \omega  \right) = -\frac{1}{m \omega^{2} + k}
\]
If we have
\[
	\omega = i \sqrt{\frac{k}{m}}
\]
then \(Y \!\left( \omega  \right) \) diverges to infinity -- meaning all real frequencies will not lead to this!

\newpage

\subsection{\textsf{2.2 - Key Facts About Complex Numbers}}

% QUESTION 1
\qs{2.2.1}{Mark the numbers \(s_{1} = 2 + i\) and \(s_{2} = 1 - 2i\) as points in the complex plane. (The plane has a real axis and an imaginary axis.) Then mark the sum \(s_{1} + s_{2}\) and the difference \(s_{1} - s_{2}\).}

Arithmetic involving complex numbers simply is vector superposition:

\begin{center}
\includegraphics[width=1.0\textwidth]{../chap2/sec2.2/chap2sec2.2ex1.eps}
\end{center}

\vspace{12pt}

% QUESTION 2
\qs{2.2.2}{Multiply \(s_{1} = 2 + i\) times \(s_{2} = 1 - 2i\). Check absolute values: \(\abs{s_{1}} \abs{s_{2}} = \abs{s_{1}s_{2}}\).}

We have
\[
	\!\left( 2 + i \right) \!\left( 1 - 2i \right) = 2 - 4i + i + 2 = 4 - 3i
\]
with \(\abs{s_{1}} = \abs{s_{2}} = \sqrt{5}\), \(\abs{s_{1}s_{2}} = \sqrt{16 + 9} = 5\), ascertaining \(\abs{s_{1}}\abs{s_{2}} = \abs{s_{1}s_{2}}\).

\vspace{12pt}

% QUESTION 3
\qs{2.2.3}{Find the real and imaginary parts of \(1 / \!\left( 2 + i \right) \). 

Multiply by \(\!\left( 2 - i \right) / \!\left( 2 - i \right) \):
\[
	\frac{1}{2 + i} \frac{2 - i}{2 - i} = \frac{2 - i}{\abs{2 + i}^{2}} = \quad ?
\]
}

The denominator is \(5\), so we have
\[
	\underbrace{\frac{2}{5}}_{\text{real}} - \underbrace{\frac{i}{5}}_{\text{imaginary}}
\]

% QUESTION 4
\qs{2.2.4}{\textit{Triple angles} \quad Multiply equation (10) by another \(e^{i \theta } = \cos\left( \theta  \right) + i \sin\left( \theta  \right)\) to find formulas for \(\cos\left( 3 \theta  \right)\) and \(\sin\left( 3 \theta  \right)\).}

Derive
\begin{alignat*}{2}
	\!\left( \cos\left( \theta  \right) + i \sin\left( \theta  \right) \right)^{3} & = \!\left( \cos^{2}\left( \theta  \right) - \sin^{2}\left( \theta  \right) + 2i \cos\left( \theta  \right) \sin\left( \theta  \right) \right) \!\left( \cos\left( \theta  \right) + i \sin\left( \theta  \right) \right)  \\
	 & = \cos^{3}\left( \theta  \right) - \sin^{2}\left( \theta  \right) \cos\left( \theta  \right) + 2i \cos^{2}\left( \theta  \right) \sin\left( \theta  \right) \\ 
	 & \qquad + i \cos^{2}\left( \theta  \right) \sin\left( \theta  \right) - i \sin^{3}\left( \theta  \right) - 2 \cos\left( \theta  \right) \sin^{2}\left( \theta  \right) \\
	 & = \underbrace{\cos^{3}\left( \theta  \right) - 3 \sin^{2}\left( \theta  \right) \cos\left( \theta  \right) }_{\text{real}} + i \underbrace{\!\left[ 3 \cos^{2}\left( \theta  \right) \sin\left( \theta  \right) - \sin^{3}\left( \theta  \right) \right] }_{\text{imaginary}}
\end{alignat*}

% QUESTION 5
\qs{2.2.5}{\textit{Addition formulas} \quad Multiply \(e^{i \theta } = \cos\left( \theta  \right) + i \sin\left( \theta  \right)\) times \(e^{i \phi } = \cos\left( \phi  \right) + i \sin\left( \phi  \right)\) to get \(e^{i \!\left( \theta + \phi  \right) }\). Its real part is \(\cos\left( \theta + \phi  \right) = \cos\left( \theta  \right) \cos\left( \phi  \right) - \sin\left( \theta  \right) \sin\left( \phi  \right)\). What is its imaginary part \(\sin\left( \theta + \phi  \right)\)?}

Derive
\begin{alignat*}{2}
	\!\left[ \cos\left( \theta  \right) + i \sin\left( \theta  \right) \right] \!\left[ \cos\left( \phi  \right) + i \sin\left( \phi  \right) \right]  & = \underbrace{\cos\left( \theta  \right)\cos\left( \phi  \right) - \sin\left( \theta  \right)\sin\left( \phi  \right)}_{\text{real}} \\
																			   & \qquad + i \underbrace{\!\left[ \cos\left( \theta  \right) \sin\left( \phi  \right) + \sin\left( \theta  \right) \cos\left( \phi  \right) \right]}_{\text{imaginary}}  \\
\end{alignat*}

% QUESTION 6
\qs{2.2.6}{Find the real part and the imaginary part of each cube root of 1. Show directly that the three roots add to zero, as equation (11) predicts.}

The cube roots, in polar form, are given by
\begin{alignat*}{2}
	e^{i 2 \pi / 3} & = -\frac{1}{2} + i \frac{\sqrt{3}}{2} \\
	e^{i 4 \pi / 3} & = -\frac{1}{2} - i \frac{\sqrt{3}}{2}\\
	 & \qquad 1 
\end{alignat*}

which vanish when summed.

\newpage

% QUESTION 7
\qs{2.2.7}{The three cube roots of 1 are \(z\) and \(z^{2}\) and 1, when \(z = e^{2 \pi i / 3}\). What are the three cube roots of 8 and the three cube roots of \(i\)? (The angle for \(i\) is \(90^{\circ}\) or \(\pi / 2\), so the angle for one of its cube roots will be \underline{\qquad}. The roots are spaced by \(120^{\circ}\).}

In polar form, we can express 8 as \(8e^{2 \pi i}\). Let \(z = e^{2 \pi i / 3}\). Its cube roots are then \(2z\), \(2z^{2}\), and \(2\). 

For \(i\), we have polar form \(e^{\pi i / 2}\). Then its cube roots are \(e^{\pi i / 6}\), \(e^{5 \pi i / 6}\), and \(e^{3 \pi i / 2}\). 

Argue by vector superposition that the cube roots in all cases sum to zero.

\vspace{12pt}

% QUESTION 8
\qs{2.2.8}{
	\begin{enumerate}[label=(\alph*)]
	\item The number \(i\) is equal to \(e^{\pi i / 2}\). Then its \(i^{\text{th}}\) power \(i^{i}\) comes out equal to a real number, using the fact that \(\!\left( e^{s} \right)^{t} = e^{st}\). What is that real number \(i^{i}\)?
	\item \(e^{i \pi / 2}\) is also equal to \(e^{5 \pi i / 2}\). Increasing the angle by \(2 \pi \) does not change \(e^{i \theta }\) - it comes around a full circle and back to \(i\). Then \(i^{i}\) has another real value \(\!\left( e^{5 \pi i / 2} \right)^{i} = e^{-5 \pi / 2}\). What are all the possible values of \(i^{i}\)?
\end{enumerate}}

\begin{enumerate}[label=(\alph*)]
	\item We have \(\!\left( e^{\pi i / 2} \right)^{i} = e^{-\pi / 2}\).
	\item All possible values of \(i^{i}\) are \(e^{\!\left( - \pi \pm 4n \pi  \right) / 2}\).
\end{enumerate}

% QUESTION 9
\qs{2.2.9}{The numbers \(s = 3 + i\) and \(\overline{s} = 3 - i \) are complex conjugates. Find their sum \(s + \overline{s} = -B\) and their product \(\!\left( s \right) \!\left( \overline{s} \right) = C\). Then show that \(s^{2} + Bs + C = 0\) and also \(\overline{s}^{2} + B \overline{s} + C = 0\). Those numbers \(s\) and \(\overline{s}\) are the two roots of the quadratic equation \(x^{2} + Bx + C = 0\).}

The sum is \(s + \overline{s} = 6\), so \(B = -6\). Their product is \(s \overline{s} = 10 = C\). Then we have
\[
	s^{2} + Bs + C = s^{2} - \!\left( s + \overline{s} \right) s + s \overline{s} = 0
\]
\[
	\overline{s}^{2} + B \overline{s} + C = \overline{s}^{2} - \!\left( s + \overline{s} \right) \overline{s} + s \overline{s} = 0
\]

% QUESTION 10
\qs{2.2.10}{The numbers \(s = a + i \omega \) and \(\overline{s} = a - i \omega \) are complex conjugates. Find their sum \(s + \overline{s} = -B\) and their product \(\!\left( s \right) \!\left( \overline{s} \right) = C\). Then show that \(s^{2} + Bs + C = 0\). The two solutions of \(x^{2} + Bx + C = 0\) are \(s\) and \(\overline{s}\).}

The sum is \(s + \overline{s} = 2a\), so \(B = -2a\). Their product is \(s \overline{s} = a^{2} + \omega^{2}\). By the same argument as the previous problem, the solutions of the given polynomial are \(s\) and \(\overline{s}\).

\vspace{12pt}

% QUESTION 11
\qs{2.2.11}{\begin{enumerate}[label=(\alph*)]
	\item Find the numbers \(\!\left( 1 + i \right)^{4} \) and \(\!\left( 1 + i \right)^{8}\).
	\item Find the polar form \(re^{i \theta }\) of \(\!\left( 1 + i \sqrt{3} \right) / \!\left( \sqrt{3} + i \right) \).
\end{enumerate}}

\begin{enumerate}[label=(\alph*)]
	\item We have \(1 + i = \sqrt{2} e^{i \pi / 4}\). Then
	\[
		\!\left( \sqrt{2} e^{i \pi / 4} \right)^{4} = 4 e^{i \pi } = -4
	\]
	Raising to the eighth power, we simply square our previous answer to get \(\!\left( 1 + i \right)^{8} = 16\).
	\item The numerator is \(2e^{i \pi / 3}\) and the denominator \(2 e^{i \pi / 6}\). The quotient is \(e^{i \pi / 6}\).
\end{enumerate}

% QUESTION 12
\qs{2.2.12}{The number \(z = e^{2 \pi i / n}\) solves \(z^{n} = 1\). The number \(Z = e^{2 \pi i / 2n}\) solves \(Z^{2n} = 1\). How is \(z\) related to \(Z\)? (This plays a big part in the Fast Fourier Transform.)}

They are related by \(z = Z^{2}\).

\vspace{12pt}

% QUESTION 13
\qs{2.2.13}{\begin{enumerate}[label=(\alph*)]
	\item If you know \(e^{i \theta }\) and \(e^{-i \theta }\), how can you find \(\sin\left( \theta  \right)\)?
	\item Find all angles \(\theta \) with \(e^{i \theta } = -1\) and all angles \(\phi \) with \(e^{i \phi } = -i\).
\end{enumerate}}

\begin{enumerate}[label=(\alph*)]
	\item One way to write \(\sin\left( \theta  \right)\) is
	\[
		\frac{e^{i \theta } - e^{-i \theta }}{2} = \sin\left( \theta  \right)
	\]
\item We have \(\theta = \pi \pm 2n \pi \) (which eliminates the imaginary term), and \(\phi = - \frac{\pi }{2} \pm 2n \pi \). 
\end{enumerate}

% QUESTION 14
\qs{2.2.14}{Locate all these points on one complex plane:
\begin{enumerate}[label=(\alph*)]
	\item \(2 + i\)
	\item \( \!\left( 2 + i \right)^{2} \)
	\item \(\displaystyle{\frac{1}{2 + i}}\)
	\item \(\abs{2 + i}\)
\end{enumerate}}

% QUESETION 15
\qs{2.2.15}{Find the absolute values \(r = \abs{z}\) of these four numbers. If \(\theta \) is the angle for \(6 + 8i\), what are the angles for these four numbers?
\begin{enumerate}[label=(\alph*)]
	\item \(6 - 8i\)
	\item \(\!\left( 6 - 8i \right)^{2}\)
	\item \(\displaystyle{\frac{1}{6 - 8i}}\)
	\item \(8i + 6\)
\end{enumerate}}

\begin{enumerate}[label=(\alph*)]
	\item \(r = 10\), \(\phi = -\theta \), in polar form \(z = 10e^{-i \theta }\)
	\item \(r = 100\), \(\phi = -2 \theta \), in polar form \(z = 100e^{-2 \theta }\)
	\item \(r = 1 / 10\), \(\phi = \theta \), in polar form \(\displaystyle{z = \frac{1}{10} e^{i \theta }}\)
	\item \(r = 10\), \(\phi = \theta \), the original number is unchanged
\end{enumerate}

% QUESTION 16
\qs{2.2.16}{What are the real and imaginary parts of \(e^{a + i \pi }\) and \(e^{a + i \omega }\)?}

The real parts are \(-e^{a}\) and \(e^{a} \cos\left( \omega  \right)\) and the imaginary parts are \(0\) and \(e^{a} \sin\left( \omega  \right)\).

\vspace{12pt}

% QUESTION 17
\qs{2.2.17}{\begin{enumerate}[label=(\alph*)]
	\item If \(\abs{s} = 2\) and \(\abs{z} = 3\), what are the absolute values of \(sz\) and \(s / z\)?
	\item Find upper and lower bounds in \(L \le \abs{s + z} \le U\). When does \(\abs{s + z} = U\)?
\end{enumerate}}

\begin{enumerate}[label=(\alph*)]
	\item \(\abs{sz} = \abs{s}\abs{z} = 6\), \(\abs{s / z} = \abs{s}/\abs{z} = 2/3\)
	\item The lower bound is achieved when \(s\) and \(z\) are antiparallel, meaning the modulus of their vector sum is at best \(L = \abs{s + z} = 1\). The upper bound is reached when they are parallel, so \(U = \abs{s + z} = 5\).
\end{enumerate}

\newpage
% QUESTION 18
\qs{2.2.18}{
\begin{enumerate}[label=(\alph*)]
	\item Where is the product \(\!\left( \sin\left( \theta  \right) + i \cos\left( \theta  \right) \right) \!\left( \cos\left( \theta  \right) + i \sin\left( \theta  \right) \right) \) in the complex plane?
	\item Find the absolute value  \(\abs{S}\) and the polar angle \(\phi \) for \(S = \pmb{\sin\left( \theta  \right) + i \cos\left( \theta  \right)}\).
\end{enumerate}
This is my favorite problem, because \(S\) combines \(\cos\left( \theta  \right)\) and \(\sin\left( \theta  \right)\) in a new way. To find \(\phi \), you could plot \(S\) or add angles in the multiplication of part (a).}

\begin{enumerate}[label=(\alph*)]
	\item Note that \(\sin\left( \theta  \right) = \displaystyle{\cos\left( \frac{\pi }{2} - \theta  \right)}\) and \(\cos\left( \theta  \right) = \displaystyle{\sin\left( \frac{\pi }{2} - \theta  \right)}\). Then the first term is equal to \(\displaystyle{\cos\left( \frac{\pi }{2} - \theta  \right) + i \sin\left( \frac{\pi }{2} - \theta  \right) = e^{i \!\left( \pi / 2 - \theta  \right) }}\). Then the product is
	\[
		e^{i \!\left( \pi / 2 - \theta  \right) } e^{i \theta } = e^{i \pi / 2} = i
	\]
	\item \(\abs{S} = 1\) and \(\displaystyle{\phi = \frac{\pi }{2} - \theta }\).
\end{enumerate}

% QUESTION 19
\qs{2.2.19}{Draw the spirals \(e^{\!\left( 1 - i \right)t}\) and \(e^{\!\left( 2 - 2i \right)t}\). Do those follow the same curves? Do they go clockwise or anticlockwise? When the first one reaches the negative \(x\)-axis, what is the time \(T\)? What point has the second one reached at that time?}

% QUESTION 20
\qs{2.2.20}{The solution to \(d^{2}y / dt^{2} = -y\) is \(y = \cos\left( t \right)\) if the initial conditions are \(y \!\left( 0 \right) = \underline{\qquad}\) and \(y' \!\left( 0 \right) = \underline{\qquad}\). The solution is \(y = \sin\left( t \right)\) when \(y \!\left( 0 \right) = \underline{\qquad}\) and \(y'\!\left( 0 \right) = \underline{\qquad}\). Write each of those solutions in the form \(c_{1}e^{i t} + c_{2}e^{-i t}\), to see that real solutions can come from complex \(c_{1}\) and \(c_{2}\).}

For the case of \(y = \cos\left( t \right)\), the initial conditions are \(y \!\left( 0 \right) = 1\) and \(y'\!\left( 0 \right) = 0\). In the case of \(y  = \sin\left( t \right)\), we have \(y \!\left( 0 \right) = 0\) and \(y'\!\left( 0 \right) = 1\). In the given form, the initial conditions require

\underline{\(y = \cos\left( t \right)\)}
\begin{align*}
	y \!\left( 0 \right) & = c_{1} + c_{2} = 1 \\
	y' \!\left( 0 \right) & = i \!\left( c_{1} - c_{2} \right) = 0
\end{align*}
Implying \(c_{1} = c_{2} = 1/2\).

\underline{\( y = \sin\left( t \right) \)}
\begin{align*}
	y \!\left( 0 \right)  & = c_{1} + c_{2} = 0 \\
	y'\!\left( 0 \right)  & = i \!\left( c_{1} - c_{2} \right) = 1  
\end{align*}
Implying \(c_{1} = 1 / 2i\) and \(c_{2} = -1 / 2i\).

Thus we can write \(\cos\left( t \right)\) and \(\sin\left( t \right)\) as
\[
	\cos\left( t \right) = \frac{e^{i t} + e^{-i t}}{2}, \quad \sin\left( t \right) = \frac{e^{i t} - e^{-i t}}{2i}
\]

The general solution is given by
\[
	y \!\left( t \right) = C_{1} \cos\left( t \right) + C_{2} \sin\left( t \right)
\]
which we rewrite as
\begin{align*}
	y \!\left( t \right)  & = C_{1} \cos\left( t \right) + C_{2} \sin\left( t \right) \\
	 & = C_{1} \!\left[ \frac{e^{i t} + e^{-i t}}{2} \right] + C_{2} \!\left[ \frac{e^{i t} - e^{-i t}}{2i} \right]   \\ 
	 & = \!\left[ \frac{C_{1} - iC_{2}}{2} \right] e^{i t} + \!\left[ \frac{C_{1} + iC_{2}}{2} \right] e^{-i t}
\end{align*}

% QUESTION 21
\qs{2.2.21}{Suppose \(y \!\left( t \right) = e^{-t}e^{i t}\) solves \(y'' + By' + Cy = 0\). What are \(B\) and \(C\)? If this equation is solved by \(y = e^{3it}\), what are \(B\) and \(C\)?}

Note that \(y \!\left( t \right) = e^{-t}e^{i t} = e^{\!\left( -1 + i \right) t}\). Then we have
\[
	y'' + By' + Cy = \!\left( -1 + i \right)^{2} e^{\!\left( -1 + i \right) t} + B \!\left( -1 + i \right)  e^{\!\left( -1 + i \right) t} + Ce^{\!\left( -1 + i \right) t} = 0
\]
from which we derive
\[
	\!\left( -1 + i \right)^{2} + B \!\left( -1 + i \right) + C = 0
\]
Now wait -- this looks suspiciously familiar to the form seen in question 9:
\[
	s^{2} + Bs + C = 0
\]
Then we have that \(s + \overline{s} = -B\) and \(s \overline{s} = C\). Thus
\[
	s + \overline{s} = -2 = -B \qquad s \overline{s} = 2 = C
\]
Therefore, \(B = C = 2\), and the equation is \(y'' + 2y' + 2y = 0\).

In the case of \(y = e^{3it}\), we have polynomial
\[
	-9 + 3Bi + C = 0
\]
where it follows that \(s = 3i\), \(s + \overline{s} = 0 = -B\), and \(s \overline{s} = 9 = C\). The equation is \(y'' + 9y = 0\).

\newpage

% QUESTION 22
\qs{2.2.22}{From the multiplication \(e^{iA}e^{-iB} = e^{i \!\left( A - B \right) }\), find the ``subtraction formulas'' for \(\cos\left( A - B \right)\) and \(\sin\left( A - B \right)\).}

Derive
\begin{align*}
	e^{iA}e^{-iB} & = \!\left[ \cos\left( A \right) + i \sin\left( A \right) \right] \!\left[ \cos\left( B \right) - i \sin\left( B \right) \right]  \\
	 & =  \!\left[ \cos\left( A \right)\cos\left( B \right) + \sin\left( A \right) \sin\left( B \right) \right] + i \!\left[ \sin\left( A \right) \cos\left( B \right) - \cos\left( A \right) \sin\left( B \right) \right] \\ 
	 & = \cos\left( A - B \right) + i \sin\left( A - B \right) \\
	 & = e^{i \!\left( A - B \right) }
\end{align*}

% QUESTION 23
\qs{2.2.23}{
\begin{enumerate}[label=(\alph*)]
	\item If \(r\) and \(R\) are the absolute values of \(s\) and \(S\), show that \(rR\) is the absolute value of \(sS\). (Hint: Polar form!)
	\item If \(\overline{s}\) and \(\overline{S}\) are the complex conjugates of \(s\) and \(S\), show that \(\overline{s}\overline{S}\) is the complex conjugate of \(sS\). (Polar form!)
\end{enumerate}}

\begin{enumerate}[label=(\alph*)]
	\item Write \(s = re^{i \theta }\) and \(S = Re^{i \phi }\). Then \(sS = rR e^{i \!\left( \theta + \phi  \right) }\). Then \(rR = \abs{sS}\).
	\item Using the aforementioned definitions of \(s\) and \(S\), we have 
	\[
		\overline{sS} = rRe^{-i \!\left( \theta + \phi  \right)} = \!\left( re^{-i \theta } \right) \!\left( Re^{-i \phi } \right) = \overline{s}\overline{S}
	\]
\end{enumerate}

% QUESTION 24
\qs{2.2.24}{Suppose a complex number \(s\) solves a real equation \(s^{3} + As^{2} + Bs + C = 0\) (with \(A\), \(B\), \(C\) real). Why does the complex conjugate \(\overline{s}\) also solve this equation? ``\textit{Complex solutions to real equations come in conjugate pairs \(s\) and \(\overline{s}\)}.''}



\end{document}
