%%%%%%%%%%%%%%%%%%%%
% AUTHOR AND HEADER
% define \nbook
%%%%%%%%%%%%%%%%%%%%

\makeatletter
\ifx \nauthor\undefined
	\def\nauthor{David A. Lee}
\else
\fi

\author{\nbook \\ \small Gilbert Strang \\ \small Solutions by \nauthor}
\date{}

%%%%%%%%%%%%%%%%%%%%
% PACKAGES
%%%%%%%%%%%%%%%%%%%%

\usepackage{alltt} 
\usepackage{amsfonts}
\usepackage{amsmath}
\usepackage{amssymb}
\usepackage{amsthm}
\usepackage{bm}
\usepackage{booktabs}
\usepackage{caption}
\usepackage{centernot}
\usepackage{enumitem}
\usepackage{fancyhdr}
\usepackage[T1]{fontenc}
\usepackage{graphicx}
\usepackage{mathdots}
\usepackage{mathtools}
\usepackage{microtype}
\usepackage{multirow}
\usepackage{pdflscape}
\usepackage{pgfplots}
\pgfplotsset{compat=1.18}
\usepackage{siunitx}
\usepackage{slashed}
\usepackage{tabularx}
\usepackage{tikz}
\usepackage{titletoc}
\usepackage{tkz-euclide}
\usepackage[normalem]{ulem}
\usepackage{url}
\usepackage[all]{xy}
\usepackage{imakeidx}

% fonts
% use \textsf to change to Computer Modern Bright
\renewcommand{\sfdefault}{cmbr} % keeps default font Computer Modern Roman

% makeindex for table of contents
\makeindex[intoc, title=Index]
\indexsetup{othercode={\lhead{ Index }}}

% table of contents remove numbering
\usepackage{tocloft} % for coloring subsections -- note that sections have standard document hyperlink coloring, i.e. color = doc
\setcounter{secnumdepth}{0} % remove numbering
\renewcommand{\cftsubsecfont}{\hypersetup{linkcolor=subsect!90}} % subsection coloring
\renewcommand{\contentsname}{\textsf{Table of Contents}}

% pagestyle
\pagestyle{fancyplain}

% header
\lhead{	\nouppercase{\leftmark} }
\ifx \nextra \undefined
  \rhead{
    \ifnum\thepage=1
    \else
      \nbookshort \  | \nauthor
    \fi}
\else
  \rhead{
    \ifnum\thepage=1
    \else
      \nbookshort (\nextra)
    \fi}
\fi

% proof environment

\newcommand{\prooffont}{\scshape}
\usepackage{xpatch}
\tracingpatches
\xpatchcmd{\proof}{\itshape}{\prooffont}{}{}

% Python environment (for typesetting Python code)

% Default fixed font does not support bold face
\DeclareFixedFont{\ttb}{T1}{txtt}{bx}{n}{8} % for bold
\DeclareFixedFont{\ttm}{T1}{txtt}{m}{n}{8}  % for normal

% Custom colors
\usepackage{color}
\definecolor{deepblue}{rgb}{0,0,0.5}
\definecolor{deepred}{rgb}{0.6,0,0}
\definecolor{deepgreen}{rgb}{0,0.5,0}

\usepackage{listings}

% Python style for highlighting
\newcommand\pythonstyle{\lstset{
language=Python,
basicstyle=\ttm,
morekeywords={self},              % Add keywords here
keywordstyle=\ttb\color{deepblue},
emph={MyClass,__init__},          % Custom highlighting
emphstyle=\ttb\color{deepred},    % Custom highlighting style
stringstyle=\color{deepgreen},
frame=tb,                         % Any extra options here
showstringspaces=false
}}


% Python environment
\lstnewenvironment{python}[1][]
{
\pythonstyle
\lstset{#1}
}
{}

% Python for external files
\newcommand\pythonexternal[2][]{{
\pythonstyle
\lstinputlisting[#1]{#2}}}

% Python for inline
\newcommand\pythoninline[1]{{\pythonstyle\lstinline!#1!}}

%%%%%%%%%%%%%
% FUNCTIONS %
%%%%%%%%%%%%%

\DeclarePairedDelimiter\abs{\lvert}{\rvert}%
\DeclarePairedDelimiter\norm{\lVert}{\rVert}%

%%%%%%%%%%%%%%%%%%%%
% DOCUMENT GEOMETRY
%%%%%%%%%%%%%%%%%%%%

\ifx \ntrim \undefined
\else
  \usepackage{geometry}
  \geometry{
	  papersize={379pt, 699pt},
	  textwidth=345pt,
	  left=17pt,
	  top=54pt,
	  right=17pt,
  }
\fi

% maketitle statement
\title{}
\ifx \nisofficial \undefined
\let\@real@maketitle\maketitle
\renewcommand{\maketitle}{\@real@maketitle\begin{center}\begin{minipage}[c]{0.9\textwidth}\centering\footnotesize All errors, typographical and substantive, and other offenses, are entirely my own.\end{minipage}\end{center}}
\else
\fi

%%%%%%%%%%%%%%%%%%%%
% THEOREMS
%%%%%%%%%%%%%%%%%%%%

%\theoremstyle{definition}
\newtheorem*{axiom}{Axiom}
\newtheorem*{claim}{Claim}
\newtheorem*{conjecture}{Conjecture}
\newtheorem*{cor}{Corollary}
%\newtheorem*{dfn}{Definition}
\newtheorem*{eg}{Example}
\newtheorem*{ex}{Exercise}
%\newtheorem*{lemma}{Lemma}
\newtheorem*{prop}{Proposition}
%\newtheorem*{thm}{Theorem}
\newtheorem*{remark}{Remark}
\newtheorem*{warning}{Warning}

%%%%%%%%%%%%%%%%%%%%
% FORMATTING AESTHETICS
%%%%%%%%%%%%%%%%%%%%

% itemize bullets
\renewcommand{\labelitemi}{-}
\renewcommand{\labelitemii}{$\circ$}
\renewcommand{\labelenumi}{(\roman{*})}

% new page section
%\let\stdsection\section
%\renewcommand\section{\newpage\stdsection}

% new page subsection
%\let\stdsubsection\subsection
%\renewcommand\subsection{\newpage\stdsubsection}

%%%%%%%%%%%%%%%%%%%%
% \mathbb commands
%%%%%%%%%%%%%%%%%%%%

\newcommand{\C}{\mathbb{C}}
\newcommand{\N}{\mathbb{N}}
\newcommand{\Q}{\mathbb{Q}}
\newcommand{\R}{\mathbb{R}}
\newcommand{\Z}{\mathbb{Z}}

%%%%%%%%%%%%%%%%%%%%
% Complex Numbers
%%%%%%%%%%%%%%%%%%%%

\DeclareMathOperator{\Real}{Re}
\DeclareMathOperator{\Imag}{Im}

%%%%%%%%%%%%%%%%%%%%
% Probability Operators
%%%%%%%%%%%%%%%%%%%%

\DeclareMathOperator{\Bernoulli}{Bernoulli}
\DeclareMathOperator{\betaD}{Beta}
\DeclareMathOperator{\bias}{\textsf{bias}}
\DeclareMathOperator{\binomial}{Binomial}
\DeclareMathOperator{\corr}{Corr}
\DeclareMathOperator{\cov}{\textsf{Cov}}
\DeclareMathOperator{\Exp}{Exp}
\DeclareMathOperator{\gammaD}{Gamma}
\DeclareMathOperator{\mse}{\textsf{MSE}}
\DeclareMathOperator{\multinomial}{Multinomial}
\DeclareMathOperator{\Poisson}{Poisson}
\DeclareMathOperator{\sd}{sd}
\DeclareMathOperator{\se}{\textsf{se}}
\DeclareMathOperator{\Uniform}{Uniform}
\newcommand{\Prob}{\mathbb{P}}
\newcommand{\var}{\mathbb{V}}
\newcommand{\E}{\mathbb{E}}

%%%%%%%%%%%%%%%%%%%%
% TCOLORBOX CODE FROM SeniorMars
%%%%%%%%%%%%%%%%%%%%

%\usepackage[varbb]{newpxmath} changes math font to newpxmath
\usepackage{xfrac}
\usepackage[makeroom]{cancel}
\usepackage{mathtools}
\usepackage{bookmark}
\usepackage{enumitem}
\usepackage{hyperref,theoremref}
\hypersetup{
	pdftitle={Assignment},
	colorlinks=true, linkcolor=doc!90,
	bookmarksnumbered=true,
	bookmarksopen=true
}
\usepackage[most,many,breakable]{tcolorbox}
\usepackage{xcolor}
\usepackage{varwidth}
\usepackage{varwidth}
\usepackage{etoolbox}
%\usepackage{authblk}
\usepackage{nameref}
\usepackage{multicol,array}
\usepackage{tikz-cd}
\usepackage[ruled,vlined,linesnumbered]{algorithm2e}
\usepackage{comment} % enables the use of multi-line comments (\ifx \fi)
\usepackage{import}
\usepackage{xifthen}
\usepackage{pdfpages}
\usepackage{transparent}

\newcommand\mycommfont[1]{\footnotesize\ttfamily\textcolor{blue}{#1}}
\SetCommentSty{mycommfont}
\newcommand{\incfig}[1]{%
    \def\svgwidth{\columnwidth}
    \import{./figures/}{#1.pdf_tex}
}

\usepackage{tikzsymbols}

% colors -- change them as you may!

\definecolor{myg}{RGB}{56, 140, 70}
\definecolor{myb}{RGB}{45, 111, 177}
\definecolor{myr}{RGB}{199, 68, 64}
\definecolor{mytheorembg}{HTML}{F2F2F9}
\definecolor{mytheoremfr}{HTML}{00007B}
\definecolor{mylenmabg}{HTML}{FFFAF8}
\definecolor{mylenmafr}{HTML}{983b0f}
\definecolor{mypropbg}{HTML}{f2fbfc}
\definecolor{mypropfr}{HTML}{191971}
\definecolor{myexamplebg}{HTML}{F2FBF8}
\definecolor{myexamplefr}{HTML}{88D6D1}
\definecolor{myexampleti}{HTML}{2A7F7F}
\definecolor{mydefinitbg}{HTML}{E5E5FF}
\definecolor{mydefinitfr}{HTML}{3F3FA3}
\definecolor{notesgreen}{RGB}{0,162,0}
\definecolor{myp}{RGB}{197, 92, 212}
\definecolor{mygr}{HTML}{880808}
\definecolor{myred}{RGB}{127,0,0}
\definecolor{myyellow}{RGB}{169,121,69}
\definecolor{myexercisebg}{HTML}{F2FBF8}
\definecolor{myexercisefg}{HTML}{88D6D1}
\definecolor{doc}{RGB}{0,0,255}
\definecolor{subsect}{RGB}{255,0,0}
\definecolor{question}{HTML}{3FB69E}

%================================
% THEOREM BOX
%================================

%\providecommand\theoremnumber{}
\tcbuselibrary{theorems,skins,hooks}
\newtcbtheorem{Theorem}{Theorem}
{%
	enhanced,
	breakable,
	colback = mytheorembg,
	frame hidden,
	boxrule = 0sp,
	borderline west = {2pt}{0pt}{mytheoremfr},
	sharp corners,
	detach title,
	before upper = \tcbtitle\par\smallskip,
	coltitle = mytheoremfr,
	fonttitle = \bfseries\sffamily,
	description font = \mdseries,
	separator sign none,
	segmentation style={solid, mytheoremfr},
}
{th}


%================================
% Question BOX
%================================

\makeatletter
\newtcbtheorem{question}{Question}{enhanced,
	breakable,
	colback=white,
	colframe=question,
	attach boxed title to top left={yshift*=-\tcboxedtitleheight},
	fonttitle=\bfseries\sffamily,
	title={#2},
	boxed title size=title,
	boxed title style={%
			sharp corners,
			rounded corners=northwest,
			colback=tcbcolframe,
			boxrule=0pt,
		},
	underlay boxed title={%
			\path[fill=tcbcolframe] (title.south west)--(title.south east)
			to[out=0, in=180] ([xshift=5mm]title.east)--
			(title.center-|frame.east)
			[rounded corners=\kvtcb@arc] |-
			(frame.north) -| cycle;
		},
	#1
}{def}
\makeatother

%================================
% PROPERTIES BOX
%================================

\tcbuselibrary{theorems,skins,hooks}
\newtcbtheorem{Property}{Property}
{%
	enhanced,
	breakable,
	colback = mytheorembg,
	frame hidden,
	boxrule = 0sp,
	borderline west = {2pt}{0pt}{mytheoremfr},
	sharp corners,
	detach title,
	before upper = \tcbtitle\par\smallskip,
	coltitle = mytheoremfr,
	fonttitle = \bfseries\sffamily,
	description font = \mdseries,
	separator sign none,
	segmentation style={solid, mytheoremfr},
}
{th}

%================================
% LEMMA
%================================

\tcbuselibrary{theorems,skins,hooks}
\newtcbtheorem[number within=section]{Lemma}{Lemma}
{%
	enhanced,
	breakable,
	colback = mylenmabg,
	frame hidden,
	boxrule = 0sp,
	borderline west = {2pt}{0pt}{mylenmafr},
	sharp corners,
	detach title,
	before upper = \tcbtitle\par\smallskip,
	coltitle = mylenmafr,
	fonttitle = \bfseries\sffamily,
	description font = \mdseries,
	separator sign none,
	segmentation style={solid, mylenmafr},
}
{th}


% Commands for special boxes
\newcommand{\thm}[2]{\begin{Theorem*}{#1}{}#2\end{Theorem*}}
\newcommand{\qs}[2]{\begin{question*}{#1}{}#2\end{question*}}
\newcommand{\prp}[2]{\begin{Property*}{#1}{}#2\end{Property*}}
\newcommand{\lemma}[2]{\begin{Lemma*}{#1}{}#2\end{Lemma*}}
